Verifique en cada caso si las funciones que se definen son homomorfismos:

a) $f(Z* , * ) -> (Z *, *) f(x)=x^2$
$\forall a \in Z*, Para todo b \in Z* : f(a*b) = f(a)*f(b)$
Si el primero es una suma el otro tambien debe ser suma es decir
f(a+b) = f(a)+f(b)

El primer producto hace referencia a (Z*, *) y el segundo producto hace referencia a (Z*,*)


a > f(a)=a^2
b > f(b)=b^2

f(a*b)=f(a)*f(b)

(a*b)^2 = a^2 * b^2
a^2*b^2 = a^2*b^2

Por la propiedad distributiva de la potenciacion.


c) (Z, + ) -> (Z,+)

x -> f(x) = x+3


\forall a \in Z, \forall b in Z, f(a+b)=f(a)+f(b)
f(a+b) = f(a)+f(b)

Escribimos las imagenes
a+b+3 = (a+3)+(b+3)
a+b+3 = a+b+6

No se cumple, en efecto no es un homomorfismo, ya que a+b+3 \notequal a+b+6

Arrancamos el practico 5 o hacemos uno mas del 4 y el 3?


Ejercicio Nº 9:
En Z se definen las operaciones siguientes:

a o b = a+b+1
a \perp b = a+b+a*b

Determine que estructura tiene (,o,\perp)


Lleva mucho tiempo hacer este ejercicio.
Este ejercicio era un ejercicio de parcial de anios anteriores.


Guía de ejercicios Prácticos Nº 4:
######

Ejercicio Nº 1: En una division entera, el divisor es 12 y el cociente es 5. ¿Cuál es el dividendo? 
si a) el resto es el mayor posible. b) el resto es el menor posible.

1) divisor = 12
cociente = 5 
calcular el dividendo = ¿?

el dividendo será igual a 
a = b*q+r, 0 <= r < |b|


a |_b__
r  q

a |__b__
r  5


a) r = 11
a = 12*5+11
a = 71

b) r = 0
a = 12*5+0
a = 60 



2) Teniendo en cuenta el algoritmo de la division en Z, determine q y r si:

a) a = 4231, b = 7
 4231 |__7__
-42   604
-----
   31
  -28
-----
    3


 4231 |_7___
-42    -604
-----
  -31
 --28
-----
   -3
El resto nunca puede ser negativo.


 4231 |_7___
-42    -605
-----
  -31
 --35
-----
    4

Se puede verificar haciendo 7*604+3=4231
y se puede verificar haciendolo -4231 = 7*(-605)+4
Algoritmo euclides para la division.

Ahora tenemos que hacer el c)
El resto nunca puede ser negativo.
c) a = 132, b = -89

 132 |_-89__
 -89   -1
----
  43
  //fin.

Ejercicio 3: Para a igual a 12 y b = 18 , halle D(a) y D(b) y determine por inspeccion al maximo comun divisor entre a y b.

Tengo que encontrar todos los divisores de 12 todos los divisores de 18 encontrar cuales son iguales y cual es el maximo.
Puedo hacer por factoreo 

\textbf{Metodo por factoreo}
12|2
 6|2
 3|3
 1


12 = 2^2 * 3


18|2
 9|3
 3|3
 1

18 = 2*3^2


MCD(12,18) = 2*3 = 6

-----------------
Otra manera de hacerlo:

$D(a) = { x \in Z / x|a }$
$D(12) = {+-1,+-2,+-3,+-4,+-6,+-12}
$D(18) = {+-1,+-2,+-3,+-6,+-9,+-18}


$MCD(12,18) = Max {D(12) n\cap D(18)} = 6$


 18 |__12__
-12  1
---
  6




Sucesivas divisiones me van a permitir encontrar
La primera division que tenemos que hacer l dvidendo el divisor

El divisor para ser el nuevo dividendo.
Luego el resto va a pasar a ser el divisor de la siguiente division.

 12 |__6__
-12  2
---
  0

Como el último resto me dió 0 considero el último resto no nulo de las dos divisiones y ese será el MCD.
En este caso fue sencillo hicimos 0 divisiones.


El último resto no nulo será el maximo comun divisor.


Ejercicio 4: Halle el mcd(a,b) = d para:
a) (84,45)

a) MCD(84,45)
 84 |_45__ -- Si aca despejo el resto me queda: 39=84-45*1
-45  1
---
 39

 45 |__39__ -- Si aca despejo el resto me queda: 6=45-39*1
-39  1
----
  6

Tomamos nuevamente divisor y resto.
  39 |_6__ -- Si aca despejo el resto me queda: 3=39-6*6
 -36  6
----
   3 --- Este es el maximo comun divisor (el ultimo resto no nulo)

 6 |_3__
-6  2
 0

Sean a y b \in Z, no simultaneamente nulos, de N tal que d|a^b|b.
Entonces, son equivalentes:
i. d = (a,b)
ii. \exists u,v \in Z tq. d=u.a+v.b
iii. Si s \in Z es tal que s|a^s|b entonces s|d.

\textbf{Teorema de Bezout} :
del algoritmo de euclides sabemos = a=bq+r -- r=a-b(divisor)q(cociente).


La clave es no resolver las operaciones , sino que resolver algunas y dejar otras expresadas por producto.

queremos llegar a d=u.a+v.b

Del ejercicio anterior llegamos a 
3=39-6*6
3=39-6(45-39*1)
3=39-6*45 +6*39
3=7*39-6*45

d = ua+vb
d = u*84 + v*45
Todos los terminos que tengan 84 y 45 no los voy a operar.


3 = 7*(84-45*1)-6*45
3 = 7*84 - 7*45 - 6*45
3 = 7*84 - 13*45
3 = \underbrace{7}_{=u}*84 + \underbrace{(-13)}_{=v}*45


Son coprimos cuando el maximo comun divisor entre ellos es 1.

Ahora hacer el ejercicio Nº 4 b) (-84,45)

-84 |_45__
-90  -2 -- Maximo comun divisor
---  ------- Despejo: 6=(-84)-45*(-2)
  6

 45 |__6__
-42  7   
--- ------- Despejo: 3=45-6*7
  3


 6 |_3__
-6  2
--- 
 0

Maximo comun divisor (-84,45) = 3

3 = 45-6*7
3 = 45-[-84-45*(-2)]*7
3 = 45-7*(-84)-27*45
3 = 45-7*(-84)-14*45
3 = -13*45-7*(-84)
3 = \underbrace{(-7)}_{u}*\underbrace{(-84)}_{a} + \underbrace{(-13)}_{v}*\underbrace{45}_{b}


Ejercicio Nº7: Pruebe que 81 y 35 son coprimos y halle u,v \in Z/ u.81 + v.35 = d, siendo d=mcd(35,81)

26 es el parcial.
























