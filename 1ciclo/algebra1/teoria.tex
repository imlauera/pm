Fecha de parcial es la próxima semana.
Miercoles 26 segundo parcial
Miercoles 
Se toma estructura algebraicas y enteros.

12|2
 6|2
 3|3
 1


D(12) = {1,2,3,4,6,12}

mcd(a,b)=a.u+b.v


Teorema fundamental de la aritmética:
Todo numero entero no primo, salvo 1 y -1 es el producto de factores primos.
La descomposicion es unica.

Estos no son términos, son factores.
a = p_1^alfa * p_2^beta *** p_k*1


Congruencia modulo N.
Es otra forma de clasificar a los numeros enteros.
esta relacionado con el tema de la disivilidad.

Ejemplo: todos los numeros enteros que al dividirlos por 2 tienen resto  1.

4 |_3_
1

10 |_3_
1

16 |_3_
1


4==10==16

Si un numero a divide a la diferencia entre dos numeros b y c, entonces b y c se llaman congruentes y en caso contrario.... --- Esta en algebra 1 CONGRUENCIA WORD.

b === c (moda) 
b es congruente con 

a es congruente con b modulo n si solo si b divide a a -b

4 == 10 (3)
3|(10-4)



Decidir si 5^2013+7^2012 es divisible por 6.

5 == -1(6)
5² == 1(6)
(5²)^n == 1^n (6) \forall n \in N
(5^2n) == 1(6) \forall n \in N
7 == 1(6)
7^n==1(6) \forall n \in N


5^2013 + 7^2013 = 5^2012 * 5 +1 = 1*5 + 1 = 1*(-1)+1 = 0 mod 6


No nos vamos a ver la semana que viene, resolve el practico del aula virtual.
Dada esta interna esto forma un grupo, anillo.
Aplicacion de divisiones.
Calculo de maximo comun divisor.
Encontrar numeros de besou.
Verificar si dos numeros son congruentes.

