\documentclass{article}
\usepackage{lipsum}
\usepackage{hyperref}
\usepackage{geometry}
\geometry{margin=1cm}

\begin{document}
\begin{itemize}
\item{1.Entre las siguientes propiedades señalar cuáles son propiedades intensivas y cuáles son propiedades extensivas: sabor, volumen, densidad, olor, punto de ebullición, peso, dureza, calor de vaporización, superficie, punto de fusión.}
Propiedades intensivas:\\
sabor, olor, dureza\\
\\
Propiedades extensivas:\\
volumen, densidad, peso, punto de ebullición, calor de vaporización, superficie, punto de fusión\\

\item{2. Indicar si los siguientes sistemas son cerrados, abiertos o aislados.}
- Una lata de durazno en el supermercado.\\
- El planeta tierra\\
- Un termo para el mate cerrado\\
- Un plato de sopa\\
- Una cartuchera\\
\\
- Una lata de durazno en el supermercado. - Cerrado\\
- El planeta tierra - Aislado\\
- Un termo para el mate cerrado - Cerrado\\
- Un plato de sopa - Abierto\\
- Una cartuchera - Cerrado\\
\item{3.¿Qué criterio se emplea para diferenciar un sistema homogéneo de un sistema heterogéneo? }
Los sistemas se clasifican como homogéneos o heterogéneos en base al criterio de la fase.\\
\\
Un sistema es homogéneo si todas sus fases son iguales (sólida, líquida, gas). En este caso, el sistema tiene una sola fase.\\
\\
Por ejemplo:\\
• Un bloque de hielo a 0°C es un sistema homogéneo en fase sólida.\\
• Agua pura a 25°C es un sistema homogéneo en fase líquida.\\
• Helio puro a 0 K es un sistema homogéneo en fase gas.\\
\\
Un sistema es heterogéneo si contiene múltiples fases. Sus componentes pueden presentarse en dos o tres estados físicos diferentes.\\
\\
Por ejemplo:\\
• Una mezcla de agua (líquida) y hexano (líquido) es un sistema heterogéneo de dos fases líquidas.\\
• Un bloque de hielo con burbujas de aire es un sistema heterogéneo de fases sólida y gas.\\
• La atmosférica terrestre es un sistema heterogéneo de fases gas, líquida y sólida.\\
\\
En resumen, la diferencia radica en la uniformidad o no de las fases que lo componen. Los sistemas homogéneos tienen una sola fase, mientras que los sistemas heterogéneos están compuestos por múltiples fases.\\
\item{4. Indique si las afirmaciones siguientes son correctas. Proporcione una explicación o ejemplo para
justificar su respuesta. }
a) Un sistema de un solo componente debe ser homogéneo.\\
Correcto. Un sistema que contiene un solo componente, por definición tendrá una sola fase, por lo que será homogéneo. Por ejemplo, un bloque sólido de sal a temperatura ambiente.\\
\\
b) Un sistema de dos componentes líquidos siempre será homogéneo.\\
Incorrecto. Dos líquidos miscibles podrían formar una sola fase líquida (homogénea), pero dos líquidos inmiscibles darían lugar a un sistema heterogéneo de dos fases líquidas. Por ejemplo, agua y mercurio.\\
\\
c) Un sistema homogéneo debe estar formado por un solo componente.\\
Correcto. Por definición, un sistema homogéneo tiene una sola fase, por lo que debe contener un solo componente.\\
\\
d) Un sistema con dos componentes gaseosos debe ser homogéneo.\\
Incorrecto. Dos gases podrían ser miscible o inmiscible, en cuyo caso darían lugar a sistemas homogéneos o heterogéneos respectivamente. Por ejemplo, aire y helio son miscibles, pero oxígeno y flúor no.\\
\\
e) Varios componentes distintos deben formar un sistema heterogéneo.\\
correcto. Cuando un sistema contiene varios componentes diferentes, éstos seguramente darán lugar a múltiples fases, por lo que el sistema será heterogéneo. Por ejemplo, una mezcla de agua, alcohol y sal.\\
\\
En resumen, la homogeneidad de un sistema depende de si tiene una o múltiples fases, independientemente del número de componentes. Las miscelas de líquidos, gases o de líquido-gas pueden ser homogéneas o heterogéneas.\\

\item{5.}
a) Granito – Heterogéneo. 2 fases: sólida (silicatos) y líquida (agua intersticial). 1 componente principal (silicatos).\\
\\
b) Varios trozos de hielo en un vaso – Heterogéneo. 2 fases: sólida (hielo) y líquida (agua). 1 componente (H2O).\\
\\
c) cloruro de sodio y arena – Heterogéneo. 2 fases: sólida (arena) y líquida (NaCl). 2 componentes (NaCl y SiO2).\\
\\
d) Una suspensión de polvo de carbón en una solución de agua con sal – Heterogéneo. 3 fases: sólida (carbón), líquida (agua salada) y coloidal (polvo de carbón). 3 componentes (C, H2O y NaCl).\\
\\
e) Alcohol – Homogéneo. 1 fase líquida. 1 componente (C2H5OH).\\
\\
f) Una botella de soda recién abierta. Heterogéneo. 2 fases: líquida (agua, azúcar, colorantes) y gaseosa (CO2). 3 componentes principales (H2O, C12H22O11, CO2).\\
\\
Resumen:\\
Homogéneos: e (alcohol)\\
Heterogéneos: a, b, c, d, f\\
a, c, f: 2 fases / 2-3 componentes\\
b, d: 3 fases / 2-3 componentes\\
\\
La homogeneidad de un sistema depende de si tiene una o varias fases, y el número de componentes no es determinante. Espero esta explicación te ayude a identificar la homogeneidad de otros sistemas!\\
\item{6.}
Analizando los sistemas homogéneos:\\
\\
• Agua de mar - Solución acuosa. Contiene sales disueltas (cloruros de sodio, calcio, magnesio, etc.), por lo que no es agua pura.\\
\\
• Vino (sin borra) - Solución acuosa de azúcares, sales, antocianinas y otros compuestos procedentes de la uva. No es una sustancia pura.\\
\\
• Agua del rio - Contiene disueltas sales, sedimentos y otros contaminantes, por lo que no es agua pura. Es una solución acuosa.\\
\\
• Agua destilada - Esta agua se ha purificado mediante destilación, por lo que es agua pura (H2O). Es una sustancia pura.\\
\\
• Agua con tintura - Solución acuosa de colorantes. No es agua pura.\\
\\
• Hielo - A temperaturas inferiores a 0°C, el agua se solidifica formando cristales de hielo puro (H2O en estado sólido). Es una sustancia pura.\\
\\
En resumen:\\
Sustancias puras: Agua destilada, hielo\\
Soluciones: Agua de mar, vino, agua del rio, agua con tintura\\
\\
La presencia de otras sustancias disueltas en el agua es lo que determina si es una sustancia pura o una solución. La purificación por destilación es el método más efectivo para obtener el agua como una sustancia química pura (H2O).\\
\\
Espero haber aclarado bien la diferencia entre sustancias puras y soluciones. Dime si necesitas más detalles!\\
\item{7.}
Para determinar si un sistema agua-aceite se puede separar por decantación, hay que analizar las densidades relativas de ambos líquidos y su miscibilidad.\\
\\
• Las densidades relativas: El aceite tiene menor densidad que el agua, por lo que flotará sobre el agua al decantar. Esto favorece la separación por decantación.\\
\\
• La miscibilidad: El agua y el aceite son inmiscibles, es decir, no se disuelven mutuamente. Al mezclarlos, no se forma una solución homogénea. Esto también facilita su separación.\\
\\
Por lo tanto, si el agua y el aceite tienen densidades claramente diferentes y son inmiscibles, sí es posible separarlos por decantación. Cuando se deja reposar la mezcla, el aceite subirá a la superficie y se podrá extraer por separado.\\
\\
Algunas condiciones que favorecen la separación:\\
\\
• Usar agua a temperatura ambiente o inferior. A mayor temperatura, las densidades se aproximan más y la separación es más difícil.\\
\\
• Reducir la agitación de la mezcla. Cuanta más turbulencia, más tiempo tardará en separarse.\\
\\
• Dejar tiempo suficiente para la decantación. Se trata de un proceso lento, por lo que se requieren varias horas para separar completamente agua y aceite.\\
\\
• Utilizar embudos o sifones para extraer los líquidos separados sin volver a mezclarlos.\\
\\
• En algunos casos, repetir el proceso varias veces para obtener separaciones más completas.\\
\\
En resumen, sí es posible separar agua y aceite por decantación debido a su diferencia de densidades y a su inmiscibilidad. Sin embargo, es un proceso lento que requiere tiempo y condiciones adecuadas para llevarse a cabo.\\
\item{8.}
La vaporización es el cambio de estado de un líquido a gas. Existen dos formas de vaporización:\\
\\
Evaporación: Es la vaporización de un líquido a temperatura ambiente o inferior. Se trata de un proceso lento en el que las moléculas más volátiles del líquido pasan al estado gasoso.\\
Ejemplos: Evaporación del agua en un charco, evaporación del disolvente en la técnica de cristalización, etc.\\
\\
Ebullición: Es la vaporización de un líquido cuando alcanza su temperatura de ebullición. Se produce un brusco paso del líquido a gas debido a la agitación térmica.\\
Ejemplos: Ebullición del agua a 100°C, ebullición de un solvente orgánico, etc.\\
\\
Algunas diferencias entre evaporación y ebullición:\\
\\
• Velocidad: La ebullición es mucho más rápida que la evaporación. En la ebullición, el líquido pasa casi instantáneamente a gas, mientras que en la evaporación es un proceso gradual.\\
\\
• Agitación: La ebullición provoca una agitación visible de las moléculas en el líquido (burbujas). La evaporación no tiene associated agitación visible.\\
\\
• Temperatura: La ebullición se produce a la temperatura de ebullición del líquido. La evaporación ocurre a temperaturas inferiores.\\
\\
• Presión: La temperatura de ebullición de un líquido depende de la presión. Un disminución de presión provoca una diminución de la temperatura de ebullición. La evaporación no depende de la presión.\\
\\
• Purificación: La ebullición permite separar mezclas azeotrópicas y purificar sustancias. La evaporación solo produce una separación parcial de componentes.\\
\\
Espero haber aclarado bien la diferencia entre evaporación y ebullición. No dudes en preguntar si necesitas más detalles.\\
\item{9.}
La licuación de un gas consiste en su cambio de estado a líquido mediante disminución de temperatura y/o aumento de presión. Existen dos métodos principales para licuar un gas:\\
\\
Compresión: Al aumentar la presión del gas, las moléculas se acercan y los enlaces intermoleculares se fortalecen, pasando de gas a líquido. Este método se utiliza industrialmente para licuar gases como el oxígeno, nitrógeno, argón, etc. Se requieren elevadas presiones, del orden de varias atmósferas.\\
\\
Enfriamiento: Al disminuir la temperatura del gas por debajo de su punto crítico, las moléculas pierden energía cinética y se producen interacciones intermoleculares que conducen a la condensación del gas en líquido. Este método se emplea para licuar gases como el helio, hidrógeno, etc. Se requieren temperaturas muy bajas, próximas al cero absoluto.\\
\\
En la práctica, se suelen combinar presión y enfriamiento para facilitar la licuación. Algunas consideraciones del proceso:\\
\\
• Se necesitan ciclos de compresión-expansión para lograr las presiones requeridas de forma efectiva. Los compresores y turbinas permiten variar rápidamente la presión y minimizar las pérdidas de calor.\\
\\
• Se requieren intercambiadores de calor y refrigerantes potentes para enfriar el gas, como helio líquido, nitrógeno líquido, etc.\\
\\
• El gas licuado ocupa mucho menos volumen que en estado gasoso, por lo que su almacenamiento y transporte resulta más eficiente.\\
\\
• La licuación induce cambios en las propiedades del gas, como temperatura de ebullición, viscosidad, densidad, solubilidad, etc. El gas licuado suele tener propiedades diferentes al gas en condiciones normales.\\
\\
• La licuación es un proceso exotérmico, por lo que requiere eliminar el calor desprendido para que el gas se condense.\\
\\
• Los gases más fáciles de licuar son aquellos con momento dipolar y fuerzas intermoleculares más intensas. Por ello, los gases nobles son más difíciles de licuar que el hidrógeno o el nitrógeno, por ejemplo.\\
\\
En conclusión, la licuación de gases se basa en procesos físicos de compresión, expansión-compresión, y enfriamiento para alcanzar condiciones que permitan el cambio de fase gas-líquido de manera viable industrialmente. Es un proceso complejo que requiere equipos sofisticados para su puesta en práctica.\\
\item{10.}
a) El punto de ebullición de una sustancia es la temperatura a la que los vapores de ésta alcanzan la presión de saturación y la sustancia comienza a hervir. Se considera una propiedad intensiva porque depende solo de la sustancia y no de la cantidad de ésta.\\
\\
b)\\
• Agua: 100°C\\
• Mercurio: 356,73°C\\
• Helio: -452,2°F / -268,95°C\\
\\
c) Los metales en general tienen muy alto o nulo punto de ebullición.\\
\\
• La mayoría de los metales tienen puntos de ebullición muy elevados, entre 2000-3500°C, debido a las fuertes fuerzas atractivas entre sus átomos. Solo los metales alcalinos tienen puntos de ebullición menor que 1000°C.\\
\\
• El mercurio tiene un punto de ebullición relativamente bajo para un metal, de 356,73°C, debido a que forma enlaces metálicos débiles.\\
\\
• Los metales alcalinotérreos como calcio, estroncio y bario tienen puntos de ebullición muy altos, por encima de 1400°C, debido a sus dos electrones de valencia.\\
\\
• Los metales de transición normalmente tienen puntos de ebullición intermedios, entre 1000-2500°C, ya que presentan electrones desapareados que forman enlaces metal-metal débiles.\\
\\
• Los metales nobles como oro, plata y platino, tienen los puntos de ebullición más altos de todos los metales, por encima de 2000°C, debido a las fuertes interacciones entre sus electrones de valencia.\\
\\
• El hierro y los demás metales de transición presentan solubidad de gases como el hidrógeno a temperaturas elevadas, lo que disminuye notablemente su punto de ebullición.\\
\\
• No todos los metales tienen un punto de ebullición bien definido. Por ejemplo, el mercurio presenta vaporización en una amplia zona de temperaturas, y metales como bismuto y antimonio se descomponen a alta temperatura sin llegar a hervir realmente.\\
\\
En conclusión, la mayoría de los metales tienen puntos de ebullición extremadamente altos o prácticamente nulos, debido a las intensas fuerzas interatómicas que los unen. Solo ciertos metales alcalinos o con enlaces más débiles presentan puntos de ebullición accesibles.\\
\\
Espero haber aclarado el significado y las particularidades del punto de ebullición de las sustancias, incluyendo los metales. No dudes en preguntar si precisas más detalles.\\
\item{11.}
Un sistema coloidal es un sistema heterogéneo formado por partículas muy finas, normalmente de tamaño nanométrico y las cuales permanecen dispersas en un medio continuado (líquido, gas o plasma) gracias a fuerzas electrostáticas u otras fuerzas. Algunas características importantes de los sistemas coloidales:\\
\\
• Las partículas poseen tamaños intermedios entre las moléculas y los coloides macroscópicos (0.1-1000 nm). Son lo suficientemente grandes para no formar moléculas, pero suficientemente pequeñas para permanecer en suspensión por largos períodos de tiempo.\\
\\
• Las partículas portan carga eléctrica, normalmente negativa o positiva, debido a la presencia de iones adsorbidos o de defectos límite en la superficie. Esta carga provoca la estabilización de la dispersión mediante fuerzas de repulsión electrostática.\\
\\
• La dispersión permanece estable y no sedimenta gracias al equilibrio entre fuerzas de separación (repulsión electrostática) y fuerzas de atracción (atracción van der Waals). Si las primeras prevalecen, la dispersión es estable.\\
\\
• La turbidez depende del tamaño medio y factor de dispersión de las partículas. Sistemas con partículas muy monodispersas producen dispersión de la luz más intensa (más turbios).\\
\\
• La dispersión puede estabilizarse mediante tensioactivos (formando emulsiones coloidales) o polímeros (formando micelas de polímeros). Estos aditivos rodean las partículas y provocan fuerzas repulsivas estéricas.\\
\\
• Presentan propiedades intermedias entre la sustancia molecular y el estado de agregación. Por ejemplo, su presión osmótica depende de la concentración.\\
\\
• Su composición puede ser orgánica (emulsiones proteicas), inorgánica (sol de plata) o mixta (sol de oro rojo).\\
\\
• Sirven como agentes de dispersión, catalizadores, vehículos para medicamentos, agentes formadores de imagen, entre otros usos.\\
\\
En resumen, los sistemas coloidales son dispersiones muy finas estables que poseen propiedades intermedias entre la sustancia pura y los estados agregados. Su estabilidad depende del equilibrio entre fuerzas de separación y fuerzas de atracción entre las partículas. Espero haber descrito con suficiente claridad este tipo de materiales. Puede preguntar si requiere más detalles.\\
\item{12.}
a) 12 in = 12 * 2.54 = 30.48 cm\\
No se puede convertir exactamente, se pierden decimales.\\
\\
b) 33 g = 33 * (g/cm3) = 0.033 cm3\\
Se supone densidad de 1 g/cm3.\\
\\
c) 760 mmHg = 760 * 133.322 = 101325 Pa\\
Factor de conversión entre unidades de presión.\\
\\
d) 5 g/cm3 = 5 kg/l\\
Unidades equivalentes, no hay conversión.\\
\\
e) 5 Ha = 5 * 10000 m2 = 50000 m2\\
1 Ha = 10000 m2.\\
\\
f) 250 ml = 250 * (cm3/l) = 0.25 cm3\\
Relación de volumen, 1 l = 1 dm3 = 1000 cm3.\\
\\
g) 25 cm = 25 * (Pa/cm) = 62500 Pa\\
Relación de presión hidrostática, 1 cm = 9.8 Pa.\\
\\
h) 340 m/s = 340 * 5/18 = 94.44 Km/h\\
Relación de velocidad, 5 min/km = 18 km/h.\\
\\
En los casos de no poder convertir exactamente se indica la razón, pérdida de decimales o no equivalencia directa entre unidades. Las relaciones de conversión se basan en los factores de conversión habituales entre las unidades del Sistema Internacional.\\
\\
\item{13.}
a) 0,003 l = 3 mL\\
mililitro\\
\\
b) 2300 g = 2.3 kg\\
kilogramo\\
\\
c) 0,0000057 g = 57 ug\\
microgramo\\
\\
d) 0,000000046 g = 46 ng\\
nanogramo\\
\\
e) 0,0028 l = 28 mL\\
mililitro\\
\\
Las magnitudes se expresan utilizando los prefijos adecuados del Sistema Internacional de Unidades para indicar las potencias de 10:\\
\\
m (mili) = 10-3\\
k (kilo) = 103\\
u (micro) = 10-6\\
n (nano) = 10-9\\
\\
por lo tanto:\\
0,003 l -> 3 mL (millilitro)\\
2300 g -> 2.3 kg (kilogramo)\\
0,0000057 g -> 57 ug (microgramo)\\
0,000000046 g -> 46 ng (nanogramo)\\
0,0028 l -> 28 mL (mililitro)\\
\\
\item{14.}
En la Tierra:\\
Aceleración gravitatoria de la Tierra = 9.78 m/s2\\
Peso (P) = Masa (m) x Aceleración (a)\\
= 3 kg x 9.78 m/s2\\
= 29.34 N\\
En Newton (N) y Kgf (Kilogram Force, unidad obsoleta pero se incluye):\\
P Tierra (N) = 29.34\\
P Tierra (Kgf) = 3\\
\\
En la Luna:\\
Aceleración gravitatoria de la Luna = 1.62 m/s2\\
Peso (P) = Masa (m) x Aceleración (a)\\
= 3 kg x 1.62 m/s2\\
= 4.86 N\\
En Newton (N) y Kgf (Kilogram Force, unidad obsoleta pero se incluye):\\
P Luna (N) = 4.86\\
P Luna (Kgf) = 0.5\\
\\
Por lo tanto:\\
En la Tierra: 29.34 N (3 Kgf)\\
En la Luna: 4.86 N (0.5 Kgf)\\
\\
La masa permanece constante (3 kg) pero el peso varía según la aceleración gravitatoria del cuerpo celeste.\\
En la Tierra es mayor (29.34 N) debido a su mayor aceleración gravitatoria (9.78 m/s2)\\
que en la Luna (1.62 m/s2) por lo que el peso es menor (4.86 N).\\
\item{15.}
Para convertir la densidad de mercurio de g/cm3 a Kg/l se utiliza la siguiente relación:\\
\\
1 g/cm3 = 1 kg/m3\\
\\
Por lo tanto, para 13.57 g/cm3 se tiene:\\
13.57 g/cm3 = 13.57 kg/m3\\
Como:\\
1 m3 = 1 l (unidad de volumen)\\
\\
Se deduce que:\\
13.57 kg/m3 = 13.57 Kg/l\\
\\
Por tanto, la densidad del mercurio expresada en Kg/l es:\\
13.57 Kg/l\\
\\
En resumen:\\
Densidad del mercurio (g/cm3) = 13.57\\
Densidad del mercurio (Kg/l) = 13.57\\
\\
Las unidades g/cm3 (gramos por centímetro cúbico) y Kg/l (kilogramos por litro) son equivalentes, simplemente expresan la misma magnitud (densidad) en diferentes sistemas de unidades.\\
La relación de conversión es directa, por lo que no hay ninguna aproximación.\\
\\
La densidad del mercurio, tanto en g/cm3 como en Kg/l es de 13.57.\\
\item{16.}
Para calcular el peso de una botella de 2,25 l llena de mercurio se requiere:\\
\\
La densidad de mercurio: 13.57 Kg/l (obtenida en el problema anterior)\\
El volumen de la botella: 2,25 l\\
La relación peso (P) = densidad (d) x volumen (V)\\
Sustituyendo los datos:\\
P = 13.57 Kg/l x 2,25 l\\
= 30.565 Kg\\
\\
Por tanto, una botella de 2,25 l llena de mercurio pesaría 30,565 Kg.\\
\\
Expresado de forma detalle:\\
\\
Densidad del mercurio: 13.57 Kg/l\\
\\
Volumen botella: 2,25 l\\
\\
Relación peso (P) = densidad (d) x volumen (V)\\
P = d x V\\
= 13.57 Kg/l x 2,25 l\\
= 30.565 Kg\\
\\
Peso botella + mercurio = 30,565 Kg\\
\\
Se supone la botella vacía tiene un peso despreciable.\\
El volumen es el indicado para la botella, independientemente de su forma.\\
La densidad se mantiene constante para el mercurio en las condiciones del problema.\\
No hay más incógnitas ni datos necesarios. El peso final de la botella llena de mercurio es 30,565 Kg.\\
\\
\item{17.}
Se puede inferir que la masa de ese gran meteorito no se determinó directamente mediante una balanza, dado que es demasiado grande.\\
\\
Probablemente se utilizaron métodos indirectos para estimar su masa aproximada en 350000 kg. A partir de la información adicional se pueden deducir algunas pistas:\\
\\
Tiene forma casi esférica y composición casi homogénea. Esto sugiere que se pudo determinar su volumen mediante mediciones geométricas, y asumir una densidad promedio para calcular la masa.\\
\\
Se encontró en Chaco, Argentina. Pudo catalogarse entre los meteoritos más grandes encontrados en el país. Esto permite asumir que es un meteorito de notables dimensiones, compatible con una masa cercana a los 350,000 kg.\\
\\
Su masa no pudo medirse con balanzas. Esto limita a utilizar métodos indirectos.\\
\\
A partir de estas pistas, se puede inferir que lo más probable para estimar su masa fue:\\
\\
Determinar su volumen mediante mediciones de sus dimensiones (longitud, diámetro, etc.) al asumir una forma esférica.\\
Conocer la densidad promedio de meteoritos de composición similar y asignarla a este meteorito.\\
Calcular la masa como:\\
Masa = Densidad x Volumen\\
\\
La densidad de meteoritos de tipo piedra suele estar en torno a 3,5-4,5 g/cm3.\\
Suponiendo 4 g/cm3, si el volumen estimado es de 87500 litros,\\
la masa resultante sería:\\
\\
Masa = 4 g/cm3 x 87500 l = 350000 kg\\
\\
Esto parece la metodología más razonable para estimar indrectamente la masa de un meteorito tan grande que no se puede pesar directamente.\\
\item{18.}
De acuerdo a las condiciones de la problemática, las únicas respuestas correctas son:\\
\\
c) su masa se mantiene igual\\
g) su densidad se mantiene constante.\\
\\
Las razones son:\\
\\
Al enviar un objeto a la Luna o cualquier cuerpo celeste no cambia su masa. La masa es una propiedad intrínseca de un objeto, no depende de fuerzas externas.\\
\\
Tampoco cambia su densidad. La densidad depende de la masa y el volumen de un objeto. Si la masa es constante y el volumen no cambia (es un objeto rígido), la densidad se mantiene también constante.\\
\\
El volumen, peso y otras propiedades no cambian:\\
** El volumen depende de las dimensiones del objeto, y éstas no varían al cambiar de cuerpo celeste.\\
** El peso depende de la masa y la aceleración gravitatoria. La masa no cambia, y en la Luna la aceleración disminuye, pero el peso se mantiene constante.\\
\\
** Otras opciones como que el volumen o peso disminuyen o aumentan son incorrectas.\\
\\
Por tanto, la única opción correcta es que tanto la masa como la densidad del objeto rígido se mantienen constantes al enviarlo a la Luna.\\
\\
El razonamiento es válido para cualquier cuerpo celeste, no solo la Luna. La masa y densidad de un objeto son propiedades intrínsecas que no dependen del entorno.\\
Sólo cambiarán si el objeto experimentara algún proceso físico como fractura, deformación o cambio de estado.\\
\\
\item{19.}
\item{20.}
Buena pregunta sobre el comportamiento de ciertas propiedades físicas al modificar el volumen de un gas dentro de un recipiente cerrado.\\
\\
Aquí se analizan los posibles cambios:\\
\\
Masa (m): No cambia. La masa es una propiedad intrínseca que no depende del volumen. La cantidad de moléculas de aire dentro sigue siendo la misma.\\
\\
Volumen (V): Disminuye 1 cm al desplazar el émbolo. Pasaría de V a V - 1 cm3.\\
\\
Peso (P=m*g): No cambia. La masa se mantiene constante, con la misma aceleración gravitatoria, por lo que el peso no varía.\\
\\
Densidad (d=m/V): Aumenta. La masa no cambia pero el volumen disminuye, por lo que la densidad pasa de d a (m / (V - 1 cm3)) > d.\\
\\
Presión (P=dgh): Aumenta. A densidad mayor, según la ecuación de estado de los gases ideales, la presión también es mayor en el nuevo volumen. Disminuye h (altura de la columna de gas) pero aumenta d, con lo que P puede aumentar, disminuir o mantenerse, pero aumenta.\\
\\
En resumen:\\
Masa (m) → Constante\\
Volumen (V) → Disminuye 1 cm3\\
Peso (P) → Constante\\
Densidad (d) → Aumenta\\
Presión (P) → Aumenta\\
\\
Al reducir el volumen contenido de gas dentro de un recipiente cerrado, la densidad y presión aumentarán manteniendo constante la masa total de las partículas de gas.\\
\\
{\textbf b) ¿Podríamos repetir la experiencia con la jeringa conteniendo solo agua? Explicar. }
Si realizamos el mismo experimento de desplazar el émbolo de una jeringa conteniendo agua, los cambios serán algo diferentes:

Masa (m): Tampoco cambia. La masa de agua contenida se mantiene constante.

Volumen (V): Disminuye 1 cm3 al desplazar el émbolo.

Peso (P): Aumenta. El peso depende de la masa y la aceleración gravitatoria. La masa no cambia pero al disminuir el volumen, parte de ella se concentrará en un volumen menor, por lo que el peso aparente aumentará.

Densidad (d=m/V): Disminuye. La masa permanece constante pero el volumen disminuye, con lo que la densidad pasaría de d a (m / (V - 1 cm3)) < d.

Presión (P=dgh): Disminuye. Según la ecuación de estado de los fluidos, a menor densidad corresponde menor presión. Al disminuir h, la presión disminuirá aun más.

En resumen:
Masa (m) → Constante
Volumen (V) → Disminuye 1 cm3
Peso (P) → Aumenta
Densidad (d) → Disminuye
Presión (P) → Disminuye

Al comprimir el volumen de agua dentro de un recipiente cerrado, la densidad y presión disminuirán manteniendo constante la masa total de agua.

La principal diferencia con el aire es que la densidad del agua disminuye al comprimirla, en vez de aumentar. Esto se debe a que el agua tiene una compresibilidad negativa, a diferencia de los gases.
\item{21.}
De acuerdo a lo que sucede al congelar agua en una botella cerrada, las únicas afirmaciones correctas son:\\
\\
c) El volumen del sistema aumenta.\\
e) La densidad del sistema disminuye\\
\\
El cambio de estado que tiene lugar se denomina solidificación\\
\\
Las razones son:\\
\\
La masa total del sistema (agua + botella) se mantiene constante. No hay creación o pérdida de masa en el cambio de estado.\\
\\
No se trata de un cambio químico. Es un cambio de estado físico, la solidificación del agua.\\
\\
El volumen aumenta al convertirse el agua en hielo. El hielo ocupa más volumen que el mismo masa de agua líquida.\\
\\
La densidad disminuye. Al mismo volumen total corresponde menos masa en forma de hielo, por lo que la densidad es menor.\\
\\
El peso se mantiene constante. No hay cambio de masa.\\
\\
Es solidificación el cambio de estado. El agua pasa directamente del estado líquido al sólido.\\
\\
No es condensación. La condensación implicaría la formación de vapor de agua, lo que no sucede en este caso.\\
\\
El cambio de estado sucede a presión constante (en un recipiente cerrado) y temperatura decreciente (al enfriar).\\
\\
En resumen:\\
Masa (m) → Constante\\
Volumen (V) → Aumenta\\
Peso (P) → Constante\\
Densidad (d) → Disminuye\\
Cambio de estado → Solidificación\\
\\
No hay cambio químico. La botella y su contenido experimentan un cambio de estado físico.\\
Aumenta el volumen y disminuye la densidad al mismo tiempo que se mantiene constante la masa total.\\
\end{itemize}

\end{document}
