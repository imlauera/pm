\documentclass{article}
\usepackage{geometry}
\geometry{margin=1.4cm}


% Paquete para generar texto Lorem Ipsum
\usepackage{lipsum}

% Configuración de la cabecera de página
\usepackage{fancyhdr}
\pagestyle{fancy}
\fancyhead{} % borra la configuración predefinida
\fancyhead[C]{Título corto} % establece el título en el centro de la cabecera
\fancyfoot{} % borra la configuración predefinida
\fancyfoot[C]{\thepage} % establece el número de página en el centro del pie de página

% Configuración del título del documento
\title{FISICOQUIMICA - Teoria}
\author{Imlauer, Andres}
\date{} % borra la fecha predefinida

\begin{document}

% Creación del título
\maketitle
% Creación de las secciones
\section{Definiciones}
\subsection{Propiedad Intensiva}
Una propiedad intensiva es aquella cuyo valor no depende de la cantidad de materia de un objeto, sino que solo depende de la naturaleza del material en sí mismo. \textbf{Se trata de propiedades que NO dependen del tamaño, volumen o masa de un objeto y son las mismas para cualquier porción del mismo material}.\\
\\
Algunos ejemplos de propiedades intensivas son:\\
• Temperatura: El valor de la temperatura depende de la energía cinética promedio de las partículas de un material, no de su cantidad.\\
• Densidad: La densidad es la masa por unidad de volumen, depende de los materiales, no de la cantidad.\\
• Punto de fusión: El punto de fusión depende de las fuerzas intermoleculares de un material, no cambia al variar la cantidad.\\
• Color: El color de un material depende de las propiedades electrónicas de las moléculas, no cambia con la cantidad.\\
• Dureza: La dureza está relacionada con las fuerzas intermoleculares, es una propiedad intensiva.\\
A diferencia, las propiedades extensivas si dependen de la cantidad de materia, como la masa, el volumen, la energía, etc. Estas propiedades cambian proporcionalmente al cambio de tamaño o cantidad de un objeto.\\

\subsection{Propiedad Extensiva}
Una propiedad extensiva es aquella cuyo valor depende de la cantidad de materia de un objeto. Estas propiedades cambian proporcionalmente al cambio de tamaño, volumen o masa de un objeto.\\
\\
Algunos ejemplos de propiedades extensivas son:\\
• Masa: La masa de un objeto depende de la cantidad de materia que contiene. Si duplicamos la masa de un objeto, su masa se duplicará también.\\
• Volumen: El volumen de un objeto depende de sus dimensiones, así que al cambiar el tamaño del objeto su volumen también cambia. El volumen es aditivo, así que el volumen de dos objetos separados es la suma de los volúmenes de cada uno.\\
• Energía: La energía total de un objeto depende de la cantidad de materia y está relacionada con la masa. Al duplicar la masa, la energía también se duplica.\\
• Calor: El calor específico y el calor latente de cambio de estado de un material son propiedades intensivas, pero el calor total en un proceso depende de la cantidad de materia y es una propiedad extensiva.\\
• Temperatura del gas ideal: Aunque la temperatura es una propiedad intensiva, la energía cinética total de las moléculas en un gas depende de su cantidad y es una propiedad extensiva.\\
\subsection{Clasificacion de Sistemas}
Un sistema en la física y la química puede ser clasificado como cerrado, abierto o aislado. Estos términos se utilizan para describir las interacciones de un sistema con su entorno. A continuación, te explico cada tipo de sistema:\\
\\
• Sistema cerrado: Un sistema cerrado puede intercambiar energía (como calor o trabajo) con su entorno, pero no puede intercambiar materia. La masa del sistema permanece constante, pero la energía puede variar. Un ejemplo de sistema cerrado sería una botella sellada de agua caliente, donde la energía térmica se puede intercambiar con el entorno, pero la masa de agua permanece constante.\\
• Sistema abierto: Un sistema abierto puede intercambiar tanto energía como materia con su entorno. Este tipo de sistema no tiene límites fijos, lo que permite el flujo de sustancias y energía entre el sistema y su entorno. Un ejemplo de sistema abierto podría ser un río, donde fluye agua constantemente, y al mismo tiempo, se intercambia energía (calor, trabajo) con su entorno.\\
• Sistema aislado: Un sistema aislado no intercambia ni energía ni materia con su entorno. En teoría, un sistema aislado no tiene ninguna interacción con su entorno y, por lo tanto, no experimenta cambios en su energía o masa. En la práctica, es difícil lograr un sistema perfectamente aislado, pero un ejemplo aproximado sería una botella termo sellada, que minimiza el intercambio de energía térmica y no permite el intercambio de materia con el entorno.\\
\\
En resumen, un sistema cerrado permite el intercambio de energía pero no de materia, un sistema abierto permite el intercambio de ambas, y un sistema aislado no permite el intercambio de ninguna de las dos.\\
\section{Segunda sección}
\lipsum[10-12]

\subsection{Subsección 1}
\lipsum[13-15]

\subsection{Subsección 2}
\lipsum[16-18]

\end{document}
