\documentclass{article}
\usepackage[margin=1.5cm]{geometry}

\begin{document}

\title{Trabajo práctico FiscoQuimica (anterior a los coloquios)}
\author{Andres Imlauer}
\date{\today}
\maketitle

\noindent\rule{\textwidth}{1pt} \\
\textbf{Respuesta Numero 2} \\

Aquí están los datos solicitados para cada elemento:

\begin{center}
\begin{tabular}{|c|c|c|c|c|}
\hline
Elemento & Número atómico & Grupo & Periodo & Clasificación \\ 
\hline
Oxígeno & 8 & 16 & 2 & No metal \\ 
\hline
Cloro & 17 & 17 & 3 & No metal \\ 
\hline
Silicio & 14 & 14 & 3 & Metaloide(tiene propiedades de un metal y un no metal) \\ 
\hline
Potasio & 19 & 1 & 4 & Metal \\ 
\hline
Bario & 56 & 2 & 6 & Metal \\ 
\hline
Cobalto & 27 & 9 & 4 & Metal \\ 
\hline
Argón & 18 & 18 & 3 & Gas noble \\ 
\hline
Mercurio & 80 & 12 & 6 & Metal \\ 
\hline
\end{tabular}
\end{center}
Es importante tener en cuenta que aunque algunos elementos son clasificados principalmente como metales, no metales o gas inerte, muchos elementos tienen propiedades intermedias o propiedades que no se ajustan perfectamente a ninguna de estas categorías.

\noindent\rule{\textwidth}{1pt} \\
\textbf{Respuesta Número 3} \\
a) El plomo (Pb) tiene número atómico 82, lo que significa que tiene 82 protones en su núcleo. Si cuenta con 125 neutrones, entonces su número másico será la suma de los protones y neutrones, es decir:\\
\\
Número másico = número de protones + número de neutrones\\
Número másico = 82 + 125\\
Número másico = 207\\
\\
Por lo tanto, el átomo de plomo con 125 neutrones tiene número atómico 82 y número másico 207.\\
\\
b) El calcio (Ca) tiene número atómico 20, lo que significa que tiene 20 protones en su núcleo. Si tiene igual cantidad de neutrones que electrones, entonces su número atómico y número másico serán iguales a la suma de protones, neutrones y electrones, es decir:\\
\\
Número atómico = número de protones = 20\\
Número másico = número de protones + número de neutrones = 20 + 20 = 40\\
\\
Por lo tanto, el átomo de calcio con igual cantidad de neutrones que electrones tiene número atómico 20 y número másico 40.\\
\\
c) El número de protones en un átomo neutro es igual al número atómico, por lo que el número de protones en este átomo es 56. Además, el número de neutrones excede en 25 al número de protones, por lo que el número de neutrones será:\\
\\
Número de neutrones = número de protones + 25\\
Número de neutrones = 56 + 25\\
Número de neutrones = 81\\
\\
Por lo tanto, el átomo neutro con 56 electrones y el número de neutrones excede en 25 al número de protones de su núcleo, tiene número atómico 56, número másico 137 y 81 neutrones en su núcleo.\\
\noindent\rule{\textwidth}{1pt} \\
\textbf{Respuesta Número 4} \\
\textbf{a)} El número atómico del plomo es 82, lo que indica que un átomo de plomo neutral tiene 82 protones en su núcleo. Si el átomo de plomo dado tiene 125 neutrones, entonces su número másico será:\\
\\
número másico = número de protones + número de neutrones\\
número másico = 82 + 125\\
número másico = 207\\
\\
Por lo tanto, el átomo de plomo con 125 neutrones en su núcleo tiene un número atómico de 82 y un número másico de 207.\\
\\
b) Si el átomo de calcio tiene la misma cantidad de neutrones que electrones, entonces su número de protones será igual a su número atómico, ya que el átomo es eléctricamente neutro. El número atómico del calcio es 20, lo que significa que hay 20 protones en su núcleo. Por lo tanto, el número de neutrones y electrones en el átomo de calcio será:\\
\\
número de neutrones = número de electrones = número másico - número atómico\\
número de neutrones = número de electrones = 40 - 20\\
número de neutrones = número de electrones = 20\\
\\
Por lo tanto, el átomo de calcio con igual cantidad de neutrones y electrones tiene un número atómico de 20, y su número másico es 40.\\
\\
c) Si el átomo neutro tiene 56 electrones, entonces su número atómico es 56, lo que significa que hay 56 protones en su núcleo. Si el número de neutrones excede en 25 al número de protones, entonces el número de neutrones será:\\
\\
número de neutrones = número de protones + 25\\
número de neutrones = 56 + 25\\
número de neutrones = 81\\
\\
Por lo tanto, el átomo neutro con 56 electrones y el número de neutrones excediendo en 25 al número de protones tiene un número atómico de 56 y un número másico de 137.\\
\noindent\rule{\textwidth}{1pt} \\
\textbf{Respuesta Número 5} \\
La aseveración correcta es a) números atómicos crecientes. La tabla periódica se ordena en función del número atómico creciente, que es el número de protones en el núcleo de un átomo. El número atómico determina las propiedades químicas de un elemento y es un número único e irrepetible para cada elemento.\\
\\
La ordenación de la tabla periódica según las masas atómicas crecientes no es precisa, ya que la masa atómica es una medida promedio que tiene en cuenta tanto el número de protones como el número de neutrones en el núcleo del átomo, y este último puede variar en diferentes átomos del mismo elemento debido a los isótopos.\\
\\
Un ejemplo de dos elementos consecutivos en la tabla periódica que demuestra que la ordenación según las masas atómicas crecientes no es precisa es el cobalto (Co) y el níquel (Ni). El cobalto tiene una masa atómica de 58.93 u, mientras que el níquel tiene una masa atómica de 58.69 u, a pesar de que el número atómico de Co es 27 y el de Ni es 28. Por lo tanto, en este caso, la ordenación según las masas atómicas no sigue la secuencia de los números atómicos.\\
\\
En resumen, la tabla periódica se ordena estrictamente según los números atómicos crecientes, ya que este número es el que determina las propiedades químicas de un elemento y es un número único e irrepetible para cada elemento. La ordenación según las masas atómicas no es precisa debido a la variabilidad de los isótopos.\\
\noindent\rule{\textwidth}{1pt} \\
\textbf{Respuesta Número 6} \\
a) La masa molecular del agua (H2O) es la suma de las masas atómicas de dos átomos de hidrógeno (H) y un átomo de oxígeno (O):\\
\\
Masa molecular de H2O = (2 x masa atómica de H) + (1 x masa atómica de O)\\
Masa molecular de H2O = (2 x 1.008) + 15.999\\
Masa molecular de H2O = 18.015 g/mol\\
\\
Por lo tanto, la masa molecular del agua es de 18.015 g/mol.\\
\\
b) La masa molecular del ácido nítrico (HNO3) es la suma de las masas atómicas de un átomo de hidrógeno (H), un átomo de nitrógeno (N) y tres átomos de oxígeno (O):\\
\\
Masa molecular de HNO3 = (1 x masa atómica de H) + (1 x masa atómica de N) + (3 x masa atómica de O)\\
Masa molecular de HNO3 = 1.008 + 14.007 + (3 x 15.999)\\
Masa molecular de HNO3 = 63.012 g/mol\\
\\
Por lo tanto, la masa molecular del ácido nítrico es de 63.012 g/mol.\\
\\
c) La masa molecular del hidróxido de sodio (NaOH) es la suma de las masas atómicas de un átomo de sodio (Na), un átomo de oxígeno (O) y un átomo de hidrógeno (H):\\
\\
Masa molecular de NaOH = (1 x masa atómica de Na) + (1 x masa atómica de O) + (1 x masa atómica de H)\\
Masa molecular de NaOH = 22.990 + 15.999 + 1.008\\
Masa molecular de NaOH = 39.997 g/mol\\
\\
Por lo tanto, la masa molecular del hidróxido de sodio es de 39.997 g/mol.\\
\\
d) La masa molecular del sulfato de plomo (PbSO4) es la suma de las masas atómicas de un átomo de plomo (Pb), un átomo de azufre (S) y cuatro átomos de oxígeno (O):\\
\\
Masa molecular de PbSO4 = (1 x masa atómica de Pb) + (1 x masa atómica de S) + (4 x masa atómica de O)\\
Masa molecular de PbSO4 = 207.2 + 32.06 + (4 x 15.999)\\
Masa molecular de PbSO4 = 399.87 g/mol\\
\\
Por lo tanto, la masa molecular del sulfato de plomo es de 399.87 g/mol.\\
\\
e) La masa molecular del acetato de metilo (CH3COOCH3) es la suma de las masas atómicas de dos átomos de carbono (C), cuatro átomos de hidrógeno (H) y dos átomos de oxígeno (O):\\
\\
Masa molecular de CH3COOCH3 = (2 x masa atómica de C) + (4 x masa atómica de H) + (2 x masa atómica de O)\\
Masa molecular de CH3COOCH3 = (2 x 12.011) + (4 x 1.008) + (2 x 15.999)\\
Masa molecular de CH3COOCH3 = 73.078 g/mol\\
\\
Por lo tanto, la masa molecular del acetato de metilo es de 73.078 g/mol.\\
\noindent\rule{\textwidth}{1pt} \\
\textbf{Respuesta Número 7 (Resuelto en la pagina del trabajo practico)} \\
\noindent\rule{\textwidth}{1pt} \\

\noindent\rule{\textwidth}{1pt} \\
\textbf{Respuesta Número 8} \\
a) La masa molar del agua (H2O) es de 18.015 g/mol. Por lo tanto, la masa correspondiente a 1,2 mol de H2O es:\\
masa = número de moles x masa molar = 1,2 mol x 18.015 g/mol = 21.618 g\\
Por lo tanto, 1,2 mol de H2O corresponden a 21.618 gramos.\\
b) La masa molar del dióxido de carbono (CO2) es de 44.01 g/mol. Por lo tanto, la masa correspondiente a 0,04 mol de CO2 es:\\
masa = número de moles x masa molar = 0,04 mol x 44.01 g/mol = 1.7604 g\\
Por lo tanto, 0,04 mol de CO2 corresponden a 1.7604 gramos.\\
c) La masa molar del ozono (O3) es de 47.998 g/mol. Por lo tanto, la masa correspondiente a 20 moles de O3 es:\\
masa = número de moles x masa molar = 20 mol x 47.998 g/mol = 959.96 g\\
Por lo tanto, 20 moles de O3 corresponden a 959.96 gramos.\\
d) La masa molar del acetato de metilo (CH3COOCH3) es de 73.078 g/mol. Por lo tanto, la masa correspondiente a $1,2x10^-4$ mol de CH3COOCH3 es:\\
masa = número de moles x masa molar = $1,2x10^-4$ mol x 73.078 g/mol = 0.00876936 g\\
Por lo tanto, $1,2x10^-4$ mol de CH3COOCH3 corresponden a 0.00876936 gramos.\\
e) El número de Avogadro es 6.022 x $10^23$ moléculas/mol. El doble del número de Avogadro es 2 x 6.022 x $10^23$ moléculas. La masa molar del ácido sulfúrico (H2SO4) es de 98.079 g/mol. Por lo tanto, la masa correspondiente a dos veces el número de Avogadro de moléculas de H2SO4 es:\\
masa = número de moléculas x masa molar / número de Avogadro = 2 x 6.022 x $10^23$ x 98.079 g/mol / 6.022 x $10^23$ moléculas/mol = 196.158 g\\
Por lo tanto, un número igual al de dos veces el de Avogadro de moléculas de H2SO4 corresponde a 196.158 gramos.\\



\begin{center}
{\Huge \Large \textbf{Tema Principal}} \\
\end{center}




\noindent\rule{\textwidth}{1pt} \\
\textbf{Respuesta Número 1} \\
a. Metal + Oxígeno → Óxido básico\\
b. Sodio + Oxígeno → Óxido básico\\
c. No metal + Oxígeno → Óxido ácido\\
d. Carbono + Oxígeno → Dióxido de carbono (CO2)\\
e. Metal + Hidrógeno → Hidruro metálico\\
f. No metal + Hidrógeno → Hidruro no metálico\\
g. Óxido ácido + agua → Ácido\\
h. Sodio + Hidrógeno → Hidruro de sodio\\
i. Trióxido de dinitrógeno + agua → Ácido nítrico (HNO3)\\
j. Óxido básico + agua → Hidróxido\\
k. Óxido de sodio + agua → Hidróxido de sodio (NaOH)\\
l. Anhídrido + agua → Ácido o Hidróxido, dependiendo del tipo de anhídrido\\
m. Pentóxido de dinitrógeno + agua → Ácido nítrico (HNO3)\\
n. Hidrácido + hidróxido → Sal + agua\\
o. Cloruro de Hidrógeno + Hidróxido Ferroso → Cloruro ferroso + agua\\
p. Oxácido + Base → Sal + agua\\
q. Ácido clórico + Hidróxido férrico → Clorato férrico + agua\\
\noindent\rule{\textwidth}{1pt} \\
\textbf{Respuesta Número 2} \\
a. Los símbolos y valencias de los elementos dados son:\\
\\
- Boro (B): valencias +3\\
- Magnesio (Mg): valencias +2\\
- Hierro (Fe): valencias +2, +3\\
- Carbono (C): valencias -4, +4\\
- Cobre (Cu): valencias +1, +2\\
- Plomo (Pb): valencias +2, +4\\
- Azufre (S): valencias -2, +4, +6\\
- Cloro (Cl): valencias -1, +1, +3, +5, +7\\
- Fósforo (P): valencias -3, +3, +5\\
En términos generales, Boro, Magnesio, Hierro, Cobre y Plomo son metales, mientras que Carbono, Azufre, Cloro y Fósforo son no metales.\\
b. Las fórmulas de los óxidos posibles para cada elemento son:\\
- Boro: B2O3 (trióxido de di-boro)\\
- Magnesio: MgO (óxido de magnesio)\\
- Hierro (II): FeO (óxido de hierro II o monóxido de hierro)\\
- Hierro (III): Fe2O3 (óxido de hierro III o trióxido de di-hierro)\\
- Carbono (IV): CO2 (dióxido de carbono)\\
- Cobre (I): Cu2O (óxido cuproso)\\
- Cobre (II): CuO (óxido cuproso)\\
- Plomo (II): PbO (óxido de plomo II o monóxido de plomo)\\
- Plomo (IV): PbO2 (dióxido de plomo)\\
- Azufre (-2): SO2 (dióxido de azufre)\\
- Azufre (+4): SO3 (trióxido de azufre)\\
- Azufre (+6): S2O6 (hexaoxodisulfato VI)\\
- Cloro (-1): Cl2O (monóxido de dicloro)\\
- Cloro (+1): Cl2O2 (dioxido de dicloro)\\
- Cloro (+3): Cl2O3 ( trióxido de dicloro)\\
- Cloro (+5): Cl2O5 (pentóxido de dicloro)\\
- Cloro (+7): Cl2O7 (heptóxido de dicloro)\\
- Fósforo (-3): P2O3 (trióxido de di-fósforo)\\
- Fósforo (+3): P2O3 (trióxido de di-fósforo)\\
- Fósforo (+5): P2O5 (pentóxido de di-fósforo)\\
\\
Los nombres de los óxidos son bastante descriptivos y se pueden formar a partir de los nombres de los elementos y los prefijos que indican la cantidad de átomos de oxígeno presentes en la molécula del óxido. En algunos casos, es necesario especificar la valencia del elemento para determinar el nombre completo del óxido. Por ejemplo, el dióxido de carbono (CO2) y el trióxido de di-boro (B2O3) son algunos de los óxidos mencionados anteriormente.\\
\noindent\rule{\textwidth}{1pt} \\
\textbf{Respuesta Número 3} \\
a. El óxido ferroso es FeO.\\
b. El monóxido de hierro es FeO.\\
c. El trióxido de dinitrógeno es N2O3.\\
d. El óxido plúmbico es PbO2.\\
e. El anhídrido perclórico es Cl2O7.\\
f. El pentóxido de diyodo es I2O5.\\
g. El óxido cobáltico es Co2O3.\\
h. El óxido mercúrico es HgO.\\
i. El anhídrido sulfuroso es SO2.\\
j. El anhídrido nítrico es N2O5.\\
\noindent\rule{\textwidth}{1pt} \\
\textbf{Respuesta Número 4} \\
a. El óxido de paladio (II) se expresa simbólicamente como PdO.\\
b. El óxido de uranio (VI) se expresa simbólicamente como UO3.\\
c. El óxido de azufre (VI) se expresa simbólicamente como SO3.\\
d. El óxido de mercurio (I) se expresa simbólicamente como Hg2O.\\
e. El óxido de nitrógeno (III) se expresa simbólicamente como N2O3.\\
f. El óxido de carbono (IV) se expresa simbólicamente como CO2.\\
\noindent\rule{\textwidth}{1pt} \\
\textbf{Respuesta Número 5 (resuelto en las consignas del trabajo)} \\
\noindent\rule{\textwidth}{1pt} \\
\textbf{Respuesta Número 6} \\
Para nombrar compuestos binarios, se utilizan las reglas de nomenclatura química, que se basan en la combinación de los nombres de los elementos y prefijos que indican la cantidad de átomos de cada elemento presentes en la molécula. A continuación, describo cómo llegué a las soluciones para cada compuesto:\\
a. HCl es un compuesto formado por hidrógeno (H) y cloro (Cl). Este compuesto es un ácido, por lo que se le llama ácido clorhídrico.\\
b. NiO es un compuesto formado por níquel (Ni) y oxígeno (O). Este compuesto es un óxido, por lo que se le llama óxido de níquel.\\
c. N2O3 es un compuesto formado por nitrógeno (N) y oxígeno (O). Este compuesto es un óxido, por lo que se le llama trióxido de dinitrógeno.\\
d. Cl2O5 es un compuesto formado por cloro (Cl) y oxígeno (O). Este compuesto es un óxido, por lo que se le llama pentaóxido de dicloro.\\
e. CO2 es un compuesto formado por carbono (C) y oxígeno (O). Este compuesto es un óxido, por lo que se le llama dióxido de carbono.\\
f. LiH es un compuesto formado por litio (Li) e hidrógeno (H). Este compuesto es un hidruro, por lo que se le llama hidruro de litio.\\
g. Li2O es un compuesto formado por litio (Li) y oxígeno (O). Este compuesto es un óxido, por lo que se le llama óxido de litio.\\
h. SO3 es un compuesto formado por azufre (S) y oxígeno (O). Este compuesto es un óxido, por lo que se le llama trióxido de azufre.\\
i. P2O3 es un compuesto formado por fósforo (P) y oxígeno (O). Este compuesto es un óxido, por lo que se le llama trióxido de difósforo.\\
j. CaCl2 es un compuesto formado por calcio (Ca) y cloro (Cl). Este compuesto es un cloruro, por lo que se le llama cloruro de calcio.\\
k. Sb2O3 es un compuesto formado por antimonio (Sb) y oxígeno (O). Este compuesto es un óxido, por lo que se le llama trióxido de diantimonio.\\
l. H2S es un compuesto formado por hidrógeno (H) y azufre (S). Este compuesto es un sulfuro, por lo que se le llama sulfuro de hidrógeno.\\
\noindent\rule{\textwidth}{1pt} \\
\textbf{Respuesta Número 7} \\
Los hidróxidos son compuestos binarios formados por un catión hidróxido (OH-) y un catión metálico. A continuación se presentan las fórmulas de los hidróxidos de los elementos indicados:\\
\\
a) K: KOH (hidróxido de potasio)\\
b) Ca: Ca(OH)2 (hidróxido de calcio)\\
c) Al: Al(OH)3 (hidróxido de aluminio)\\
d) Cu: Cu(OH)2 (hidróxido de cobre II)\\
e) Ni: Ni(OH)2 (hidróxido de níquel II)\\
f) Pb: Pb(OH)2 (hidróxido de plomo II)\\
\noindent\rule{\textwidth}{1pt} \\
\textbf{Respuesta Número 8} \\
Las expresiones incompletas corresponden a reacciones químicas de hidrólisis, en las cuales un compuesto reacciona con agua para producir ácidos o bases. Para completarlas, es necesario escribir los productos de las reacciones. A continuación se presentan las expresiones completas:\\
a. N2O5 + H2O → 2HNO3 (ácido nítrico)\\
La reacción de N2O5 con agua produce ácido nítrico (HNO3).\\
b. CO2 + H2O → H2CO3 (ácido carbónico)\\
La reacción de CO2 con agua produce ácido carbónico (H2CO3).\\
c. Cl2O5 + H2O → 2HClO3 (ácido clórico)\\
La reacción de Cl2O5 con agua produce ácido clórico (HClO3).\\
d. P2O5 + H2O → 2H3PO4 (ácido fosfórico)\\
La reacción de P2O5 con agua produce ácido fosfórico (H3PO4).\\
e. La reacción química que produce ácido nítrico (HNO3) a partir de otros compuestos puede ser realizada de diferentes maneras, dependiendo de los reactivos que se utilicen. A continuación se presentan algunas opciones:\\
NO2 + O2 + H2O → 2HNO3\\
En esta reacción, el dióxido de nitrógeno (NO2) reacciona con oxígeno (O2) y agua (H2O) para producir ácido nítrico (HNO3).\\
NH3 + 4NO2 + H2O → 4HNO3 + NH4NO3\\
En esta reacción, el amoníaco (NH3) reacciona con cuatro moléculas de dióxido de nitrógeno (NO2) y agua (H2O) para producir ácido nítrico (HNO3) y nitrato de amonio (NH4NO3).\\
N2O5 + H2O → 2HNO3\\
En esta reacción, el pentóxido de dinitrógeno (N2O5) reacciona con agua (H2O) para producir ácido nítrico (HNO3).\\
En resumen, para producir ácido nítrico se pueden utilizar diferentes reactivos, pero en todos los casos se requiere la presencia de nitrógeno y oxígeno, que son elementos fundamentales en la estructura de este compuesto.\\
\noindent\rule{\textwidth}{1pt} \\
\textbf{Respuesta Número 9} \\
La ecuación química dada: ácido + hidróxido → sal + agua, representa una reacción de neutralización entre un ácido y una base. Al reaccionar, los iones hidrógeno (H+) del ácido se combinan con los iones hidroxilo (OH-) de la base para formar agua (H2O), mientras que los átomos restantes forman una sal. A continuación se presentan las ecuaciones completas para cada caso:\\
\\
a. Ácido Clorhídrico + Hidróxido de Sodio → Cloruro de Sodio + Agua\\
En esta reacción, los iones hidrógeno (H+) del ácido clorhídrico (HCl) reaccionan con los iones hidroxilo (OH-) del hidróxido de sodio (NaOH) para formar agua (H2O), mientras que los iones cloruro (Cl-) del ácido y los iones sodio (Na+) de la base forman cloruro de sodio (NaCl).\\
b. Ácido Bromhídrico + Hidróxido Férrico → Bromuro Férrico + Agua\\
En esta reacción, los iones hidrógeno (H+) del ácido bromhídrico (HBr) reaccionan con los iones hidroxilo (OH-) del hidróxido férrico [Fe(OH)3] para formar agua (H2O), mientras que los iones bromuro (Br-) del ácido y los iones hierro (Fe3+) de la base forman bromuro férrico (FeBr3).\\
c. Ácido Sulfúrico + Hidróxido de Calcio → Sulfato de Calcio + Agua\\
En esta reacción, los iones hidrógeno (H+) del ácido sulfúrico (H2SO4) reaccionan con los iones hidroxilo (OH-) del hidróxido de calcio (Ca(OH)2) para formar agua (H2O), mientras que los iones sulfato (SO42-) del ácido y los iones calcio (Ca2+) de la base forman sulfato de calcio (CaSO4).\\
d. Ácido Hipocloroso + Hidróxido de Calcio → Hipoclorito de Calcio + Agua\\
En esta reacción, los iones hidrógeno (H+) del ácido hipocloroso (HClO) reaccionan con los iones hidroxilo (OH-) del hidróxido de calcio (Ca(OH)2) para formar agua (H2O), mientras que los iones hipoclorito (ClO-) del ácido y los iones calcio (Ca2+) de la base forman hipoclorito de calcio (Ca(ClO)2).\\
e. Sulfato Ferroso + Hidróxido Férrico → Sulfito Férrico + Agua\\
En esta reacción, los iones hidróxido (OH-) del hidróxido férrico [Fe(OH)3] reaccionan con los iones sulfato (SO42-) del sulfato ferroso (FeSO4) para formar sulfito férrico (Fe2(SO3)3), mientras que se forma agua (H2O) como producto.\\
f. Ácido Perclórico + Hidróxido de Potasio → Perclorato de Potasio + Agua\\
En esta reacción, los iones hidrógeno (H+) del ácido perclórico (HClO4) reaccionan con los iones hidroxilo (OH-) del hidróxido de potasio (KOH) para formar agua (H2O), mientras que los iones perclorato (ClO4-) del ácido y los iones potasio (K+) de la base forman perclorato de potasio (KClO4).\\
g. Ácido Sulfuroso + Hidróxido Cúprico → Sulfito Cúprico + Agua\\
En esta reacción, los iones hidróxido (OH-) del hidróxido cúprico (Cu(OH)2) reaccionan con los iones sulfuroso (SO32-) del ácido sulfuroso (H2SO3) para formar sulfito cúprico (CuSO3), mientras que se forma agua (H2O) como producto.\\
\noindent\rule{\textwidth}{1pt} \\
\textbf{Respuesta Número 10} \\
Para conocer de qué ácidos provienen las sales dadas, es necesario analizar la fórmula de cada una de ellas y determinar a qué ácido corresponde. A continuación se presentan las sales dadas y el ácido correspondiente:\\
a. Cloruro de sodio: NaCl\\
El cloruro de sodio (NaCl) es una sal formada por el catión sodio (Na+) y el anión cloruro (Cl-). Esta sal no proviene de la reacción de un ácido y una base, sino que se forma por la combinación directa de los iones. \\
b. Sulfato de potasio: K2SO4\\
El sulfato de potasio (K2SO4) es una sal formada por el catión potasio (K+) y el anión sulfato (SO42-). Esta sal proviene del ácido sulfúrico (H2SO4), que al reaccionar con la base KOH forma K2SO4 y agua (H2O): H2SO4 + 2KOH → K2SO4 + 2H2O\\
c. Nitrito de bario: Ba(NO2)2\\
El nitrito de bario (Ba(NO2)2) es una sal formada por el catión bario (Ba2+) y el anión nitrito (NO2-). Esta sal proviene del ácido nitroso (HNO2), que al reaccionar con la base Ba(OH)2 forma Ba(NO2)2 y agua (H2O): 2HNO2 + Ba(OH)2 → Ba(NO2)2 + 2H2O\\
d. Hipoclorito de sodio: NaClO\\
El hipoclorito de sodio (NaClO) es una sal formada por el catión sodio (Na+) y el anión hipoclorito (ClO-). Esta sal proviene del ácido hipocloroso (HClO), que al reaccionar con la base NaOH forma NaClO y agua (H2O): HClO + NaOH → NaClO + H2O\\
En resumen, el cloruro de sodio no proviene de la reacción de un ácido y una base, mientras que el sulfato de potasio proviene del ácido sulfúrico, el nitrito de bario proviene del ácido nitroso, y el hipoclorito de sodio proviene del ácido hipocloroso.\\
\noindent\rule{\textwidth}{1pt} \\
\textbf{Respuesta Número 11} \\
a. Amoníaco: NH3\\
Sinónimos: Nitruro de hidrógeno, trihidruro de nitrógeno.\\
b. Anhídrido clórico: Cl2O7\\
Sinónimos: Óxido de cloro(VII), heptóxido de dicloro.\\
c. Anhídrido nítrico: N2O5\\
Sinónimos: Óxido de nitrógeno(V), pentóxido de dinitrógeno.\\
d. Trióxido de azufre: SO3\\
Sinónimos: Óxido de azufre(VI), anhídrido sulfúrico.\\
e. Hidróxido cúprico: Cu(OH)2\\
Sinónimos: Hidróxido de cobre(II).\\
f. Óxido de Cobre: CuO\\
Sinónimos: Óxido cuproso, monóxido de cobre.\\
g. Pentóxido de difósforo: P2O5\\
Sinónimos: Óxido de fósforo(V), anhídrido fosfórico.\\
h. Fosfatina: PH3 (trihidruro de fósforo)\\
Sinónimos: Fosfina, hidruro de fósforo(III).\\
\noindent\rule{\textwidth}{1pt} \\
\textbf{Respuesta Número 12} \\
a. HClO: Ácido hipocloroso\\
b. HClO2: Ácido cloroso\\
c. HClO3: Ácido clórico\\
d. HClO4: Ácido perclórico\\
e. HPO2: Ácido hipofosforoso\\
f. H3PO3: Ácido fosforoso\\
g. H4P2O5: Anhídrido difosfórico o tetraóxido de difósforo\\
h. Sn(OH)2: Hidróxido de estaño(II) o hidróxido estannoso\\
i. Sn(OH)4: Hidróxido de estaño(IV) o hidróxido estannico\\
j. Na2SO4: Sulfato de sodio\\
k. NaHSO4: Hidrogenosulfato de sodio o bisulfato de sodio\\
l. NaKSO4: Sulfato de sodio y potasio\\
m. Na2HPO4: Hidrogenofosfato de disodio o fosfato disódico\\
n. NaH2PO4: Dihidrogenofosfato de sodio o fosfato monosódico
\noindent\rule{\textwidth}{1pt} \\
\textbf{Respuesta Número 13} \\
a. Co2O3: Trióxido de dicobalto o Óxido de cobalto(III)\\
b. Cu2O: Monóxido de dicobre o Óxido cuproso\\
c. ZnO: Óxido de zinc\\
d. PbO2: Dióxido de plomo o Óxido plúmbico\\
e. MoO3: Trióxido de molibdeno o Óxido de molibdeno(VI)\\
f. Al2(SO4)3: Sulfato de aluminio\\
g. FeBr3: Bromuro de hierro(III)\\
h. BaCl2: Cloruro de bario\\
i. TiCl4: Tetracloruro de titanio\\
j. BiCl3: Cloruro de bismuto(III)\\
k. NiS: Sulfuro de níquel(II)\\
l. MnS: Sulfuro de manganeso(II)\\
m. Ag2S: Sulfuro de plata(I)\\
n. Cu(NO3)2: Nitrato de cobre(II)\\
o. KNO2: Nitrito de potasio\\
p. NaIO3: Yodato de sodio\\
q. Sn(SO4)2: Sulfato de estaño(IV) o Sulfato de estaño tetraédrico\\
\noindent\rule{\textwidth}{1pt} \\
\textbf{Respuesta Número 14} \\
a. KClO3: Clorato de potasio\\
b. PbSO4: Sulfato de plomo(II) o Sulfato plumboso\\
c. Pb(SO3)2: Sulfito de plomo(IV) o Sulfito plúmbico\\
d. NH4NO2: Nitrito de amonio\\
e. Ca3(PO4)2: Fosfato tricálcico o Fosfato de calcio y calcio\\
\noindent\rule{\textwidth}{1pt} \\
\textbf{Respuesta Número 15} \\
a. NH4NO3: Este compuesto se puede analizar como la combinación de dos iones, el catión amonio (NH4+) y el anión nitrato (NO3-). Por lo tanto, se trata de un nitrato de amonio.\\
b. Mg(ClO)2: Este compuesto se puede analizar como la combinación de un catión magnesio (Mg2+) y un anión hipoclorito (ClO-). Dado que el hipoclorito es un ion poliatómico, se utiliza el prefijo "hipo" para indicar que se trata del ion con menor cantidad de oxígenos. Por lo tanto, este compuesto se llama hipoclorito de magnesio.\\
c. FeSO4: Este compuesto se puede analizar como la combinación de un catión hierro (II) (Fe2+) y un anión sulfato (SO42-). Por lo tanto, se trata de un sulfato de hierro (II) o sulfato ferroso.\\
d. Ba(PO3)2: Este compuesto se puede analizar como la combinación de un catión bario (Ba2+) y un anión metafosfato (PO33-). El metafosfato es un ion poliatómico que se forma a partir de la unión de varios grupos fosfato (PO43-). Por lo tanto, este compuesto se llama metafosfato de bario.\\
e. HClO3: Este compuesto es un ácido que se forma a partir de un catión hidrógeno (H+) y un anión hipoclorito (ClO3-). Dado que es un ácido, se utiliza el prefijo "ácido" seguido del nombre del anión con la terminación "-ico" y la palabra "ácido". Por lo tanto, este compuesto se llama ácido clórico.\\
\noindent\rule{\textwidth}{1pt} \\
\textbf{Respuesta Número 16 (RESUELTO EN EL TP)} \\
\noindent\rule{\textwidth}{1pt} \\
\textbf{Respuesta Número 17} \\
a. H3PO3: Ácido fosforoso\\
b. H2SiO3: Ácido metasilícico\\
c. HPO2: Ácido hipofosforoso\\
d. H6Si2O7: Ácido disilícico o Ácido pirósilícico\\
e. H3BO3: Ácido bórico\\
f. H4SiO4: Ácido ortosilícico\\
\noindent\rule{\textwidth}{1pt} \\
\textbf{Respuesta Número 18} \\
\begin{table}[h]
\centering
\begin{tabular}{|c|c|c|c|}
\hline
Sal     & Hidróxido             & Ácido            & Nombre                                       \\ \hline
KNO3    & KOH + HNO3            & Ácido nítrico    & Nitrato de potasio                           \\ \hline
CaBr2   & Ca(OH)2 + 2HBr        & Ácido bromhídrico & Bromuro de calcio                            \\ \hline
FeCl3   & Fe(OH)3 + 3HCl        & Ácido clorhídrico & Cloruro de hierro(III) o Cloruro férrico     \\ \hline
K2CO3   & 2KOH + CO2            & Ácido carbónico  & Carbonato de potasio                         \\ \hline
SnCl4   & Sn(OH)4 + 4HCl        & Ácido clorhídrico & Cloruro de estaño(IV) o Cloruro estannoso    \\ \hline
PbSO4   & Pb(OH)2 + H2SO4       & Ácido sulfúrico  & Sulfato de plomo(II) o Sulfato plumboso      \\ \hline
PbS2    & Pb(OH)2 + H2S         & Ácido sulfhídrico & Sulfuro de plomo(IV) o Sulfuro plúmbico      \\ \hline
NH4Cl   & NH3 + HCl             & Ácido clorhídrico & Cloruro de amonio                            \\ \hline
NaHSO3  & NaOH + H2SO3          & Ácido sulfuroso  & Hidrogenosulfito de sodio o Bisulfito de sodio\\ \hline
\end{tabular}
\end{table}En la columna "Hidróxido", se indica la hidrólisis de la sal, es decir, los hidróxidos que se forman a partir de la reacción de la sal con agua. En la columna "Ácido", se indica el ácido que se forma a partir de la reacción de los hidróxidos obtenidos en la hidrólisis. En la columna "Nombre", se indica el nombre de la sal correspondiente, que se puede deducir a partir del ácido e hidróxido que la forman.\\\\
\\\\
Es importante tener en cuenta que, en algunos casos, puede haber más de una opción para nombrar una sal, dependiendo de los ácidos e hidróxidos que se utilicen. En estos casos, se suele utilizar la nomenclatura más común y aceptada por la IUPAC.\\\\
\noindent\rule{\textwidth}{1pt} \\
\textbf{Respuesta Número 19} \\

\noindent\rule{\textwidth}{1pt} \\
\textbf{Respuesta Número 20} \\
a. Ca + O2 → CaO\\
   La ecuación ya está balanceada.\\
b. 4Al + 3O2 → 2Al2O3\\
   En esta reacción, hay que añadir un coeficiente 4 delante del Al para igualar el número de átomos de Al en ambos lados de la ecuación.\\
c. 4K + O2 → 2K2O\\
   Para equilibrar la ecuación, es necesario colocar un coeficiente 4 delante del K en el lado reactivo y un 2 delante del K2O en el lado de los productos.\\
d. N2 + 2O2 → 2N2O5\\
   Para equilibrar la ecuación, es necesario colocar un coeficiente 2 delante del NO2 en el lado reactivo y un 4 delante del N2O5 en el lado de los productos.\\
e. 2Al + 6HCl → 2AlCl3 + 3H2\\
   Para equilibrar esta ecuación, se deben colocar coeficientes 2 y 3 delante del Al y H2, respectivamente, en el lado reactivo. En el lado de los productos, se coloca un coeficiente 2 delante de AlCl3.\\
f. 4FeS + 7O2 → 2Fe2O3 + 4SO2\\
   En esta reacción, es necesario colocar un coeficiente 2 delante de Fe2O3 en el lado de los productos para equilibrar la ecuación.\\
g. H2SO4 + 2Al(OH)3 → Al2(SO4)3 + 6H2O\\
   Para equilibrar la ecuación, es necesario colocar un coeficiente 2 delante del Al(OH)3 en el lado reactivo y un coeficiente 3 delante de H2O en el lado de los productos.\\
h. 5I2 + 6HNO3 → 2HIO3 + 12NO2 + 5H2O\\
   Para equilibrar la ecuación, es necesario colocar coeficientes 5, 6, y 2 delante de I2, HNO3, y HIO3, respectivamente, en el lado reactivo. En el lado de los productos, se deben colocar coeficientes 12 y 5 delante de NO2 y H2O, respectivamente.\\
i. C3H8 + 5O2 → 3CO2 + 4H2O\\
   En esta reacción, es necesario colocar un coeficiente 5 delante del O2 en el lado reactivo para igualar el número de átomos de O en ambos lados de la ecuación.\\
\noindent\rule{\textwidth}{1pt} \\
\textbf{Respuesta Número 21} \\

\noindent\rule{\textwidth}{1pt} \\
\textbf{Respuesta Número 22} \\
a. C10H2O + 8O2 → 10CO2 + 5H2O\\
   Para balancear esta ecuación química, primero contamos los átomos de cada elemento a ambos lados. En el lado izquierdo, hay 10 átomos de C, 2 átomos de H y 16 átomos de O. En el lado derecho, hay 10 átomos de C, 10 átomos de O y 5 átomos de H. Para igualar los átomos de cada elemento, se puede colocar un coeficiente 8 delante del O2 en el lado reactivo y un coeficiente 5 delante del H2O en el lado de los productos.\\
\\
b. H4P2O7 + 3Al(OH)3 → 2Al(H2P2O7) + 3H2O\\
   Primero, contamos los átomos de cada elemento en ambos lados de la ecuación. En el lado izquierdo, hay 4 átomos de H, 2 átomos de P, 14 átomos de O y 9 átomos de Al. En el lado derecho, hay 4 átomos de H, 4 átomos de P, 14 átomos de O y 2 átomos de Al. Para igualar los átomos de cada elemento, se puede colocar un coeficiente 3 delante del Al(OH)3 en el lado reactivo y un coeficiente 2 delante del Al(H2P2O7) en el lado de los productos.\\
\noindent\rule{\textwidth}{1pt} \\
\textbf{Respuesta Número 23} \\

\noindent\rule{\textwidth}{1pt} \\
\textbf{Respuesta Número 24} \\
a. Fe(NO3)3 + 3CO + 3H2O → Fe(NO2)3 + 3H2CO3\\
   Primero, equilibramos los átomos de Fe a ambos lados, colocando un coeficiente 1 delante de Fe(NO3)3 y Fe(NO2)3. Luego, equilibramos los átomos de O, primero con los NO3, colocando un coeficiente 3 delante de CO2 y H2CO3 y luego con el H2O, colocando un coeficiente 3 delante de H2O en el lado reactivo.\\
b. 3MgS + 2HNO3 → 3H2SO3 + 2HNO2 + Mg(NO3)2\\
   En esta reacción, primero equilibramos los átomos de Mg a ambos lados colocando un coeficiente 1 delante de MgS y Mg(NO3)2. Luego, equilibramos los átomos de N y H, colocando un coeficiente 2 delante de HNO3 en el lado reactivo y un coeficiente 2 delante de HNO2 en el lado de los productos. Finalmente, equilibramos los átomos de S y O, colocando un coeficiente 3 delante de MgS y un coeficiente 3 delante de H2SO3 en el lado de los productos.\\
c. 2Al(NO3)3 + 3CaCO3 → 3Ca(NO3)2 + Al2(CO3)3\\
   Para equilibrar esta ecuación química, primero equilibramos los átomos de Al a ambos lados colocando un coeficiente 2 delante de Al(NO3)3 y un coeficiente 2 delante de Al2(CO3)3. Luego, equilibramos los átomos de Ca y N, colocando un coeficiente 3 delante de Ca(NO3)2 en el lado de los productos.\\
d. 3NaMnO4 + 2H2O + 2PbO → 2Pb(OH)4 + 3MnO2 + 3NaOH\\
   En esta reacción, primero equilibramos los átomos de Na a ambos lados colocando un coeficiente 3 delante de NaMnO4 y NaOH. Luego, equilibramos los átomos de Mn y O, colocando un coeficiente 3 delante de MnO2 en el lado de los productos y un coeficiente 4 delante de H2O en el lado reactivo. Finalmente, equilibramos los átomos de Pb e H, colocando un coeficiente 2 delante de PbO en el lado reactivo y un coeficiente 4 delante de Pb(OH)4 en el lado de los productos.\\
\noindent\rule{\textwidth}{1pt} \\
\textbf{Respuesta Número 25} \\
a. B2O3 + 3Mg → 3MgO + 2B\\
En esta reacción, primero equilibramos los átomos de B y Mg a ambos lados colocando un coeficiente 2 delante de B2O3 y un coeficiente 3 delante de MgO. \\
b. Al2O3 + 3C + 3Cl2 → 2AlCl3 + 3CO\\
Para equilibrar esta ecuación química, primero equilibramos los átomos de Al a ambos lados colocando un coeficiente 2 delante de AlCl3. Luego, equilibramos los átomos de C y Cl, colocando un coeficiente 3 delante de C y Cl2 en el lado reactivo.\\
c. Al2O3 + 3H2S → Al2S3 + 3H2O\\
En esta reacción, primero equilibramos los átomos de Al y S a ambos lados colocando un coeficiente 2 delante de Al2S3. Luego, equilibramos los átomos de H y O, colocando un coeficiente 3 delante de H2O en el lado de los productos.\\
d. PbO + PbS → 2SO2 + 2Pb\\
En esta reacción, primero equilibramos los átomos de Pb a ambos lados colocando un coeficiente 2 delante de PbS y Pb. Luego, equilibramos los átomos de S y O, colocando un coeficiente 2 delante de SO2 en el lado de los productos.\\
e. As2S3 + 3O2 → 2SO2 + As2O3\\
En esta reacción, equilibramos los átomos de S y O colocando un coeficiente 3 delante de O2 en el lado reactivo y un coeficiente 2 delante de SO2 en el lado de los productos.\\
f. Ca3(PO4)2 + SiO2 + 5C → 3CaSiO3 + 5CO + 2P\\
En esta reacción, primero equilibramos los átomos de C y P a ambos lados colocando un coeficiente 5 delante de C y un coeficiente 2 delante de P en el lado de los productos. Luego, equilibramos los átomos de Ca y Si, colocando un coeficiente 3 delante de CaSiO3 en el lado de los productos.\\
\noindent\rule{\textwidth}{1pt} \\
\textbf{Respuesta Número 26} \\
a) La ecuación química balanceada es:\\
3Mg3(PO4)2 + 2l2O3 + 6H2O → 2Mg(lO2)2 + 6H3PO4\\
Primero, equilibramos los átomos de Mg a ambos lados colocando un coeficiente 2 delante de Mg(lO2)2 y 3 delante de Mg3(PO4)2. Luego, equilibramos los átomos de O, primero con los PO4, colocando un coeficiente 2 delante de l2O3 y Mg(lO2)2, y luego con el agua, colocando un coeficiente 6 delante de H2O en el lado reactivo.\\
\\
b) Para calcular la relación en masas entre l2O3 y Mg3(PO4)2 reaccionados en proporción estequiométrica, necesitamos conocer los coeficientes estequiométricos de ambas sustancias. A partir de la ecuación balanceada, podemos ver que la relación estequiométrica es 2:3 para l2O3 y Mg3(PO4)2, respectivamente. Por lo tanto, si tomamos 2 moles de l2O3, necesitaremos 3 moles de Mg3(PO4)2. Conociendo las masas molares de cada sustancia, podemos calcular la relación en masas:\\
\\
masa de l2O3 / masa de Mg3(PO4)2 = (2 x masa molar de l2O3) / (3 x masa molar de Mg3(PO4)2)\\
\\
c) Para determinar la relación en moles entre Mg(lO2)2 y H3PO4 obtenidos, necesitamos conocer los coeficientes estequiométricos de ambas sustancias en la ecuación balanceada. A partir de la ecuación balanceada, podemos ver que la relación estequiométrica es 2:6 o 1:3 para Mg(lO2)2 y H3PO4, respectivamente. Por lo tanto, si se forman "x" moles de Mg(lO2)2, se formarán 3x moles de H3PO4. La relación en moles será:\\
\\
moles de Mg(lO2)2 / moles de H3PO4 = 1 / 3\\
\noindent\rule{\textwidth}{1pt} \\
\textbf{Respuesta Número 27} \\
La ecuación química de la reacción entre hierro y ácido clorhídrico es:\\
\\
Fe + 2HCl → FeCl2 + H2\\
\\
Podemos ver que para obtener 1 mol de hidrógeno se necesitan 1 mol de hierro y 2 moles de ácido clorhídrico. Por lo tanto, para obtener 8 gramos de hidrógeno, necesitamos calcular primero el número de moles de hidrógeno que corresponden a esa masa:\\
\\
masa de H2 = n x masa molar de H2\\
8 g = n x 2 g/mol\\
n = 4 moles de H2\\
\\
Como se necesitan 2 moles de HCl para obtener 1 mol de H2, se necesitan 8 moles de HCl para obtener 4 moles de H2. A partir de la ecuación química, podemos ver que 1 mol de hierro reacciona con 2 moles de HCl para formar 1 mol de FeCl2. Por lo tanto, se necesitan 2 moles de hierro para reaccionar con los 8 moles de HCl necesarios para obtener 4 moles de H2. \\
\\
La masa molar del hierro es de 55.85 g/mol y la del ácido clorhídrico es de 36.46 g/mol. Por lo tanto, las masas de hierro y ácido clorhídrico necesarias para obtener 8 gramos de H2 son:\\
\\
masa de hierro = n x masa molar de hierro\\
masa de hierro = 2 mol x 55.85 g/mol\\
masa de hierro = 111.7 g\\
\\
masa de HCl = n x masa molar de HCl\\
masa de HCl = 8 mol x 36.46 g/mol\\
masa de HCl = 291.7 g\\
\\
Para determinar la relación de número de moles de ácido clorhídrico a número de moles de cloruro férrico, primero necesitamos conocer los coeficientes estequiométricos de ambas sustancias en la ecuación balanceada. A partir de la ecuación balanceada, podemos ver que la relación estequiométrica es 2:1 para HCl y FeCl2, y 1:1 para FeCl2 y FeCl3. Por lo tanto, la relación entre el número de moles de HCl y FeCl3 es 2:1.\\
\\
nHCl/nFeCl3 = 2/1\\
\noindent\rule{\textwidth}{1pt} \\
\textbf{Respuesta Número 28} \\
La ecuación química de la reacción entre dióxido de manganeso y ácido clorhídrico es:\\
\\
MnO2 + 4HCl → MnCl2 + Cl2 + 2H2O\\
\\
Podemos ver que para obtener 1 mol de cloro se necesitan 1 mol de MnO2 y 4 moles de HCl. Por lo tanto, para obtener 1420 gramos de cloro, necesitamos calcular primero el número de moles de cloro que corresponden a esa masa:\\
\\
masa de Cl2 = n x masa molar de Cl2\\
1420 g = n x 71 g/mol\\
n = 20 moles de Cl2\\
\\
A partir de la ecuación química, podemos ver que 1 mol de MnO2 reacciona con 4 moles de HCl para formar 1 mol de Cl2. Por lo tanto, se necesitan 20 moles de MnO2 y 80 moles de HCl para obtener los 20 moles de Cl2 necesarios.\\
\\
La masa molar del dióxido de manganeso es de 86.94 g/mol y la del ácido clorhídrico es de 36.46 g/mol. Por lo tanto, las masas de dióxido de manganeso y ácido clorhídrico necesarias para obtener 1420 gramos de Cl2 son:\\
\\
masa de MnO2 = n x masa molar de MnO2\\
masa de MnO2 = 20 mol x 86.94 g/mol\\
masa de MnO2 = 1738.8 g\\
\\
masa de HCl = n x masa molar de HCl\\
masa de HCl = 80 mol x 36.46 g/mol\\
masa de HCl = 2916.8 g\\
\\
Para calcular la relación nMnO2/mHCl, dividimos el número de moles de MnO2 entre la masa de HCl en gramos:\\
\\
nMnO2/mHCl = 20 mol / 2916.8 g\\
nMnO2/mHCl = 0.00685 mol/g\\
\\
La relación nMnO2/mHCl indica la cantidad de moles de MnO2 necesarios para reaccionar con un gramo de HCl. En este caso, se necesitan aproximadamente 0.00685 moles de MnO2 por cada gramo de HCl.\\
\noindent\rule{\textwidth}{1pt} \\
\textbf{Respuesta Número 29} \\
Para resolver este problema, primero necesitamos conocer la relación estequiométrica entre el ácido sulfúrico y el hidróxido de aluminio, que es 3:2. Esto significa que se necesitan 3 moles de ácido sulfúrico para reaccionar completamente con 2 moles de hidróxido de aluminio.\\
\\
a) Para saber si las cantidades dadas guardan las proporciones estequiométricas, podemos calcular el número de moles de cada reactivo y compararlos con la relación estequiométrica.\\
\\
$6,022 x 10^23$ moléculas de H2SO4 = $6,022 x 10^-1$ moles de H2SO4\\
$3,011 x 10^23$ moléculas de Al(OH)3 = $5,018 x 10^-1$ moles de Al(OH)3\\
\\
La relación entre los moles de H2SO4 y Al(OH)3 es:\\
\\
$6,022 x 10^-1$ moles de H2SO4 / $5,018 x 10^-1$ moles de Al(OH)3 = 1.20\\
\\
Como la relación no es igual a la relación estequiométrica de 3:2, las cantidades dadas no guardan las proporciones estequiométricas.\\
\\
b) Para calcular la relación de masas entre los reactivos, calculamos los moles de cada reactivo a partir de las masas dadas y las masas molares correspondientes.\\
\\
20 g de H2SO4 = 0.204 moles de H2SO4 (masa molar de H2SO4 = 98.08 g/mol)\\
15 g de Al(OH)3 = 0.204 moles de Al(OH)3 (masa molar de Al(OH)3 = 78.0 g/mol)\\
\\
La relación entre las masas de H2SO4 y Al(OH)3 es:\\
\\
20 g de H2SO4 / 15 g de Al(OH)3 = 4/3\\
\\
Esta relación no es igual a la relación estequiométrica de 3:2, por lo que las cantidades dadas no guardan las proporciones estequiométricas.\\
\\
c) Para calcular la relación de masas entre los reactivos, calculamos los moles de cada reactivo a partir de las masas dadas y las masas molares correspondientes.\\
\\
1,5 g de H2SO4 = 0.0153 moles de H2SO4 (masa molar de H2SO4 = 98.08 g/mol)\\
78,01 g de Al(OH)3 = 1.00 moles de Al(OH)3 (masa molar de Al(OH)3 = 78.0 g/mol)\\
\\
La relación entre las masas de H2SO4 y Al(OH)3 es:\\
\\
1,5 g de H2SO4 / 78,01 g de Al(OH)3 = 0.0192\\
\\
Esta relación tampoco es igual a la relación estequiométrica de 3:2, por lo que las cantidades dadas no guardan las proporciones estequiométricas.\\
En resumen, ninguna de las cantidades dadas en los tres casos guarda las proporciones estequiométricas de la reacción.\\
\noindent\rule{\textwidth}{1pt} \\
\textbf{Respuesta Número 30} \\
Primero, necesitamos balancear la ecuación química para poder utilizarla en cálculos estequiométricos:\\
P2O5 + 5C → 5CO + 2P\\
\\
A partir de la ecuación balanceada, podemos ver que se necesitan 5 moles de carbono para producir 2 moles de fósforo. Por lo tanto, podemos utilizar una relación estequiométrica para calcular la cantidad de moles de carbono necesarios para producir una determinada cantidad de fósforo.\\
La masa molar del fósforo es de 30.97 g/mol. Por lo tanto, la masa de 2,480 Kg de fósforo es:\\
\\
2,480 Kg = 2,480,000 g\\
n = masa / masa molar = 2,480,000 g / 30.97 g/mol = 80,050 moles de P\\
\\
A partir de la relación estequiométrica, podemos calcular la cantidad de moles de carbono necesarios para producir 80,050 moles de fósforo:\\
80,050 moles de P x (5 moles de C / 2 moles de P) = 200,125 moles de C\\
Por lo tanto, se necesitan 200,125 moles de carbono para producir 2,480 Kg de fósforo a partir del anhídrido fosfórico.\\

\textbf{Respuesta Número 31} \\
La ecuación química balanceada para la reacción entre óxido estánnico y carbono es:\\
SnO2 + 2C → Sn + 2CO\\
Podemos ver que por cada mol de óxido estánnico se necesitan 2 moles de carbono para producir 2 moles de monóxido de carbono y 1 mol de estaño. Podemos utilizar esta relación estequiométrica para calcular cuántos moles de óxido estánnico reaccionan con los 30 moles de carbono dados en el problema.\\
Primero, calculamos los moles de carbono a partir de la masa dada y la masa molar del carbono:\\
30 moles de C = 30 x 12.01 g/mol = 360.3 g de C\\
A continuación, calculamos los moles de óxido estánnico a partir de la masa dada y la masa molar del SnO2:\\
2260.5 g de SnO2 = 2260.5 / 150.71 g/mol = 15 moles de SnO2\\
Como se necesitan 2 moles de carbono para cada mol de SnO2, se necesitan 30 moles de carbono para reaccionar con:\\
15 moles de SnO2 x (2 moles de C / 1 mol de SnO2) = 30 moles de C\\
Por lo tanto, la cantidad de óxido estánnico dada en el problema reacciona completamente con los 30 moles de carbono dados.\\
A partir de la ecuación química balanceada, podemos ver que por cada 2 moles de monóxido de carbono se produce 1 mol de estaño. Por lo tanto, los 0.84 kg de monóxido de carbono (840 g) corresponden a:\\
840 g de CO x (1 mol de Sn / 2 moles de CO) = 420 moles de Sn\\
Por lo tanto, se producen 420 moles de estaño en la reacción.\\
Para determinar la cantidad de la segunda sustancia producida, podemos utilizar la ley de conservación de la masa. Sabemos que la masa total de los productos de la reacción es igual a la masa total de los reactivos, por lo que podemos calcular la masa de la segunda sustancia a partir de las masas de los reactivos y productos.\\
Masa total de reactivos = masa total de productos\\
2260.5 g de SnO2 + 360.3 g de C = 420 moles de Sn x 118.71 g/mol (masa molar de Sn)\\
Masa total de productos = 49923.2 g\\
La masa de la segunda sustancia producida es la diferencia entre la masa total de productos y la masa de monóxido de carbono:\\
Masa de la segunda sustancia = 49923.2 g - 840 g = 49083.2 g\\
A partir de la relación estequiométrica de la ecuación química, podemos ver que la segunda sustancia producida es el óxido de estaño (SnO), y que se producen 1 mol de SnO por cada mol de Sn producido. Por lo tanto, la cantidad de SnO producida es:\\
420 moles de Sn x (1 mol de SnO / 1 mol de Sn) = 420 moles de SnO\\
La masa de SnO producida es:\\
420 moles de SnO x 134.71 g/mol (masa molar de SnO) = 56,557.2 g\\
Por lo tanto, la segunda sustancia producida es el óxido de estaño (SnO), y se producen 56,557.2 g de SnO.\\
\textbf{Respuesta Número 32} \\
La ecuación química balanceada para la reacción entre hidrógeno y nitrógeno para formar amoníaco es:\\
\\
N2 + 3H2 → 2NH3\\
\\
Podemos ver que para producir 2 moles de amoníaco se necesitan 3 moles de hidrógeno y 1 mol de nitrógeno. Utilizaremos esta relación estequiométrica para responder a las preguntas.\\
\\
a) Primero, necesitamos calcular cuántos moles de amoníaco se produjeron en la reacción. La masa molar del amoníaco es de 17.03 g/mol.\\
\\
n = masa / masa molar = 6.8 g / 17.03 g/mol = 0.399 moles de NH3\\
\\
Como se necesitan 3 moles de hidrógeno para producir 2 moles de amoníaco, el número de moles de hidrógeno que reaccionaron es:\\
\\
nH2 = (2/3) x nNH3 = (2/3) x 0.399 moles = 0.266 moles de H2\\
\\
Por lo tanto, 0.266 moles de hidrógeno reaccionaron.\\
\\
b) La cantidad de nitrógeno que se consumió en la reacción se puede calcular utilizando la relación estequiométrica de la ecuación química balanceada.\\
\\
Para producir 2 moles de amoníaco, se necesitan 1 mol de nitrógeno. Por lo tanto, el número de moles de nitrógeno que se consumió es:\\
\\
nN2 = (1/2) x nNH3 = (1/2) x 0.399 moles = 0.1995 moles de N2\\
\\
Para calcular el volumen de nitrógeno consumido, utilizaremos la ley de los gases ideales. Primero, necesitamos conocer las condiciones estándar de temperatura y presión (CEPT), que son 0 °C (273.15 K) y 1 atm (101.325 kPa).\\
\\
El volumen de gas a las condiciones estándar se puede calcular utilizando la siguiente ecuación:\\
\\
V = nRT/P\\
\\
Donde V es el volumen en litros, n es el número de moles, R es la constante de los gases ideales (0.0821 L atm/mol K), T es la temperatura en Kelvin y P es la presión en atm.\\
\\
Para el nitrógeno consumido, podemos calcular su volumen utilizando:\\
\\
V = nN2RT/P = 0.1995 mol x 0.0821 L atm/mol K x 273.15 K / 1 atm = 4.81 L\\
\\
Por lo tanto, se consumieron aproximadamente 4.81 litros de nitrógeno en la reacción.\\
\textbf{Respuesta Número 33} \\
La ecuación química balanceada para la síntesis de óxido de sodio es:\\
\\
4 Na + O2 → 2 Na2O\\
\\
Podemos ver que se necesitan 4 moles de sodio y 1 mol de oxígeno para producir 2 moles de óxido de sodio. Utilizaremos esta relación estequiométrica para responder a las preguntas.\\
\\
a) El número de moles de oxígeno consumido se puede calcular utilizando la relación estequiométrica de la ecuación química balanceada. Como se necesitan 4 moles de sodio para cada mol de oxígeno, el número de moles de oxígeno consumido es:\\
\\
nO2 = (1/4) x nNa = (1/4) x 8.12 moles = 2.03 moles de O2\\
Por lo tanto, se consumen 2.03 moles de oxígeno.\\
b) Para calcular el número de moléculas de oxígeno consumido, podemos utilizar la constante de Avogadro, que relaciona el número de moléculas con el número de moles:\\
Número de moléculas de O2 = nO2 x $N_A$\\
Donde $N_A$ es la constante de Avogadro ($6.02 x 10^23$ moléculas/mol).\\
Número de moléculas de O2 = 2.03 moles x $6.02 x 10^23$ moléculas/mol = 1.22 x $10^24$ moléculas de O2\\
Por lo tanto, se consumen aproximadamente $1.22 x 10^24$ moléculas de oxígeno.\\
c) Para calcular la masa de oxígeno consumida, podemos utilizar la masa molar del O2, que es de 32.00 g/mol.\\
Masa de O2 = nO2 x masa molar = 2.03 moles x 32.00 g/mol = 64.96 g de O2\\
Por lo tanto, se consumen aproximadamente 64.96 gramos de oxígeno.\\
d) Para calcular el volumen de oxígeno consumido, utilizaremos la ley de los gases ideales. Primero, necesitamos conocer las condiciones estándar de temperatura y presión (CEPT), que son 0 °C (273.15 K) y 1 atm (101.325 kPa).\\
\\
El volumen de gas a las condiciones estándar se puede calcular utilizando la siguiente ecuación:\\
\\
V = nRT/P\\
\\
Donde V es el volumen en litros, n es el número de moles, R es la constante de los gases ideales (0.0821 L atm/mol K), T es la temperatura en Kelvin y P es la presión en atm.\\
\\
Para el oxígeno consumido, podemos calcular su volumen utilizando:\\
\\
V = nO2RT/P = 2.03 mol x 0.0821 L atm/mol K x 273.15 K / 1 atm = 44.9 L\\
Por lo tanto, se consumen aproximadamente 44.9 litros de oxígeno en la reacción.\\
\textbf{Respuesta Número 34} \\
La síntesis de sulfato de aluminio se puede representar por la siguiente ecuación química:\\
2 Al + 3 H2SO4 → Al2(SO4)3 + 3 H2\\
Podemos ver que se necesitan 2 moles de aluminio para producir 1 mol de sulfato de aluminio. Utilizaremos esta relación estequiométrica para calcular el porcentaje de aluminio en la muestra.\\
Primero, necesitamos calcular la cantidad de aluminio presente en la muestra. Suponiendo que todo el aluminio en la muestra se convierte en sulfato de aluminio, la cantidad de aluminio en la muestra es igual a la cantidad de sulfato de aluminio producido. Por lo tanto, la cantidad de aluminio en la muestra es de:\\
17.10 g de Al2(SO4)3 x (2 moles de Al / 1 mol de Al2(SO4)3) x (26.98 g/mol de Al) = 2.31 g de Al\\
Por lo tanto, la muestra contenía 2.31 g de aluminio.\\
El porcentaje de aluminio en la muestra se puede calcular utilizando la siguiente fórmula:\\
\% de Al = (masa de Al en la muestra / masa total de la muestra) x 100%\\
\% de Al = (2.31 g / 5 g) x 100% = 46.2%\\
Por lo tanto, la muestra contenía aproximadamente un 46.2\% de aluminio.\\
\textbf{Respuesta Número 35} \\
a) La reacción química ajustada estequiométricamente es:\\
Pb + 2HNO3 → Pb(NO3)2 + NO2 + H2O\\
Podemos ver que se necesitan 1 mol de plomo y 2 moles de ácido nítrico para producir 1 mol de nitrato de plomo, 1 mol de dióxido de nitrógeno y 1 mol de agua.\\
\\
b) Primero, necesitamos calcular la cantidad de plomo presente en la muestra que se utilizó en la reacción. Suponiendo que todo el plomo en la muestra se convierte en nitrato de plomo, la cantidad de plomo en la muestra es igual a la cantidad de nitrato de plomo producido. Por lo tanto, la cantidad de plomo en la muestra es de:\\
29 g de Pb(NO3)2 x (1 mol de Pb / 1 mol de Pb(NO3)2) x (207.2 g/mol de Pb) = 3.47 g de Pb\\
Por lo tanto, la muestra contenía 3.47 g de plomo.\\
El porcentaje de plomo en la muestra se puede calcular utilizando la siguiente fórmula:\\
\% de Pb = (masa de Pb en la muestra / masa total de la muestra) x 100%\\
\% de Pb = (3.47 g / 40 g) x 100\% = 8.67\%\\
\\
Por lo tanto, el contenido porcentual de plomo en la muestra es del 8.67%.\\
\end{document}
