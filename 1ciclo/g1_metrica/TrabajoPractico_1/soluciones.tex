


% Document class and basic setup
\documentclass{article}
\usepackage[margin=1cm]{geometry} % set margins
\usepackage{lipsum} % for generating dummy text
\usepackage{amsmath}


% Title and author information
\title{Soluciones TP1}
\author{Andres Imlauer}

% Date and pagination
\date{\today}
\pagenumbering{gobble} % suppress page numbering

\begin{document}
% Title page
\maketitle
\thispagestyle{empty} % suppress page numbering on title page

\pagenumbering{arabic} % start page numbering

% Sections with Lorem Ipsum text
¿Cuántas rectas pasan por tres puntos no alineados M, N y P , tomándolos de a dos? \\
- Existen tres rectas distintas que pasan por tres puntos no alineados M, N y P, tomándolos de a dos. Cada recta se forma al unir dos de los tres puntos. Por ejemplo, la recta que pasa por M y N, la recta que pasa por M y P, y la recta que pasa por N y P.

¿Cuántas semirrectas quedan determinadas por los puntos M, N y P? \\
Existen un total de seis semirrectas que quedan determinadas por los puntos M, N y P. Cada una de las tres rectas formadas por los puntos M, N y P, tienen dos semirrectas asociadas: una que comienza en uno de los puntos y se extiende en una dirección determinada, y otra que comienza en el otro punto y se extiende en la dirección opuesta. Por lo tanto, hay un total de seis semirrectas determinadas por los puntos M, N y P.\\




{\bf Menciones los segmentos determinados.} \\
- Si los puntos M, N y P no están alineados, entonces se pueden formar tres segmentos distintos al unir dos de los tres puntos. Por lo tanto, hay un total de tres segmentos determinados por los puntos M, N y P no alineados. Cada segmento tiene una longitud finita y se puede representar como una línea recta que une dos puntos extremos. \\
Para determinar los segmentos formados por los puntos M, N y P, es necesario conocer la posición relativa de estos puntos. Si los tres puntos no están alineados, entonces se pueden formar tres segmentos distintos al unir dos de los tres puntos. Los segmentos se pueden nombrar según los puntos extremos que los definen. Por ejemplo, si M, N y P son tres puntos no alineados, entonces los segmentos formados son MN, MP y NP. Es importante destacar que un segmento es una porción finita de una recta, por lo que no se pueden formar segmentos con los puntos M, N y P si estos tres puntos están alineados. \\

Los puntos M, N y P no alineados determinan seis semirrectas, que son las siguientes: \\

1. La semirrecta que comienza en M y se extiende hacia N.\\
2. La semirrecta que comienza en M y se extiende hacia P.\\
3. La semirrecta que comienza en N y se extiende hacia M.\\
4. La semirrecta que comienza en N y se extiende hacia P.\\
5. La semirrecta que comienza en P y se extiende hacia M.\\
6. La semirrecta que comienza en P y se extiende hacia N.\\

Cada semirrecta se define por su punto de origen y la dirección en la que se extiende. En este caso, los puntos de origen son M, N y P, y la dirección se determina por los otros dos puntos. Por lo tanto, hay seis semirrectas diferentes que se pueden trazar a partir de estos tres puntos no alineados.

3. Tres amigos que viven en Posadas, El Dorado y Oberá y deciden quedar en un punto que esté a la misma distancia de sus tres casas. ¿Cómo calcular el lugar de la cita? ¿Cómo se llama en matemáticas ese punto?  \\

- El punto que está a la misma distancia de las tres casas se llama "circuncentro". Para calcular su ubicación, se puede trazar la mediatriz de cada par de segmentos que conectan las casas de dos amigos diferentes. El punto donde se intersectan las tres mediatrices es el circuncentro. \\


Para calcular el lugar de la cita, podemos utilizar el concepto de circuncentro, que es el centro de la circunferencia circunscrita al triángulo formado por los tres puntos (en este caso, la s tres ciudades). El circuncentro es el punto equidistante de los tres puntos. \\

Para encontrar el circuncentro, podemos seguir los siguientes pasos:\\
1. Dibujar los tres puntos en un plano cartesiano y trazar los segmentos de línea que los conectan para formar el triángulo.\\
2. Encontrar las coordenadas del punto medio de cada uno de los la dos del triángulo.\\
3. Encontrar las ecuaciones de las rectas perpendiculares a cada lado del triángulo que pasan por su punto medio.\\
4. Encontrar el punto de intersección de dos de estas rectas perpendiculares. Este punto es el circuncentro.\\
\\
El lugar de la cita sería el circuncentro del triángulo formado por las ciudades de Posadas, El Dorado y Oberá.\\
\\
Es importante tener en cuenta que en algunos casos, el circuncentro puede estar fuera del triángulo, por lo que la cita podría ser en un lugar que no se encuentre dentro de las tres ciudades.\\



4) Si en un terreno rectangular de 20 m por 40 m se ata un perro a un poste con una soga de 8 m de largo, ¿cuál es la zona del terreno por la que el perro puede corretear? ¿existe una única respuesta? \\ \\
La zona por la que el perro puede corretear es un sector circular de radio 8 m, centrado en el poste al que está atado. Este sector circular cubre un ángulo de 45 grados del centro del terreno, y su área es de aproximadamente 50,27 metros cuadrados.\\
\\
Sin embargo, si se cambia la ubicación del poste, la zona por la que el perro puede corretear también cambiará. Por lo tanto, no hay una única respuesta a esta pregunta ya que la ubicación del poste no se especifica.\\

5) Dos ardillas situadas en los puntos A y B corren en línea recta para el lado del arroyo, y en un determinado momento se encuentran. Si salen en el mismo instante y van a la misma velocidad (significa que recorren igual distancia en igual tiempo) , ¿Dónde tendrían que estar los lugares donde las ardillas se encuentran? ¿Por qué? Escriban la respuesta y realicen el dibujo correspondiente en el esquema de abajo \\ \\

Si las ardillas corren en línea recta hacia el lado del arroyo, entonces los lugares donde se encuentran están en la bisectriz del ángulo formado por las direcciones de las ardillas en el momento en que se encuentran. Esto se debe a que las ardillas corren a la misma velocidad y en línea recta, por lo que la distancia que cada una recorre es proporcional al tiempo transcurrido. Si trazamos las rectas que representan la dirección de cada ardilla en el momento en que se encuentran, entonces la bisectriz de ese ángulo es el lugar donde se encontrarán en cualquier otro punto del camino.

Si las ardillas salen al mismo tiempo y corren a la misma velocidad, el punto medio de la línea recta que une los puntos A y B es el lugar donde se encontrarán. Esto se debe a que cada ardilla recorrerá la mitad de la distancia total hacia el punto medio, y ambas se encontrarán allí al mismo tiempo. 

6) Graficar con los eelementos correspondientes. \\
a.Cuáles son las posiciones relativas de una circunferencia y Una recta.\\
Otra circunferencia.  \\
 \\
- Existen tres posibilidades para la posición relativa de una circunferencia y una recta: \\
 \\
1. La circunferencia y la recta no se intersectan: en este caso, la recta es exterior a la circunferencia o la circunferencia es exterior a la recta. \\
 \\
2. La circunferencia y la recta se intersectan en dos puntos: en este caso, la recta es secante a la circunferencia. \\
 \\
3. La circunferencia y la recta se intersectan en un único punto: \\
en este caso, la recta es tangente a la circunferencia en ese punto. \\
 \\
La posición relativa de una circunferencia y una recta depende del lugar donde se encuentren. Pueden ser tangentes, secantes o no intersecantes. \\

La posición relativa de dos circunferencias también depende del lugar donde se encuentren. Pueden ser tangentes externamente, tangentes internamente, secantes o no intersecantes. \\

b) Teniendo en cuenta el ítem anterior, establece cuál es la relación existente entre: La distancia entre la circunferencia y la recta con el radio de la circunferencia. La distancia entre las circunferencias y los radios de las mismas. \\

En el caso de una circunferencia y una recta, la distancia entre la circunferencia y la recta es igual al valor absoluto de
la diferencia entre el radio de la circunferencia y la distancia m
ínima entre la circunferencia y la recta. Es decir, si llamamos "r" al radio de la circunferencia y "d" a la distancia mínima entre
la circunferencia y la recta, entonces la distancia entre la circunferencia y la recta es $|r-d|$

En el caso de dos circunferencias, la distancia entre las circunferencias es igual al valor absoluto de la diferencia entre los radios de las circunferencias menos la distancia entre los centros de
las circunferencias. Es decir, si llamamos "r1" y "r2" a los radios de las circunferencias y "d" a la distancia entre los centros de las circunferencias, entonces la distancia entre las circunferencias es $|r1 - r2 - d|$.

7) Dada una recta r y un punto A exterior, traza la circunferencia con centro en el punto A, que es tangente a la recta r. ¿Qué radio tiene? \\

Para trazar la circunferencia con centro en el punto A y que sea tangente a la recta r, se debe seguir los siguientes pasos: \\
 \\
1. Desde el punto A, trazar una recta perpendicular a la recta r. Esta recta se denominará s y será la recta soporte de la circunferencia. \\
 \\
2. Tomar el radio de la circunferencia como la distancia entre la recta r y la recta s. \\
 \\
3. Desde el punto A, trazar la circunferencia con centro en el punto A y radio igual a la distancia encontrada en el paso anterior. \\
 \\
La circunferencia obtenida será tangente a la recta r en un punto de intersección con la recta s. \\
 \\
Cabe destacar que, dado que el punto A se encuentra fuera de la recta r, siempre existirá una única circunferencia con centro en A y tangente a r. \\


El radio de la circunferencia con centro en el punto A y tangente a la recta r es igual a la distancia entre la recta r y la recta perpendicular a r que pasa por el punto A. Si llamamos "d" a la distancia entre la recta r y la recta perpendicular que pasa por el punto A, entonces el radio de la circunferencia es igual a "d". Por lo tanto, para calcular el radio de la circunferencia es necesario determinar primero la distancia "d". \\

8) Recabando información :\\
a) Define ángulo.\\
b) Realiza una red conceptual que muestre las distintas clasificaciones de los mismos.\\

a)  En geometría, un ángulo es la medida de la separación entre dos rayos que comparten un punto en común, llamado vértice. Los ángulos se miden en grados, y pueden ser agudos (menos de 90 grados), rectos (exactamente 90 grados), obtusos (entre 90 y 180 grados) o llanos (exactamente 180 grados).

b) - Por su medida: los ángulos pueden ser agudos (menos de 90 grados), rectos (exactamente 90 grados), obtusos (entre 90 y 180 grados) o llanos (exactamente 180 grados).
- Por su posición: los ángulos pueden ser adyacentes (comparten un lado y un vértice), consecutivos (adyacentes y no superpuestos), opuestos por el vértice (comparten solo el vértice) o complementarios (cuya suma es igual a 90 grados).
- Por su dirección: los ángulos pueden ser directos (giran en la misma dirección), rectilíneos (giran en direcciones opuestas y suman 180 grados) o verticales (se forman al cortar dos líneas rectas por una tercera línea).
- Por su amplitud: los ángulos pueden ser mayores o menores que otro ángulo de referencia.

14) Realiza un cuadro sinóptico, diagrama o red con la clasificación de los triángulos según sus lados y sus ángulos.
Clasificación de triángulos:

Según sus lados:

- Escaleno: todos los lados tienen diferentes longitudes.
- Isósceles: dos lados tienen la misma longitud y el tercero es diferente.
- Equilátero: todos los lados tienen la misma longitud.

Según sus ángulos:

- Acutángulo: todos los ángulos internos miden menos de 90 grados.
- Rectángulo: uno de los ángulos internos mide exactamente 90 grados.
- Obtusángulo: uno de los ángulos internos mide más de 90 grados.



15) Los puntos M y N están a 7 cm y son los vértices de un triángulo. Halla un punto H que este a 3 cm de M y a 5 cm de N a la vez. Dibuja el triángulo. \\
Para encontrar el punto H, podemos utilizar la construcción clásica de la circunferencia que pasa por dos puntos dados y tiene un radio dado. En este caso, trazamos dos circunferencias, una con centro en M y radio 3 cm, y otra con centro en N y radio 5 cm. Estas dos circunferencias se intersectan en dos puntos, pero solo uno de ellos está a una distancia de 7 cm de M y N, que es el punto H que buscamos.\\
\\
Para dibujar el triángulo, trazamos los segmentos MH y NH desde el punto H hasta\\
los vértices M y N, respectivamente. El triángulo resultante es el triángulo MHN.\\

16) Responder justificando.\\
¿Será verdad que:\\
a) Todos los triángulos equiláteros son isósceles?\\
b) Algunos triángulos pueden tener un ángulo obtuso y uno recto?\\
c) Ningún triángulo puede ser isósceles y rectángulo?\\
d) Los ángulos de cualquier triángulo equilátero siempre son iguales?\\

a) Sí, es cierto que todos los triángulos equiláteros son isósceles.

Un triángulo equilátero es aquel que tiene los tres lados iguales entre sí, lo que implica que los tres ángulos internos también son iguales a 60 grados.

Por otro lado, un triángulo isósceles es aquel que tiene al menos dos lados iguales entre sí. En el caso del triángulo equilátero, al tener tres lados iguales, también cumple con la condición de tener al menos dos lados iguales, por lo que es isósceles.

Por lo tanto, todo triángulo equilátero es también isósceles, pero no todos los triángulos isósceles son equiláteros.\\
\\
b) No, un triángulo no puede tener un ángulo obtuso y uno recto al mismo tiempo. La suma de los ángulos internos de un triángulo es siempre igual a 180 grados, por lo que si un ángulo es recto (90 grados) y otro es obtuso (más de 90 grados), el tercer ángulo tendría que ser menor a cero grados, lo cual no es posible.

c) Falso. Un triángulo sí puede ser isósceles y rectángulo al mismo tiempo. Este tipo de triángulo se llama triángulo isósceles rectángulo y tiene dos lados iguales (los catetos) y un ángulo recto opuesto a la hipotenusa (el lado más largo).

d) Sí, los ángulos de cualquier triángulo equilátero siempre
son iguales. Como un triángulo equilátero tiene tres lados iguales, también tiene tres ángulos iguales. Cada ángulo interno de un triángulo equilátero mide 60 grados.

17) Contesta justificando:\\
a) ¿Cuántos ángulos obtusos puede tener un triángulo? ¿Por qué?\\
b) ¿Un triángulo Puede ser obtusángulo y rectángulo a la vez? ¿Por qué?\\
c) ¿Puede tener un triángulo dos ángulos rectos? ¿Por qué?\\
d) ¿Un triángulo puede ser rectángulo e isósceles?\\
\\
a) Un triángulo puede tener como máximo un ángulo obtuso. Un
ángulo obtuso es aquel que mide más de 90 grados, y la suma de los ángulos internos de un triángulo siempre es igual a 180 grados. Si un triángulo tuviera dos ángulos obtusos, la suma de los ángulos internos excedería los 180 grados, lo cual es imposible. Por lo tanto, un triángulo puede tener un ángulo obtuso como máximo, y los otros dos ángulos deben ser agudos (menores a 90 grados) para que la suma de los ángulos internos sea igual a 180 grados.

b) No, un triángulo no puede ser obtusángulo y rectángulo al mismo tiempo. Un triángulo obtusángulo tiene un ángulo obtuso, es decir, que mide más de 90 grados. Por otro lado, un triángulo rectángulo tiene un ángulo recto, es decir, que mide exactamente 90 grados. Como la suma de los ángulos internos de un triángulo es siempre de 180 grados, si un triángulo tuviera un ángulo obtuso y un ángulo recto, el tercer ángulo debería ser de 90 grados, lo cual es imposible en un triángulo obtusángulo, donde todos los ángulos son obtusos. Por lo tanto, un triángulo no puede ser obtusángulo y rectángulo al mismo tiempo.

c) No, un triángulo no puede tener dos ángulos rectos. La suma de los ángulos internos de un triángulo es siempre igual a 180 grados. Si un triángulo tuviera dos ángulos rectos, cada uno de ellos mediría 90 grados y la suma total de los ángulos internos sería de 270 grados, lo cual es imposible.

d) Sí, un triángulo puede ser rectángulo e isósceles al mismo tiempo. Un triángulo rectángulo e isósceles tiene un ángulo recto y dos ángulos iguales, lo que significa que la base del triángulo se divide en dos partes iguales. Si dibujas la altura desde el vértice del ángulo recto hasta la hipotenusa, se divide en dos segmentos iguales también, lo que hace que el triángulo rectángulo e isósceles tenga una serie de propiedades interesantes y útiles en geometría.

18) Construye un triángulo, sabiendo que:\\
a) dos lados miden 4 cm y 2 cm, y el ángulo comprendido entre ellos es de 60°.\\
b) dos lados miden 6 cm y el ángulo comprendido entre ellos es de 75°.\\
c) un lado mide 8 cm y los ángulos adyacentes a él son de 45° y 65°.\\
d) un lado mide 4 cm y los ángulos adyacentes a él son de 120° y 55°.\\

19) Dados dos ángulos de 45° y 60°. ¿Puedes dibujar dos triángulos distintos? ¿Cuántos se pueden construir? \\ 
\\
En cuanto a cuántos triángulos se pueden construir con dos ángulos dados, en general es posible construir un único triángulo si conocemos dos ángulos y un lado, o dos lados y un ángulo opuesto a uno de ellos. Pero si conocemos únicamente dos ángulos, entonces es posible construir infinitos triángulos diferentes, ya que podemos variar la longitud del tercer lado y obtener triángulos con propiedades distintas. \\

20) Calcular el valor de los ángulos de los siguientes triángulos. Graficar con las medidas halladas\\

a) $\alpha = 3x+20°$ , 
$\gamma = 3x+10°$ ,
$\beta = 40°$ 
\\ 
La suma de los ángulos internos de un triángulo es siempre igual a 180°. Podemos usar esta propiedad para encontrar el valor de x y luego calcular el valor de $\alpha$ y $\gamma$.\\
\\

\[
\alpha + \beta + \gamma = 180° \\
(3x + 20°) + 40° + (3x + 10°) = 180°\\
6x + 70° = 180°\\
6x = 110°\\
x = 18.33°
\]
Ahora que conocemos el valor de x, podemos calcular $\alpha$ y $\gamma$:\\

\begin{equation}
\begin{array}{l}
\alpha = 3x + 20 = 3(18.33) + 20 = 74.99 \\
\gamma = 3x + 10 = 3(18.33) + 10 = 64.99
\end{array}
\end{equation}

Por lo tanto, los ángulos del triángulo son 
\[
\alpha \approx 75°, \beta = 40° , \gamma \approx 65°.
\]
\\
b)  
\[
\pi = 5x-10 ,  \alpha = 2x+16°, \beta=90°
\]
La suma de los ángulos internos de un triángulo es siempre igual a 180°. Podemos usar esta propiedad para encontrar el valor de x y luego calcular el valor de $\pi$ y $\alpha$.\\
\\
$\beta$ es un ángulo recto, por lo que su medida es 90°.\\
\\
\begin{equation}
\begin{array}{l}
\pi + \alpha + \beta = 180°\\
(5x - 10) + (2x + 16°) + 90° = 180°\\
7x + 96° = 180°\\
7x = 84°\\
x = 12°
\end{array}
\end{equation}
\\
Ahora que conocemos el valor de x, podemos calcular $\pi$ y $\alpha$:\\
\\
\begin{equation}
\begin{array}{l}
\pi = 5x - 10 = 5(12°) - 10 = 50°\\
\alpha = 2x + 16° = 2(12°) + 16° = 40°\\
\end{array}
\end{equation}
\\
\\
Por lo tanto, los ángulos del triángulo son $\pi = 50°, \alpha = 40°\ y\ \beta = 90°$.\\
\\
21)Responder justificando.\\
a) es cierto que se puede hacer un triángulo cuyos lados midan 10 cm, 3 cm y 4cm?\\
\\
Sí, es posible hacer un triángulo con lados de 10 cm, 3 cm y 4 cm. Este triángulo es un\\
triángulo escaleno, es decir, un triángulo con todos los lados de diferentes longitudes.\\
b) es cierto que se hacer un triángulo cuyos lados midan 5 cm, 6 cm y 9cm?\\
No es posible construir un triángulo con lados de medidas 5 cm, 6 cm y 9 cm, ya que la suma de las medidas de los dos lados más cortos (5 cm + 6 cm) es menor que la medida del lado más largo (9 cm), lo que contradice la desigualdad triangular.\\
\\
c) es cierto que se construir un único triángulo sabiendo que un lado mide 3cm y el otro 5 cm?\\
No, no es posible construir un único triángulo con solo conocer las medidas de dos de sus lados. Para poder construir un triángulo único, es necesario conocer la medida de al menos un\\
lado más o el valor de algún ángulo. En este caso, la medida de un tercer lado o un ángulo adicional sería necesario para determinar completamente el triángulo.\\
\\





26) Busca en las distintas bibliografías:\\
a) Definición de polígono y sus elementos.\\
b) Clasificación.\\
\\
a) Un polígono es una figura geométrica plana formada por una serie de segmentos de línea recta que se unen en diferentes puntos para formar una figura cerrada. Los elementos de un polígono son:\\
\\
1. Lados: son los segmentos de línea recta que forman las aristas del polígono.\\
\\
2. Vértices: son los puntos donde se unen dos lados del polígono.\\
\\
3. Ángulos: son las regiones que se forman en los vértices del polígono donde se encuentran dos lados.\\
\\
4. Diagonales: son los segmentos de línea recta que conectan dos vértices no adyacentes del polígono.\\
\\
5. Perímetro: es la suma de las longitudes de todos los lados del polígono.\\
\\
6. Área: es la medida de la región encerrada por el polígono, que depende del número de lados, la longitud de los lados y los ángulos entre ellos.\\

b) Clasificacion
Los polígonos se clasifican según el número de lados que tienen. Algunas de las clasificaciones más comunes son:\\
\\
- Triángulo: polígono con tres lados.\\
- Cuadrilátero: polígono con cuatro lados.\\
- Pentágono: polígono con cinco lados.\\
- Hexágono: polígono con seis lados.\\
- Heptágono: polígono con siete lados.\\
- Octágono: polígono con ocho lados.\\
- Nonágono: polígono con nueve lados.\\
- Decágono: polígono con diez lados.\\
- Polígono regular: un polígono con todos los lados y ángulos iguales.\\
- Polígono irregular: un polígono con lados y ángulos de diferentes longitudes y medidas.\\
- Polígono convexo: un polígono en el que todos los ángulos interiores son menores que 180 grados y todas las diagonales están dentro del polígono.\\
- Polígono cóncavo: un polígono en el que al menos un ángulo interior es mayor o igual a 180 grados o al menos una diagonal está fuera del polígono.\\



\end{document}
