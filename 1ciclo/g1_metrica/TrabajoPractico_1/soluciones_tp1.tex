
% Document class and basic setup
\documentclass{article}
\usepackage[margin=1cm]{geometry} % set margins
\usepackage{lipsum} % for generating dummy text
\usepackage{amsmath}
\usepackage{hyperref} % Importa el paquete hyperref para poder usar el comando \href


% Title and author information
\title{Soluciones TP1}
\author{Andres Imlauer}

% Date and pagination
\date{\today}
\pagenumbering{gobble} % suppress page numbering

\begin{document}
% Title page
\maketitle
\thispagestyle{empty} % suppress page numbering on title page

\pagenumbering{arabic} % start page numbering

% Sections with Lorem Ipsum text
{\bf 1. ¿Cuántas rectas pasan por tres puntos no alineados M, N y P , tomándolos de a dos?} \\
- Existen tres rectas distintas que pasan por tres puntos no alineados M, N y P, tomándolos de a dos. Cada recta se forma al unir dos de los tres puntos. Por ejemplo, la recta que pasa por M y N, la recta que pasa por M y P, y la recta que pasa por N y P.

{\bf ¿Cuántas semirrectas quedan determinadas por los puntos M, N y P? } \\
Existen un total de seis semirrectas que quedan determinadas por los puntos M, N y P. Cada una de las tres rectas formadas por los puntos M, N y P, tienen dos semirrectas asociadas: una que comienza en uno de los puntos y se extiende en una dirección determinada, y otra que comienza en el otro punto y se extiende en la dirección opuesta. Por lo tanto, hay un total de seis semirrectas determinadas por los puntos M, N y P.\\



{\bf Menciones los segmentos determinados.} \\
- Si los puntos M, N y P no están alineados, entonces se pueden formar tres segmentos distintos al unir dos de los tres puntos. Por lo tanto, hay un total de tres segmentos determinados por los puntos M, N y P no alineados. Cada segmento tiene una longitud finita y se puede representar como una línea recta que une dos puntos extremos. \\
Para determinar los segmentos formados por los puntos M, N y P, es necesario conocer la posición relativa de estos puntos. Si los tres puntos no están alineados, entonces se pueden formar tres segmentos distintos al unir dos de los tres puntos. Los segmentos se pueden nombrar según los puntos extremos que los definen. Por ejemplo, si M, N y P son tres puntos no alineados, entonces los segmentos formados son MN, MP y NP. Es importante destacar que un segmento es una porción finita de una recta, por lo que no se pueden formar segmentos con los puntos M, N y P si estos tres puntos están alineados. \\

Los puntos M, N y P no alineados determinan seis semirrectas, que son las siguientes: \\

1. La semirrecta que comienza en M y se extiende hacia N.\\
2. La semirrecta que comienza en M y se extiende hacia P.\\
3. La semirrecta que comienza en N y se extiende hacia M.\\
4. La semirrecta que comienza en N y se extiende hacia P.\\
5. La semirrecta que comienza en P y se extiende hacia M.\\
6. La semirrecta que comienza en P y se extiende hacia N.\\

Cada semirrecta se define por su punto de origen y la dirección en la que se extiende. En este caso, los puntos de origen son M, N y P, y la dirección se determina por los otros dos puntos. Por lo tanto, hay seis semirrectas diferentes que se pueden trazar a partir de estos tres puntos no alineados.


{\bf 3. Tres amigos que viven en Posadas, El Dorado y Oberá y deciden quedar en un punto que esté a la misma distancia de sus tres casas. ¿Cómo calcular el lugar de la cita? ¿Cómo se llama en matemáticas ese punto? } \\

- El punto que está a la misma distancia de las tres casas se llama "circuncentro". Para calcular su ubicación, se puede trazar la mediatriz de cada par de segmentos que conectan las casas de dos amigos diferentes. El punto donde se intersectan las tres mediatrices es el circuncentro. \\


Para calcular el lugar de la cita, podemos utilizar el concepto de circuncentro, que es el centro de la circunferencia circunscrita al triángulo formado por los tres puntos (en este caso, la s tres ciudades). El circuncentro es el punto equidistante de los tres puntos. \\

Para encontrar el circuncentro, podemos seguir los siguientes pasos:\\
1. Dibujar los tres puntos en un plano cartesiano y trazar los segmentos de línea que los conectan para formar el triángulo.\\
2. Encontrar las coordenadas del punto medio de cada uno de los la dos del triángulo.\\
3. Encontrar las ecuaciones de las rectas perpendiculares a cada lado del triángulo que pasan por su punto medio.\\
4. Encontrar el punto de intersección de dos de estas rectas perpendiculares. Este punto es el circuncentro.\\
\\
El lugar de la cita sería el circuncentro del triángulo formado por las ciudades de Posadas, El Dorado y Oberá.\\
\\
Es importante tener en cuenta que en algunos casos, el circuncentro puede estar fuera del triángulo, por lo que la cita podría ser en un lugar que no se encuentre dentro de las tres ciudades.\\



{\bf 4) Si en un terreno rectangular de 20 m por 40 m se ata un perro a un poste con una soga de 8 m de largo, ¿cuál es la zona del terreno por la que el perro puede corretear? ¿existe una única respuesta? } \\ \\

El perro puede corretear en una zona circular alrededor del poste donde está atado. El radio del círculo es igual a la longitud de la soga, que es de 8 m. Por lo tanto, el área del círculo es $A = \pi r^2 = \pi(8^2) = 64\pi$ metros cuadrados (aproximadamente 201.06 metros cuadrados). \\ \\
Esta es la zona en la que el perro puede corretear.
\\
Sin embargo, si se cambia la ubicación del poste, la zona por la que el perro puede corretear también cambiará. Por lo tanto, no hay una única respuesta a esta pregunta ya que la ubicación del poste no se especifica.\\

{\bf 5) Dos ardillas situadas en los puntos A y B corren en línea recta para el lado del arroyo, y en un determinado momento se encuentran. Si salen en el mismo instante y van a la misma velocidad (significa que recorren igual distancia en igual tiempo) , ¿Dónde tendrían que estar los lugares donde las ardillas se encuentran? ¿Por qué? Escriban la respuesta y realicen el dibujo correspondiente en el esquema de abajo } \\ \\

Si las ardillas salen al mismo tiempo y corren a la misma velocidad, el punto medio de la línea recta que une los puntos A y B es el lugar donde se encontrarán. Esto se debe a que cada ardilla recorrerá la mitad de la distancia total hacia el punto medio, y ambas se encontrarán allí al mismo tiempo. Es decir se encontraran en la mediatriz del segmento que une los puntos A y B.

{\bf 6) Graficar con los eelementos correspondientes. } \\
a.Cuáles son las posiciones relativas de una circunferencia y Una recta.\\
Otra circunferencia.  \\
 \\
- Existen tres posibilidades para la posición relativa de una circunferencia y una recta: \\
 \\
1. La circunferencia y la recta no se intersectan: en este caso, la recta es exterior a la circunferencia o la circunferencia es exterior a la recta. \\
 \\
2. La circunferencia y la recta se intersectan en dos puntos: en este caso, la recta es secante a la circunferencia. \\
 \\
3. La circunferencia y la recta se intersectan en un único punto: \\
en este caso, la recta es tangente a la circunferencia en ese punto. \\
 \\
La posición relativa de una circunferencia y una recta depende del lugar donde se encuentren. Pueden ser tangentes, secantes o no intersecantes. \\

La posición relativa de dos circunferencias también depende del lugar donde se encuentren. Pueden ser tangentes externamente, tangentes internamente, secantes o no intersecantes. \\

b) Teniendo en cuenta el ítem anterior, establece cuál es la relación existente entre: La distancia entre la circunferencia y la recta con el radio de la circunferencia. La distancia entre las circunferencias y los radios de las mismas. \\

En el caso de una circunferencia y una recta, la distancia entre la circunferencia y la recta es igual al valor absoluto de
la diferencia entre el radio de la circunferencia y la distancia m
ínima entre la circunferencia y la recta. Es decir, si llamamos "r" al radio de la circunferencia y "d" a la distancia mínima entre
la circunferencia y la recta, entonces la distancia entre la circunferencia y la recta es $|r-d|$

En el caso de dos circunferencias, la distancia entre las circunferencias es igual al valor absoluto de la diferencia entre los radios de las circunferencias menos la distancia entre los centros de
las circunferencias. Es decir, si llamamos "r1" y "r2" a los radios de las circunferencias y "d" a la distancia entre los centros de las circunferencias, entonces la distancia entre las circunferencias es $|r1 - r2 - d|$.

{\bf 7) Dada una recta r y un punto A exterior, traza la circunferencia con centro en el punto A, que es tangente a la recta r. ¿Qué radio tiene? } \\

Para trazar la circunferencia con centro en el punto A y que sea tangente a la recta r, se debe seguir los siguientes pasos: \\
 \\
1. Desde el punto A, trazar una recta perpendicular a la recta r. Esta recta se denominará s y será la recta soporte de la circunferencia. \\
 \\
2. Tomar el radio de la circunferencia como la distancia entre la recta r y la recta s. \\
 \\
3. Desde el punto A, trazar la circunferencia con centro en el punto A y radio igual a la distancia encontrada en el paso anterior. \\
 \\
La circunferencia obtenida será tangente a la recta r en un punto de intersección con la recta s. \\
 \\
Cabe destacar que, dado que el punto A se encuentra fuera de la recta r, siempre existirá una única circunferencia con centro en A y tangente a r. \\


El radio de la circunferencia con centro en el punto A y tangente a la recta r es igual a la distancia entre la recta r y la recta perpendicular a r que pasa por el punto A. Si llamamos "d" a la distancia entre la recta r y la recta perpendicular que pasa por el punto A, entonces el radio de la circunferencia es igual a "d". Por lo tanto, para calcular el radio de la circunferencia es necesario determinar primero la distancia "d". \\

Video sobre como realizarlo: \href{https://youtu.be/watch?v=2oCtAhCS8uI}{https://youtu.be/watch?v=2oCtAhCS8uI}

{\bf 8) Recabando información: } \\
a) Define ángulo.\\
b) Realiza una red conceptual que muestre las distintas clasificaciones de los mismos.\\

a)  En geometría, un ángulo es la medida de la separación entre dos rayos que comparten un punto en común, llamado vértice. Los ángulos se miden en grados, y pueden ser agudos (menos de 90 grados), rectos (exactamente 90 grados), obtusos (entre 90 y 180 grados) o llanos (exactamente 180 grados).\\
\\
b) - Por su medida: los ángulos pueden ser agudos (menos de 90 grados), rectos (exactamente 90 grados), obtusos (entre 90 y 180 grados) o llanos (exactamente 180 grados).\\
- Por su posición: los ángulos pueden ser adyacentes (comparten un lado y un vértice), consecutivos (adyacentes y no superpuestos), opuestos por el vértice (comparten solo el vértice) o complementarios (cuya suma es igual a 90 grados).\\
- Por su dirección: los ángulos pueden ser directos (giran en la misma dirección), rectilíneos (giran en direcciones opuestas y suman 180 grados) o verticales (se forman al cortar dos líneas rectas por una tercera línea).\\
- Por su amplitud: los ángulos pueden ser mayores o menores que otro ángulo de referencia.\\
\\
{\bf 14) Realiza un cuadro sinóptico, diagrama o red con la clasificación de los triángulos según sus lados y sus ángulos.}\\
Clasificación de triángulos:\\
\\
Según sus lados:\\
\\
- Escaleno: todos los lados tienen diferentes longitudes.\\
- Isósceles: dos lados tienen la misma longitud y el tercero es diferente.\\
- Equilátero: todos los lados tienen la misma longitud.\\
\\
Según sus ángulos:\\
\\
- Acutángulo: todos los ángulos internos miden menos de 90 grados.\\
- Rectángulo: uno de los ángulos internos mide exactamente 90 grados.\\
- Obtusángulo: uno de los ángulos internos mide más de 90 grados.\\
\\
\\

{\bf 15) Los puntos M y N están a 7 cm y son los vértices de un triángulo. Halla un punto H que este a 3 cm de M y a 5 cm de N a la vez. Dibuja el triángulo. } \\
Para encontrar el punto H, podemos utilizar la construcción clásica de la circunferencia que pasa por dos puntos dados y tiene un radio dado. En este caso, trazamos dos circunferencias, una con centro en M y radio 3 cm, y otra con centro en N y radio 5 cm. Estas dos circunferencias se intersectan en dos puntos, pero solo uno de ellos está a una distancia de 7 cm de M y N, que es el punto H que buscamos.\\
\\
Para dibujar el triángulo, trazamos los segmentos MH y NH desde el punto H hasta\\
los vértices M y N, respectivamente. El triángulo resultante es el triángulo MHN.\\

{\bf 16) Responder justificando. } \\
¿Será verdad que:\\
a) Todos los triángulos equiláteros son isósceles?\\
b) Algunos triángulos pueden tener un ángulo obtuso y uno recto?\\
c) Ningún triángulo puede ser isósceles y rectángulo?\\
d) Los ángulos de cualquier triángulo equilátero siempre son iguales?\\

a) Sí, es cierto que todos los triángulos equiláteros son isósceles.

Un triángulo equilátero es aquel que tiene los tres lados iguales entre sí, lo que implica que los tres ángulos internos también son iguales a 60 grados.

Por otro lado, un triángulo isósceles es aquel que tiene al menos dos lados iguales entre sí. En el caso del triángulo equilátero, al tener tres lados iguales, también cumple con la condición de tener al menos dos lados iguales, por lo que es isósceles.

Por lo tanto, todo triángulo equilátero es también isósceles, pero no todos los triángulos isósceles son equiláteros.\\
\\
b) No, un triángulo no puede tener un ángulo obtuso y uno recto al mismo tiempo. La suma de los ángulos internos de un triángulo es siempre igual a 180 grados, por lo que si un ángulo es recto (90 grados) y otro es obtuso (más de 90 grados), el tercer ángulo tendría que ser menor a cero grados, lo cual no es posible.

c) Falso. Un triángulo sí puede ser isósceles y rectángulo al mismo tiempo. Este tipo de triángulo se llama triángulo isósceles rectángulo y tiene dos lados iguales (los catetos) y un ángulo recto opuesto a la hipotenusa (el lado más largo).

d) Sí, los ángulos de cualquier triángulo equilátero siempre
son iguales. Como un triángulo equilátero tiene tres lados iguales, también tiene tres ángulos iguales. Cada ángulo interno de un triángulo equilátero mide 60 grados.

{\bf 17) Contesta justificando: }\\
a) ¿Cuántos ángulos obtusos puede tener un triángulo? ¿Por qué?\\
b) ¿Un triángulo Puede ser obtusángulo y rectángulo a la vez? ¿Por qué?\\
c) ¿Puede tener un triángulo dos ángulos rectos? ¿Por qué?\\
d) ¿Un triángulo puede ser rectángulo e isósceles?\\
\\
a) Un triángulo puede tener como máximo un ángulo obtuso. Un
ángulo obtuso es aquel que mide más de 90 grados, y la suma de los ángulos internos de un triángulo siempre es igual a 180 grados. Si un triángulo tuviera dos ángulos obtusos, la suma de los ángulos internos excedería los 180 grados, lo cual es imposible. Por lo tanto, un triángulo puede tener un ángulo obtuso como máximo, y los otros dos ángulos deben ser agudos (menores a 90 grados) para que la suma de los ángulos internos sea igual a 180 grados.

b) No, un triángulo no puede ser obtusángulo y rectángulo al mismo tiempo. Un triángulo obtusángulo tiene un ángulo obtuso, es decir, que mide más de 90 grados. Por otro lado, un triángulo rectángulo tiene un ángulo recto, es decir, que mide exactamente 90 grados. Como la suma de los ángulos internos de un triángulo es siempre de 180 grados, si un triángulo tuviera un ángulo obtuso y un ángulo recto, el tercer ángulo debería ser de 90 grados, lo cual es imposible en un triángulo obtusángulo, donde todos los ángulos son obtusos. Por lo tanto, un triángulo no puede ser obtusángulo y rectángulo al mismo tiempo.

c) No, un triángulo no puede tener dos ángulos rectos. La suma de los ángulos internos de un triángulo es siempre igual a 180 grados. Si un triángulo tuviera dos ángulos rectos, cada uno de ellos mediría 90 grados y la suma total de los ángulos internos sería de 270 grados, lo cual es imposible.

d) Sí, un triángulo puede ser rectángulo e isósceles al mismo tiempo. Un triángulo rectángulo e isósceles tiene un ángulo recto y dos ángulos iguales, lo que significa que la base del triángulo se divide en dos partes iguales. Si dibujas la altura desde el vértice del ángulo recto hasta la hipotenusa, se divide en dos segmentos iguales también, lo que hace que el triángulo rectángulo e isósceles tenga una serie de propiedades interesantes y útiles en geometría.

{\bf 18) Construye un triángulo, sabiendo que: }\\
a) dos lados miden 4 cm y 2 cm, y el ángulo comprendido entre ellos es de 60°.\\
b) dos lados miden 6 cm y el ángulo comprendido entre ellos es de 75°.\\
c) un lado mide 8 cm y los ángulos adyacentes a él son de 45° y 65°.\\
d) un lado mide 4 cm y los ángulos adyacentes a él son de 120° y 55°.\\

{\bf 19) Dados dos ángulos de 45° y 60°. ¿Puedes dibujar dos triángulos distintos? ¿Cuántos se pueden construir? } \\ 
\\
En cuanto a cuántos triángulos se pueden construir con dos ángulos dados, en general es posible construir un único triángulo si conocemos dos ángulos y un lado, o dos lados y un ángulo opuesto a uno de ellos. Pero si conocemos únicamente dos ángulos, entonces es posible construir infinitos triángulos diferentes, ya que podemos variar la longitud del tercer lado y obtener triángulos con propiedades distintas. \\

{\bf 20) Calcular el valor de los ángulos de los siguientes triángulos. Graficar con las medidas halladas } \\ 

a) $\alpha = 3x+20°$ , 
$\gamma = 3x+10°$ ,
$\beta = 40°$ 
\\ 
La suma de los ángulos internos de un triángulo es siempre igual a 180°. Podemos usar esta propiedad para encontrar el valor de x y luego calcular el valor de $\alpha$ y $\gamma$.\\
\\

\begin{equation}
\begin{array}{l}
\alpha + \beta + \gamma = 180° \\
(3x + 20°) + 40° + (3x + 10°) = 180°\\
6x + 70° = 180°\\
6x = 110°\\
x = 18.33°
\end{array}
\end{equation}
Ahora que conocemos el valor de x, podemos calcular $\alpha$ y $\gamma$:\\

\begin{equation}
\begin{array}{l}
\alpha = 3x + 20 = 3(18.33) + 20 = 74.99 \\
\gamma = 3x + 10 = 3(18.33) + 10 = 64.99
\end{array}
\end{equation}

Por lo tanto, los ángulos del triángulo son 
\[
\alpha \approx 75°, \beta = 40° , \gamma \approx 65°.
\]
\\
b)  
\[
\pi = 5x-10 ,  \alpha = 2x+16°, \beta=90°
\]
La suma de los ángulos internos de un triángulo es siempre igual a 180°. Podemos usar esta propiedad para encontrar el valor de x y luego calcular el valor de $\pi$ y $\alpha$.\\
\\
$\beta$ es un ángulo recto, por lo que su medida es 90°.\\
\\
\begin{equation}
\begin{array}{l}
\pi + \alpha + \beta = 180°\\
(5x - 10) + (2x + 16°) + 90° = 180°\\
7x + 96° = 180°\\
7x = 84°\\
x = 12°
\end{array}
\end{equation}
\\
Ahora que conocemos el valor de x, podemos calcular $\pi$ y $\alpha$:\\
\\
\begin{equation}
\begin{array}{l}
\pi = 5x - 10 = 5(12°) - 10 = 50°\\
\alpha = 2x + 16° = 2(12°) + 16° = 40°\\
\end{array}
\end{equation}
\\
\\
Por lo tanto, los ángulos del triángulo son $\pi = 50°, \alpha = 40°\ y\ \beta = 90°$.\\
\\
{\bf 21)Responder justificando }.\\
a) es cierto que se puede hacer un triángulo cuyos lados midan 10 cm, 3 cm y 4cm?\\
\\
Sí, es posible hacer un triángulo con lados de 10 cm, 3 cm y 4 cm. Este triángulo es un\\
triángulo escaleno, es decir, un triángulo con todos los lados de diferentes longitudes.\\
b) es cierto que se hacer un triángulo cuyos lados midan 5 cm, 6 cm y 9cm?\\
No es posible construir un triángulo con lados de medidas 5 cm, 6 cm y 9 cm, ya que la suma de las medidas de los dos lados más cortos (5 cm + 6 cm) es menor que la medida del lado más largo (9 cm), lo que contradice la desigualdad triangular.\\
\\
c) es cierto que se construir un único triángulo sabiendo que un lado mide 3cm y el otro 5 cm?\\
No, no es posible construir un único triángulo con solo conocer las medidas de dos de sus lados. Para poder construir un triángulo único, es necesario conocer la medida de al menos un\\
lado más o el valor de algún ángulo. En este caso, la medida de un tercer lado o un ángulo adicional sería necesario para determinar completamente el triángulo.\\
\\

{\bf 24) Constriuir el triangulo }: \\
\\
{\bf a) como hago para construir un triangulo escaleno en el cual uno de sus lados mida 8 cm y hallar el baricentro?} \\
\\
Para construir un triángulo escaleno en el cual uno de sus lados mida 8 cm, podemos seguir los siguientes pasos:\\
\\
1. Dibujamos una recta y marcamos un punto A en ella para representar el primer lado del triángulo de 8 cm.\\
2. Desde el punto A, trazamos un arco con centro en A y cualquier longitud para representar el segundo lado del triángulo.\\
3. Desde uno de los puntos donde el arco intersecta la recta, trazamos otro arco con cualquier longitud diferente para representar el tercer lado del triángulo.\\
4. Dibujamos las tres medianas del triángulo. Para ello, trazamos una línea desde cada vértice del triángulo al punto medio del lado opuesto.\\
5. Las tres medianas se intersectan en un punto llamado baricentro. Podemos encontrar el baricentro trazando las tres medianas y encontrando su punto de intersección.\\
\\
Para medir la ubicación del baricentro, podemos utilizar una regla o una herramienta de medición para medir la distancia desde cada vértice del triángulo al punto de intersección de las tres medianas. El punto donde estas tres distancias son iguales es el baricentro.\\
\\
\\
{\bf b) como hago para dibujar un triangulo obtusángulo donde el ángulo $\alpha=130\textdegree$ y en el hallar el ortocentro} \\
\\
\\
Para dibujar un triángulo obtusángulo con un ángulo de 130° y encontrar su ortocentro, podemos seguir los siguientes pasos:\\
\\
1. Dibujamos una línea recta y marcamos un punto A en un extremo, que será uno de los vértices del triángulo.\\
\\
2. Desde el punto A, trazamos otro segmento de línea en cualquier dirección para representar uno de los lados del triángulo.\\
\\
3. A continuación, trazamos otro segmento de línea que forme un ángulo de 130° con el lado anterior para representar el segundo lado del triángulo.\\
\\
4. Desde el punto donde se intersectan los dos segmentos de línea, trazamos una línea perpendicular al segundo lado del triángulo. Este punto de intersección será uno de los vértices del triángulo.\\
\\
5. Para encontrar el tercer vértice del triángulo, trazamos un arco con centro en el punto donde se intersectan los dos segmentos de línea y que pase por uno de los extremos del segundo lado.\\
\\
6. El tercer vértice del triángulo será uno de los puntos donde el arco intersecta la línea perpendicular trazada en el paso 4.\\
\\
7. Dibujamos las alturas del triángulo. Para ello, trazamos una línea perpendicular desde cada vértice del triángulo al lado opuesto.\\
\\
8. Las tres alturas se intersectan en un punto llamado ortocentro. Podemos encontrar el ortocentro trazando las tres alturas y encontrando su punto de intersección.\\
\\
Una vez encontrado el ortocentro, podemos medir la distancia desde el ortocentro a cada uno de los lados del triángulo. Estas distancias serán diferentes ya que el ortocentro no es equidistante a los tres lados del triángulo en un triángulo obtusángulo.\\
\\


{\bf c) como hago para dibujar un triangulo isósceles y en el hallar el incentro} \\
\\ 
Para dibujar un triángulo isósceles y encontrar su incentro, podemos seguir los siguientes pasos:\\
\\
1. Dibujamos una línea recta y marcamos un punto A en el centro de la línea para representar la\\
base del triángulo isósceles.\\
\\
2. Desde el punto A, trazamos dos arcos de igual longitud para representar los dos lados iguales del triángulo isósceles.\\
\\
3. Los dos arcos se intersectarán en dos puntos, marcamos estos puntos como B y C para representar los vértices del triángulo isósceles.\\
\\
4. Dibujamos las bisectrices de dos ángulos iguales del triángulo. Para ello, trazamos una línea desde cada vértice del triángulo hasta el punto medio del lado opuesto.\\
\\
5. Las bisectrices se intersectan en un punto llamado incentro. Podemos encontrar el incentro trazando las bisectrices y encontrando su punto de intersección.\\
\\
Una vez encontrado el incentro, podemos medir la distancia desde el incentro a cada uno de los lados del triángulo. Estas distancias serán iguales ya que el incentro es equidistante a los tres lados del triángulo.\\
\\
\\
{\bf d) como hago para dibujar un triangulo equilátero y en el hallar el circuncentro. }\\
Para dibujar un triángulo equilátero y encontrar su circuncentro, podemos seguir los siguientes pasos:\\
\\
1. Dibujamos una línea recta y marcamos un punto A en el centro de la línea para representar el primer vértice del triángulo equilátero.\\
2. Desde el punto A, trazamos un arco de radio R (la distancia deseada del vértice al centro del triángulo equilátero) para representar el segundo vértice del triángulo equilátero.\\
3. Repetimos el paso 2, trazando otro arco con centro en el segundo vértice y radio R para encontrar el tercer vértice del triángulo equilátero.\\
4. Conectamos los tres vértices del triángulo equilátero para terminar de dibujar el triángulo.\\
5. El circuncentro del triángulo equilátero es el punto donde las tres mediatrices de los lados del triángulo se intersectan.\\
6. Para encontrar el circuncentro, trazamos la mediatriz de cada uno de los lados del triángulo equilátero.\\
7. Las tres mediatrices se intersectan en un punto llamado circuncentro, que es el centro de la circunferencia circunscrita al triángulo equilátero.\\
8. Podemos encontrar el punto medio de cada lado del triángulo equilátero utilizando la mediatriz correspondiente y luego medir la distancia desde el punto medio de cualquier lado al circuncentro. Esta distancia es igual al radio de la circunferencia circunscrita.\\
\\

{\bf 25) La bisectriz de un ángulo $\pi$ pasa por los puntos D y C, y uno de los lados de $\pi$ pasa por A y B. Dibujar el ángulo $\pi$.} \\
\\
\\
Para dibujar el ángulo $\pi$, primero dibuja un segmento de línea AB para representar uno de los lados del ángulo. Luego, traza una línea desde el punto medio de AB hasta el punto C, que está en la bisectriz del ángulo. A continuación, traza una línea desde el punto C hasta el punto D, que también está en la bisectriz del ángulo. El ángulo $\pi$ se forma entre las líneas CD y CB.\\ 



{\bf 26) Busca en las distintas bibliografías: }\\
a) Definición de polígono y sus elementos.\\
b) Clasificación.\\
\\
a) Un polígono es una figura geométrica plana formada por una serie de segmentos de línea recta que se unen en diferentes puntos para formar una figura cerrada. Los elementos de un polígono son:\\
\\
1. Lados: son los segmentos de línea recta que forman las aristas del polígono.\\
\\
2. Vértices: son los puntos donde se unen dos lados del polígono.\\
\\
3. Ángulos: son las regiones que se forman en los vértices del polígono donde se encuentran dos lados.\\
\\
4. Diagonales: son los segmentos de línea recta que conectan dos vértices no adyacentes del polígono.\\
\\
5. Perímetro: es la suma de las longitudes de todos los lados del polígono.\\
\\
6. Área: es la medida de la región encerrada por el polígono, que depende del número de lados, la longitud de los lados y los ángulos entre ellos.\\

b) Clasificacion
Los polígonos se clasifican según el número de lados que tienen. Algunas de las clasificaciones más comunes son:\\
\\
- Triángulo: polígono con tres lados.\\
- Cuadrilátero: polígono con cuatro lados.\\
- Pentágono: polígono con cinco lados.\\
- Hexágono: polígono con seis lados.\\
- Heptágono: polígono con siete lados.\\
- Octágono: polígono con ocho lados.\\
- Nonágono: polígono con nueve lados.\\
- Decágono: polígono con diez lados.\\
- Polígono regular: un polígono con todos los lados y ángulos iguales.\\
- Polígono irregular: un polígono con lados y ángulos de diferentes longitudes y medidas.\\
- Polígono convexo: un polígono en el que todos los ángulos interiores son menores que 180 grados y todas las diagonales están dentro del polígono.\\
- Polígono cóncavo: un polígono en el que al menos un ángulo interior es mayor o igual a 180 grados o al menos una diagonal está fuera del polígono.\\
\\
\section*{\large\textbf{Diagonales}}
{\bf 27) Si el número de diagonales que pueden trazarse desde un vértice de un polígono es igual a la suma de los ángulos interiores dividido por 240°, ¿De qué polígono se trata? }\\
Se trata de un polígono regular de 15 lados (pentadecágono).\\
La fórmula para calcular el número de diagonales que se pueden trazar desde un vértice en un polígono regular es: $n(n-3)/2$\\
\\
donde n es el número de lados del polígono.\\
\\
La suma de los ángulos interiores de un polígono regular de n lados es:\\
\\
$(n-2) x 180\textdegree$\\
\\
Igualando las dos fórmulas y despejando n, se obtiene:\\
\\
$n(n-3)/2 = (n-2) x 180\textdegree / 240\textdegree$\\
\\
$n(n-3) = (n-2) x 3/2$\\
\\
$n^2 - 3n = (3n - 6) / 2$\\
\\
$2n^2 - 6n = 3n - 6$\\
\\
$2n^2 - 9n + 6 = 0$\\
\\
\\
Resolviendo esta ecuación cuadrática, se obtiene:\\
n = 15 (solución positiva)\\
Por lo tanto, se trata de un polígono regular de 15 lados (pentadecágono).\\



{\bf 28) Si el número de lados de hexágono se duplica, ¿Cuál será el nuevo número de diagonales? }\\
Un hexágono tiene 9 diagonales. Si el número de lados se duplica, se obtiene un dodecágono (12 lados) y el nuevo número de diagonales sería de 54 diagonales.\\\\
\\
\\
{\bf 29) a)Calcula el número de diagonales que se pueden trazar en un polígono convexo de 7 lados: }\\
\\
El número de diagonales que se pueden trazar en un polígono convexo de 7 lados se puede calcular utilizando la fórmula:\\
$n(n-3)/2$\\
\\
donde n es el número de lados del polígono. En este caso, n es igual a 7, por lo que la fórmula se convierte en:\\
$7(7-3)/2 = 14$\\
\\
Por lo tanto, se pueden trazar 14 diagonales en un polígono convexo de 7 lados.\\
\\
{\bf b) Calcula el número de diagonales que se pueden trazar en un polígono convexo de 12 lados }\\
El número de diagonales que se pueden trazar en un polígono convexo de 12 lados se puede calcular utilizando la fórmula:  n(n-3)/2  donde n es el número de lados del polígono. En este caso, n es igual a 12, por lo que la fórmula se convierte en:  12(12-3)/2 = 54  Por lo tanto, se pueden trazar 54 diagonales en un polígono convexo de 12 lados.\\
\\
\\
{\bf c) Calcula el número de diagonales que se pueden trazar en un polígono convexo de 35 lados }\\
El número de diagonales que se pueden trazar en un polígono convexo de 35 lados se puede calcular utilizando la fórmula:  n(n-3)/2  donde n es el número de lados del polígono. En este caso, n es igual a 35, por lo que la fórmula se convierte en:  35(35-3)/2 = 560  Por lo tanto, se pueden trazar 560 diagonales en un polígono convexo de 35 lados.\\
\\
{\bf d) Si por cada vértice de un polígono convexo se pueden trazar 5 diagonales, establece el número de lados del polígono}\\
El número de diagonales que se pueden trazar desde cada vértice de un polígono convexo de n lados es n-3. Por lo tanto, si por cada vértice se pueden trazar 5 diagonales, entonces n-3 = 5, lo que implica que n = 8. Por lo tanto, el polígono tiene 8 lados.\\

%% hace un titulo grande que diga movimientos del plano en latex
{\LARGE \textbf{Movimientos del Plano }} \\
{\large \textbf{translacion}} \\
{\bf 30) Construir un triángulo rectángulo isósceles ABC, recto en A, cuyos lados iguales miden 4 cm. Al mismo realizarle las siguientes traslaciones}:
{\bf a) $\vec{v}$ es equipotente con el vector AC}
Para construir el triángulo rectángulo isósceles ABC, se puede utilizar un compás para trazar un círculo de radio 4 cm y luego dibujar dos segmentos de longitud 4 cm que se intersectan en un ángulo recto en su centro. Uno de estos segmentos será la hipotenusa del triángulo y el otro será uno de los catetos. Luego, se puede trazar el tercer lado del triángulo, que será el otro cateto y\\
también tendrá una longitud de 4 cm.\\
\\
Para realizar la traslación del triángulo por el vector v, se debe mover el triángulo de manera que el vértice A se mueva al punto C. Para hacer esto, se puede dibujar una flecha que representa el vector v a partir del punto A y luego mover el triángulo de manera\\
que el vértice A coincida con el punto C y la flecha que representa el vector v coincida con el segmento AC. Los otros dos vértices\\
del triángulo se moverán a nuevos puntos que estarán a la misma distancia y dirección del vector v que sus posiciones originales.\\
\\
El resultado será un nuevo triángulo rectángulo isósceles A'B'C',\\
donde A' es el punto final de la traslación de A por el vector v,\\
B' es el punto final de la traslación de B por el vector v, y C' es el punto final de la traslación de C por el vector v. Este triángulo tendrá la misma forma y tamaño que el triángulo original, pero estará desplazado por el vector v.\\
\\

{\bf b) $\vec{v}$ es equipotente al triple del vector AC}\\
Para construir el triángulo rectángulo isósceles ABC, se puede utilizar un compás para trazar un círculo de radio 4 cm y luego\\
dibujar dos segmentos de longitud 4 cm que se intersectan en un ángulo recto en su centro. Uno de estos segmentos será la hipotenusa del triángulo y el otro será uno de los catetos. Luego, se puede trazar el tercer lado del triángulo, que será el otro cateto y también tendrá una longitud de 4 cm.\\
\\
Para realizar la traslación del triángulo por el vector v, que es equivalente al triple del vector AB, se debe primero trazar el vector AB a partir del vértice A del triángulo. Luego, se traza un vector que es tres veces más largo que AB y que tiene la misma dirección y sentido que AB. Este vector se llama v y se dibuja a partir del extremo de AB. El punto final de v será el nuevo vértice del triángulo, que llamaremos A'. Para encontrar los otros dos vértices del triángulo trasladado, se puede trazar una recta paralela a AB que pase por el punto A', y que intersecte la hipotenusa del triángulo original en el punto C'. El punto B' será el punto donde la recta paralela a AB intersecta el cateto del triángulo original que no es la hipotenusa.\\
\\
El triángulo ABC trasladado por el vector v será el triángulo rectángulo isósceles A'B'C', donde A' es el punto final de la traslación\\
de A por el vector v, B' es el punto final de la traslación de B por la recta paralela a AB, y C' es el punto final de la traslación de C por la recta paralela a AB. Este triángulo tendrá la misma forma y tamaño que el triángulo original, pero estará desplazado por el vector v.\\
\\

{\bf 30)Aplica a las figuras la composición de traslaciones indicadas, e indica si existe una única trasformación que , en cada caso las reemplaza. } \\
a) $T_{\vec{u}}$ o $T_{\vec{v}}$ o $T_{\vec{w}}$
b) $T_{\vec{v}}$ o $T_{\vec{u}}$ 
Para aplicar la composición de traslaciones indicadas, primero debes determinar el vector de traslación correspondiente a cada
una de las transformaciones. Luego, sumas los vectores de traslación para obtener el vector de traslación resultante. Por último, aplicas la transformación resultante a la figura.

En cuanto a si existe una única transformación que reemplace la composición de traslaciones, la respuesta dependerá de las características de las traslaciones involucradas. En general, si las traslaciones son paralelas entre sí, es posible reemplazarlas por una única traslación cuyo vector sea igual a la suma de los vectores de traslación individuales. Sin embargo, si las traslaciones no son paralelas, no existe una única traslación que las reemplace.
%% arreglar la notacion aca
Para aplicar una composición de traslaciones a una figura, primero se deben conocer las transformaciones de traslación que se aplicarán. Luego, se aplican secuencialmente a la figura en el orden indicado. Por ejemplo, si se tienen tres transformaciones de traslación t\_u, t\_v y t\_w, se debe aplicar primero t\_u, luego t\_v y finalmente t\_w a la figura.

Para determinar si existe una única transformación que reemplace t\_u, t\_v o t\_w, se debe verificar si la composición de las transformaciones anteriores se puede expresar como una sola transformación de traslación. Esto se puede hacer encontrando el vector de traslación
resultante de la composición de las transformaciones individuales y creando una nueva transformación de traslación con ese vector.

Es importante recordar que no siempre existe una única transformación que pueda reemplazar una composición de traslaciones. Esto depende de la naturaleza de las transformaciones individuales y de cómo se aplican a la figura.


{\large giros}

{\bf 32) Construir un romboide MNPH cuyas diagonales midan D1=6 cm y D2= 3 cm, cuya intersección es el punto O. Al mismo realizarle los siguientes giros:} \\ \\ 
a) $G_(M,90\textdegree)$ \\
Para construir un romboide MNPH con diagonales de 6 cm y 3 cm, primero dibuja la diagonal más larga D1 = 6 cm y marca el punto\\
O donde se cruzan las diagonales. Luego dibuja la diagonal más corta D2 = 3 cm desde el punto O.\\
\\
A continuación, dibuja una línea paralela a D1 desde el extremo de D2 (P) hasta que se cruce con D2 en un punto (M). Dibuja otra línea\\
paralela a D1 desde el otro extremo de D2 (N) hasta que se cruce con D2 en otro punto (H).\\
\\
Conecta los puntos M, N, P y H para formar el romboide.\\
\\
Para girar el romboide alrededor del punto M 90 grados en sentido horario, dibuja una línea perpendicular a D1 que pase por M. Luego, marca el punto donde esta línea se cruza con la diagonal más corta D2 y llámalo K.\\
\\
Conecta los puntos K, O y H para formar un triángulo. Gira este triángulo en sentido antihorario 90 grados alrededor del punto M para que la línea KO coincida con la línea MP.\\
\\
Finalmente, conecta los puntos K, M y N para formar el nuevo romboide resultante de aplicar la rotación G(M,90) al romboide original.\\
\\
\\
{\bf b) $G_(P,-70\textdegree)$ } \\
 Para construir un romboide MNPH con diagonales de 6 cm y 3 cm, primero dibuja la diagonal más larga D1 = 6 cm y marca el punto
O donde se cruzan las diagonales. Luego dibuja la diagonal más corta D2 = 3 cm desde el punto O.   A continuación, dibuja una línea paralela a D1 desde el extremo de D2 (P) hasta que se cruce con D2 en un punto (M). Dibuja otra línea paralela a D1 desde el otro extremo de D2 (N) hasta que se cruce con D2 en otro punto (H).   Conecta los puntos M, N, P y H para formar el romboide.   Para girar el romboide alrededor del punto P en sentido horario 70 grados, dibuja una línea perpendicular a D2 que pase por P y marca el punto donde esta línea se cruza con D1 y llámalo Q.   Conecta los puntos Q, O y H para formar un triángulo. Gira este triángulo en sentido horario 70 grados alrededor del punto P para que la línea QO coincida con la línea PH.   Finalmente, conecta los puntos Q, M y P para formar el nuevo romboide resultante de aplicar la rotación G(P,-70) al romboide original.

{\bf c) $G_(O,-120\textdegree)$ } \\
Para construir un romboide MNPH con diagonales de 6 cm y 3 cm, primero dibuja la diagonal más larga D1 = 6 cm y marca el punto O donde se cruzan las diagonales. Luego dibuja la diagonal más corta D2 = 3 cm desde el punto O.   A continuación, dibuja una línea paralela a D1 desde el extremo de D2 (P) hasta que se cruce con D2 en un punto (M). Dibuja otra línea paralela a D1 desde el otro extremo de D2 (N) hasta que se cruce con D2 en otro punto (H).   Conecta los puntos M, N, P y H para formar el romboide.   Para girar el romboide alrededor del punto O en sentido horario 120 grados, dibuja una línea perpendicular a D1 que pase por O y marca el punto donde esta línea se cruza con D2 y llámalo S.   Conecta los puntos S, P y H para formar un triángulo. Gira este triángulo en sentido horario 120 grados alrededor del punto O para que la línea SO coincida con la línea NH.   Finalmente, conecta los puntos S, M y N para formar el nuevo romboide resultante de aplicar la rotación G(O,-120) al romboide original.\\
\\
\\
{\bf 33) Aplica al romboide del ejercicio anterior las siguientes composiciones de giros, e indica si existe una única transformación que los reemplace: }\\
{\bf a) G(H;60) o G(H;80) }  \\
{\bf b) G(M;130) o G(M;-70) } \\
\\
Para aplicar una composición de giro a un romboide, puedes seguir los siguientes pasos:\\
\\
1. Ubica el centro de giro en el punto donde quieres que se realice el giro.\\
2. Dibuja la línea de giro, que es la recta que pasa por el centro de giro y por cualquier punto del romboide.\\
3. Gira el romboide en la dirección indicada por el ángulo de giro, manteniendo la línea de giro fija y utilizando la regla y el transportador para asegurarte de que el ángulo de giro sea el correcto.\\
\\
Para indicar si existe una única transformación que reemplace una composición de giro en un romboide, debes verificar si el resultado de la composición de giro es igual a la figura original o si puedes obtener la figura original mediante otra transformación. Si la figura original se puede obtener mediante otra transformación, entonces no existe una única transformación que reemplace la composición de giro. Por ejemplo, si aplicas una composición de giro de 60 grados y obtienes la figura original, entonces no existe otra transformación que reemplace esta composición de giro. Sin embargo, si obtienes la figura original mediante una reflexión y una traslación, entonces\\
no existe una única transformación que reemplace la composición de giro.\\
{\large simetria central}
\\
{\bf 34) Construir un triángulo isósceles EDF cuyos lados iguales midan 4 cm. Al mismo realizarle las siguientes simetrías: }\\
a) $S_F$\\
\\
Para construir un triángulo isósceles EDF con lados iguales de 4 cm, se puede seguir los siguientes pasos:\\
1. Dibujar una línea recta horizontal EF y marcar un punto en el centro de la línea como punto D.\\
2. Desde el punto D, trazar dos líneas diagonales hacia arriba y hacia abajo, formando un ángulo de 45 grados con la línea EF. Estas dos líneas serán los lados iguales del triángulo isósceles.\\
3. Medir y marcar una distancia de 4 cm en cada una de las líneas diagonales desde el punto D. Estos puntos serán los puntos E y F, respectivamente.\\
4. Unir los puntos E y F con una línea recta para completar el triángulo isósceles EDF.\\
\\
Para realizar las siguientes simetrías s\_f en el triángulo isósceles EDF:\\
- Simetría respecto a la línea EF: trazar una línea paralela a la línea EF a una distancia igual a la altura del triángulo isósceles (la mitad de la longitud de uno de los lados iguales) por encima de la línea EF. Luego, trazar una línea perpendicular a la línea EF que pase por el punto medio de la línea EF. Esta línea será la línea de simetría. Reflejar el triángulo isósceles EDF a través de esta línea para obtener el triángulo E'D'F', donde D' es el punto medio de la línea E'F'.\\
- Simetría respecto al punto D: trazar una línea recta desde el punto D hasta el punto medio del lado EF. Esta línea será la línea de simetría. Reflejar el triángulo isósceles EDF a través de esta línea para obtener el triángulo E''D''F'', donde E'' y F'' son los puntos de intersección de la línea de simetría con los lados E'D'' y F'D'', respectivamente.\\
\\ \\
b) {\bf $S_O$ donde O es el punto medio del lado EF }\\\\
Para construir el triángulo isósceles EDF con lados iguales de 4 cm y realizar la simetría $S_O$, sigue estos pasos:\\
\\
1. Dibuja una línea horizontal EF de cualquier longitud deseada.\\
2. Marca el punto medio de la línea EF y llámalo O.\\
3. Desde el punto O, traza una línea perpendicular a EF y etiquétala como OG.\\
4. Desde el punto G, traza dos líneas diagonales de 4 cm cada una hacia los puntos E y F en la línea EF.\\
5. Etiqueta los puntos de intersección de las líneas diagonales con la línea EF como D y F.\\
6. Traza los segmentos DF y DE para completar el triángulo isósceles EDF.\\
7. Haz la simetría S\_O del triángulo EDF. Para hacerlo, simplemente refleja el triángulo EDF a través de la línea OG para obtener el triángulo E'D'F'.\\
\\
El triángulo E'D'F' será la imagen reflejada del triángulo EDF a través de la simetría $S_O$.\\



{\bf 35) Aplica a la siguiente figura la composición de simetrías centrales SO2 o SO1}  \\ \\
{\bf a)  ¿La composición de simetrías centrales cumple con la ley de cierre?} \\ \\
Para aplicar la composición de simetrías centrales SO2 o SO1 a una figura, se deben seguir los siguientes pasos:\\
\\
1. Realizar la primera simetría central en el centro de simetría deseado.\\
2. Realizar la segunda simetría central en otro centro de simetría, que puede ser el mismo o uno diferente al anterior.\\
\\
La composición de simetrías centrales cumple con la ley de cierre, ya que la composición de dos simetrías centrales siempre resulta en una tercera simetría central. Es decir, si aplicamos una simetría central SO1 o SO2 a una figura y después aplicamos otra simetría central SO1 o SO2, el resultado será una tercera simetría central SO1 o SO2.\\
\\
\\
\\
\\
\\
\\
{\bf b) ¿Cuál es el movimiento que reemplazaría a la composición de simetrías indicada en el ítem a)?}
Para saber cuál es el movimiento que reemplazaría a la composición de algunas simetrías, se puede utilizar el teorema de estructura de isometrías en el plano o el teorema de composición de movimientos.\\
\\
El teorema de estructura de isometrías en el plano establece que toda isometría del plano es la composición de una traslación, una reflexión y una rotación. Por lo tanto, cualquier composición de simetrías también puede expresarse como la composición de una traslación, una reflexión y una rotación.\\
\\
El teorema de composición de movimientos establece que cualquier movimiento en el plano puede ser reemplazado por una rotación alrededor de un punto seguida de una traslación en la misma dirección. Por lo tanto, cualquier composición de simetrías también puede expresarse como una rotación seguida de una traslación.\\
\\
Por lo tanto, para determinar el movimiento que reemplazaría a la composición de algunas simetrías, se puede aplicar uno de estos teoremas y descomponer la composición de simetrías en una traslación, una reflexión y una rotación, o en una rotación seguida de una traslación.\\

{\bf c) ¿La composición de simetrías centrales es conmutativa?}
Para saber si la composición de simetrías centrales SO2 o SO1 es conmutativa en una figura, se debe comprobar si el orden de las simetrías afecta o no al resultado final.  En general, la composición de simetrías centrales SO2 o SO1 no es conmutativa, es decir, el orden de las simetrías afecta al resultado final. Esto se debe a que el efecto de la primera simetría central cambia la posición de los objetos en la figura y, por lo tanto, afecta la ubicación de los centros de simetría de la segunda simetría central.  Sin embargo, hay casos especiales en los que la composición de simetrías centrales sí es conmutativa, como cuando la figura es un círculo o cuando las simetrías centrales se realizan sobre el mismo centro de simetría.  En resumen, para saber si la composición de simetrías centrales SO2 o SO1 es conmutativa en una figura, se debe comprobar si el orden de las simetrías afecta o no al resultado final.

{\large Simetria axial}
{\bf 36) Construir un trapecio isósceles MNBG cuyos lados iguales midan 4 cm. Al mismo realizarle las siguientes simetrías:} \\
{\bf a) $S_e$ donde e es la recta paralela al lado BG exterior a 2 cm } \\ \\\ 
Para construir el trapecio isósceles MNBG con lados iguales de 4 cm, sigue los siguientes pasos:\\
\\
1. Traza una línea recta MN de 6 cm de longitud.\\
2. Traza una línea recta BG de 4 cm de longitud, paralela a MN y separada 2 cm de ella.\\
3. Traza una línea recta perpendicular a MN en su punto medio, y marca su intersección con BG como punto C.\\
4. Traza una línea recta perpendicular a BG en su punto medio, y marca su intersección con MN como punto D.\\
5. Traza una línea recta que une los puntos M y C, y otra que une los puntos N y C.\\
6. Traza una línea recta que une los puntos M y D, y otra que une los puntos N y D.\\
7. Las líneas rectas MC y ND se cruzan en el punto O, que es el punto medio del trapecio.\\
8. Traza una línea recta que une los puntos O y B, y otra que une los puntos O y G.\\
9. Las líneas rectas OB y OG son simétricas respecto a la recta e.\\
\\
De esta manera, obtendrás un trapecio isósceles MNBG con lados iguales de 4 cm, y al aplicar la simetría $S_e$ obtendrás dos imágenes simétricas respecto a la recta e.\\
\\
{\bf b) $S_e$ donde el eje es una de las diagonales.}
Para construir un trapecio isósceles MNBG con lados iguales de 4 cm y simetría respecto a una de sus diagonales, sigue los siguientes pasos:\\
\\
1. Traza una línea recta MN de 6 cm de longitud.\\
2. Traza una línea recta BG de 4 cm de longitud, paralela a MN y separada 2 cm de ella.\\
3. Traza una línea recta perpendicular a MN en su punto medio, y marca su intersección con BG como punto C.\\
4. Traza una línea recta perpendicular a BG en su punto medio, y marca su intersección con MN como punto D.\\
5. Traza una línea recta que une los puntos M y C, y otra que une los puntos N y C.\\
6. Traza una línea recta que une los puntos M y D, y otra que une los puntos N y D.\\
7. Las líneas rectas MC y ND se cruzan en el punto O, que es el punto medio del trapecio.\\
8. Traza una línea recta que une los puntos O y B, y otra que une los puntos O y G.\\
9. Traza una línea recta que une los puntos M y G, y otra que une los puntos N y B.\\
10. Las líneas rectas MG y NB se cruzan en el punto P, que es la intersección de las diagonales del trapecio.\\
11. Traza una línea recta que une los puntos P y C.\\
De esta manera, obtendrás un trapecio isósceles MNBG con lados iguales de 4 cm y simetría respecto a la diagonal que une los puntos M y N. La línea recta PC es la otra diagonal del trapecio y es eje de simetría del mismo.\\
\\


{\bf 37) a)Aplica al trapecio la composición de simetrías axiales según lo indicado. En cada caso establece si existe un único movimiento que reemplace la composición $s_e2$ o $s_e1$ donde e1 es paralelo a e2 }\\
\\ \\
Para aplicar la composición de simetrías axiales al trapecio del ejercicio anterior, se deben seguir los siguientes pasos:\\
\\
1. Tomando como referencia las simetrías axiales $s_e1$ y $s_e2$, se deben trazar las rectas paralelas e1 y e2 a una distancia de 2 cm del\\
lado BG del trapecio, tal como se indica en el enunciado.\\
\\
2. Aplicar la primera simetría axial $s_e1$, que consiste en reflejar el trapecio respecto a la recta e1. Para ello, se deben trazar las perpendiculares desde los vértices M y N a la recta e1, y marcar los puntos simétricos M' y N' respectivamente. Luego, se deben trazar las perpendiculares desde los puntos B y G a la recta e1, y marcar los puntos simétricos B' y G' respectivamente.\\
\\
3. Aplicar la segunda simetría axial $s_e2$, que consiste en reflejar el trapecio respecto a la recta e2. Para ello, se deben trazar las perpendiculares desde los vértices M' y N' a la recta e2, y marcar los puntos simétricos M'' y N'' respectivamente. Luego, se deben trazar las perpendiculares desde los puntos B' y G' a la recta e2, y marcar los puntos simétricos B'' y G'' respectivamente.\\
\\
4. Se debe observar que la composición de las dos simetrías axiales $s_e1$ y $s_e2$ genera un movimiento de rotación de 180 grados en torno al punto medio O del trapecio. Este movimiento de rotación reemplaza la composición de las simetrías axiales $s_e1$ y $s_e2$, y es único.\\
\\
En resumen, al aplicar la composición de simetrías axiales $s_e1$ y $s_e2$ al trapecio, se obtiene un movimiento de rotación de 180 grados\\
en torno al punto medio O, que reemplaza la composición de las simetrías axiales $s_e1$ y $s_e2$, y es único.\\
\\
\\
{\bf b)Aplica al trapecio la composición de simetrías axiales según lo indicado. En cada caso establece si existe un único movimiento que reemplace la composición $s_e2$ o $s_e1$  en donde e1 es $\perp$ a e2 } \\
Para aplicar la composición de simetrías axiales a un trapecio, se deben seguir los siguientes pasos:\\
\\
1. Identificar los ejes de simetría del trapecio, que son las líneas que dividen al trapecio en dos partes iguales al reflejarlo sobre\\
ellas. Estos ejes pueden ser perpendiculares o no perpendiculares entre sí.\\
2. Aplicar la primera simetría axial sobre el trapecio, reflejando el trapecio sobre el primer eje de simetría.\\
3. Aplicar la segunda simetría axial sobre el trapecio, reflejando el trapecio reflejado sobre el segundo eje de simetría.\\
4. El resultado de la composición de simetrías axiales es una figura que es simétrica consigo misma.\\
\\
En cuanto a la segunda parte de la pregunta, si existe un único movimiento que reemplace la composición $s_e2$ o $s_e1$, depende de los ejes de simetría que se estén utilizando y de cómo se estén aplicando las simetrías axiales en el trapecio. En general, es posible que haya múltiples movimientos que reemplacen la composición $s_e1$ o $s_e2$, dependiendo de la orientación y el ángulo de los ejes de simetría.\\
\\


{\bf 38)  Construir un cuadrado MNPQ y realizar las siguientes composiciones: } \\
{\bf a) G(N’;-70°) o $S_e$ donde e es la recta paralela al lado MN exterior a 3 cm. }  \\
Para construir un cuadrado MNPQ, sigue los siguientes pasos:\\
1. Dibuja un segmento de línea recta MN del tamaño que desees que tenga tu cuadrado.\\
2. Construye una línea perpendicular a MN que pase por el punto medio de MN. Esta línea será el lado del cuadrado y la llamaremos AB.\\
3. Desde el punto A, traza una línea perpendicular a AB que se intersecte con MN en el punto P.\\
4. Desde el punto B, traza una línea perpendicular a AB que se intersecte con MN en el punto Q.\\
5. Conecta los puntos P y Q para completar el cuadrado MNPQ.\\
\\
Para realizar la composición G(N’;-70°), sigue los siguientes pasos:\\
1. Dibuja una línea paralela al lado MN del cuadrado a una distancia de 3 cm. Llamaremos a esta línea e.\\
2. Marca el punto N' en la línea e, que esté a la misma distancia del punto M que el punto N.\\
3. Toma el punto N' como centro y dibuja un arco que intersecte al lado PQ del cuadrado en el punto X.\\
4. Toma el punto X como centro y dibuja un arco que intersecte al lado MN del cuadrado en el punto Y.\\
5. Traza una línea recta desde el punto N' hasta el punto Y. Esta es la imagen de la composición G(N’;-70°) del cuadrado MNPQ.\\
\\
Para realizar la composición $S_e$, sigue los siguientes pasos:\\
1. Dibuja una línea paralela al lado MN del cuadrado a una distancia de 3 cm. Llamaremos a esta línea e.\\
2. Desde el punto M, traza una línea perpendicular a la línea e y marca el punto R donde intersecta con la línea e.\\
3. Toma el punto R como centro y dibuja un arco que intersecte al lado PQ del cuadrado en el punto S.\\
4. Traza una línea recta desde el punto M hasta el punto S. Esta es la imagen de la composición $S_e$ del cuadrado MNPQ.\\

{\bf b) $S_q$ O $T_v$ donde v es la bisectriz del MQP}
Para construir un cuadrado MNPQ, sigue los siguientes pasos:

1. Dibuja una línea recta horizontal y marca un punto en el centro. Este será el punto M.
2. Desde M, dibuja una línea perpendicular hacia arriba y hacia abajo de igual longitud. Estas serán las líneas MN y MP.
3. Desde N y P, dibuja líneas perpendiculares hacia la derecha y hacia la izquierda de igual longitud. Estas serán las líneas NP y PQ.
4. Conecta los puntos N y P con una línea recta para completar el cuadrado MNPQ.

Para realizar las composiciones $S_q$ y $T_v$, sigue estos pasos:

1. Para la composición $S_q$, toma el cuadrado MNPQ y dibuja un punto Q' en el lado MP, a la misma distancia de M que Q. Luego, dibuja una línea recta desde Q' hacia el lado NQ del cuadrado. Esta línea es la imagen de NQ bajo la reflexión sobre la perpendicular a MP que pasa por Q'.
2. Para la composición $T_v$, dibuja la bisectriz del ángulo MQP. Luego, dibuja una línea recta desde el punto medio V de MQ hacia la bisectriz, y otra línea desde el punto medio S de NP hacia la bisectriz. Estas líneas se intersectan en el punto R. La composición $T_v$ es una reflexión sobre la línea que une R y el punto medio del lado NQ del cuadrado MNPQ.


{\bf c) $S_e$ o $T_v$ o G(P;90°) donde v es la diagonal MO y el eje e es la recta que pasa por la diagonal N’Q’.}

Para construir un cuadrado MNPQ, sigue los siguientes pasos:\\
1. Dibuja un segmento de línea recta MN del tamaño que desees que tenga tu cuadrado.\\
2. Construye una línea perpendicular a MN que pase por el punto medio de MN. Esta línea será el lado del cuadrado y la llamaremos AB.\\
3. Desde el punto A, traza una línea perpendicular a AB que se intersecte con MN en el punto P.\\
4. Desde el punto B, traza una línea perpendicular a AB que se intersecte con MN en el punto Q.\\
5. Conecta los puntos P y Q para completar el cuadrado MNPQ.\\
\\
Para realizar la composición G(N’;-70°), sigue los siguientes pasos:\\
1. Dibuja una línea paralela al lado MN del cuadrado a una distancia de 3 cm. Llamaremos a esta línea e.\\
2. Marca el punto N' en la línea e, que esté a la misma distancia del punto M que el punto N.\\
3. Toma el punto N' como centro y dibuja un arco que intersecte al lado PQ del cuadrado en el punto X.\\
4. Toma el punto X como centro y dibuja un arco que intersecte al lado MN del cuadrado en el punto Y.\\
5. Traza una línea recta desde el punto N' hasta el punto Y. Esta es la imagen de la composición G(N’;-70°) del cuadrado MNPQ.\\
\\
Para realizar la composición $S_e$, sigue los siguientes pasos:\\
1. Dibuja una línea paralela al lado MN del cuadrado a una distancia de 3 cm. Llamaremos a esta línea e.\\
2. Desde el punto M, traza una línea perpendicular a la línea e y marca el punto R donde intersecta con la línea e.\\
3. Toma el punto R como centro y dibuja un arco que intersecte al lado PQ del cuadrado en el punto S.\\
4. Traza una línea recta desde el punto M hasta el punto S. Esta es la imagen de la composición $S_e$ del cuadrado MNPQ.\\



\end{document}
