\documentclass{article}
\usepackage{amsmath, amssymb, amsthm}
\usepackage[margin=1.3cm]{geometry} % set margins
\usepackage{fancyhdr}
\pagestyle{fancy}



\begin{document}
\fancyhead[C]{{\large Trabajo Practico N\textdegree 2 - Geometria 1 (Metrica)}}

{\textbf 1. Para demostrar que los triángulos ABM y BCM son congruentes, debemos demostrar que tienen los mismos tres lados y que los ángulos correspondientes son iguales. }\\
\\
Dado que el triángulo ABC es isósceles, sabemos que los lados AB y BC son iguales. Además, dado que el ángulo B se opone al lado desigual, sabemos que los ángulos en los vértices A y C son iguales.\\
\\
La altura BM divide el triángulo ABC en dos triángulos congruentes, ABM y BCM. Esto se debe a que la altura BM es también la bisectriz del ángulo B, dividiendo el ángulo A y el ángulo C en dos partes iguales.\\
\\
Por lo tanto, los lados AB y BC son iguales en ambos triángulos, y los ángulos en los vértices B y M son iguales en ambos triángulos. Por lo tanto, podemos concluir que los triángulos ABM y BCM son congruentes por el criterio LAL (lado-ángulo-lado) de congruencia de triángulos.\\
\\
En resumen, si el triángulo ABC es isósceles, el ángulo B se opone al lado desigual y se dibuja la altura BM, entonces los triángulos ABM y BCM son congruentes.\\

{\textbf 2. }\\
Podemos demostrar que la diagonal de un cuadrado divide al cuadrado en dos triángulos congruentes utilizando el criterio LAL (lado-ángulo-lado) de congruencia de triángulos.\\
\\
Para demostrarlo, consideremos el cuadrado MNPQ y tracemos la diagonal MP. Esto divide el cuadrado en dos triángulos, uno con vértices en M, N y P, y otro con vértices en P, Q y M.\\
\\
Para mostrar que estos triángulos son congruentes, debemos demostrar que tienen los mismos tres lados y que los ángulos correspondientes son iguales.\\
\\
Primero, notemos que el lado MP es común a ambos triángulos. Además, dado que estamos trabajando con un cuadrado, sabemos que todos los lados son iguales, por lo que los lados MN y NP también son iguales en ambos triángulos.\\
\\
Por lo tanto, ambos triángulos tienen los mismos tres lados: MP, MN y NP.\\
\\
Ahora, notemos que el ángulo en el vértice M es un ángulo recto en ambos triángulos, ya que estamos trabajando con un cuadrado. Además, dado que la diagonal MP es una recta, los ángulos en los vértices N y Q son ángulos opuestos y, por lo tanto, iguales.\\
\\
Por lo tanto, los dos triángulos tienen dos ángulos iguales: el ángulo en el vértice M y el ángulo en el vértice opuesto a M, ya sea en N o Q.\\
\\
Finalmente, como ambos triángulos tienen los mismos tres lados y dos ángulos iguales, podemos concluir que son congruentes por el criterio LAL de congruencia de triángulos.\\
\\
Por lo tanto, la diagonal de un cuadrado divide al cuadrado en dos triángulos congruentes.\\
\\
{\textbf 3. }\\
No es cierto que las diagonales de un cuadrado lo dividan en cuatro triángulos congruentes. En realidad, las diagonales de un cuadrado dividen al cuadrado en cuatro triángulos, pero solo dos de ellos son congruentes.\\
\\
Para demostrar esto, podemos considerar el cuadrado ABCD y trazar sus diagonales AC y BD. Esto divide al cuadrado en cuatro triángulos, uno con vértices en A, B y C, otro con vértices en A, C y D, otro con vértices en B, C y D, y otro con vértices en A, B y D.\\
\\
Ahora, notemos que el triángulo ABD y el triángulo BCD son congruentes, ya que tienen dos lados iguales (los lados AD y BD, que son las diagonales del cuadrado) y un ángulo compartido en el vértice B.\\
\\
Por lo tanto, solo hay dos triángulos congruentes en la división del cuadrado por sus diagonales: el triángulo ABD y el triángulo BCD. Los otros dos triángulos, el triángulo ABC y el triángulo ACD, no son congruentes a los primeros dos triángulos.\\
\\
En conclusión, las diagonales de un cuadrado no dividen el cuadrado en cuatro triángulos congruentes, sino en cuatro triángulos, de los cuales solo dos son congruentes.\\

{\textbf 4. }\\
No podemos afirmar que los triángulos ABC y AED son congruentes solo con la información proporcionada por el dibujo de la circunferencia. Sin embargo, si se nos proporciona información adicional sobre los lados o los ángulos de los triángulos, sí podríamos demostrar la congruencia de los triángulos utilizando un criterio de congruencia de triángulos.\\
\\
Uno de los criterios de congruencia de triángulos es el criterio LAL (lado-ángulo-lado), que establece que dos triángulos son congruentes si tienen dos lados y el ángulo opuesto a uno de ellos iguales. Para demostrar que los triángulos ABC y AED son congruentes utilizando este criterio, necesitaríamos saber que tienen dos lados iguales y que el ángulo opuesto a uno de ellos también es igual.\\
\\
Otro criterio de congruencia de triángulos es el criterio LLL (lado-lado-lado), que establece que dos triángulos son congruentes si tienen los tres lados iguales. Si se nos proporciona información que nos permita demostrar que los triángulos ABC y AED tienen los tres lados iguales, podríamos utilizar este criterio para demostrar su congruencia.\\
\\
En resumen, con la información proporcionada por el dibujo de la circunferencia no podemos afirmar que los triángulos ABC y AED son congruentes. Sin embargo, si se nos proporciona información adicional sobre los lados o los ángulos de los triángulos, podríamos demostrar su congruencia utilizando un criterio de congruencia de triángulos.\\
\\
{\textbf 5. }\\
Podemos demostrar que la diagonal principal AC divide el romboide ABCD en dos triángulos congruentes utilizando el criterio LAL (lado-ángulo-lado) de congruencia de triángulos.\\\\
\\\\
Para demostrar que los triángulos ACD y ABC son congruentes, debemos demostrar que tienen los mismos tres lados y que los ángulos correspondientes son iguales.\\\\
\\\\
Primero, notemos que el lado AC es común a ambos triángulos. Además, dado que estamos trabajando con un romboide, sabemos que los lados opuestos son iguales, por lo que los lados AD y BC también son iguales en ambos triángulos.\\\\
\\\\
Por lo tanto, ambos triángulos tienen los mismos tres lados: AC, AD y BC.\\\\
\\\\
Ahora, notemos que el ángulo en el vértice C es un ángulo recto en ambos triángulos, ya que estamos trabajando con un romboide. Además, dado que la diagonal AC es una recta, los ángulos en los vértices A y B son ángulos opuestos y, por lo tanto, iguales.\\\\
\\\\
Por lo tanto, los dos triángulos tienen dos ángulos iguales: el ángulo en el vértice C y el ángulo en el vértice opuesto a C, ya sea en A o B.\\\\
\\\\
Finalmente, como ambos triángulos tienen los mismos tres lados y dos ángulos iguales, podemos concluir que son congruentes por el criterio LAL de congruencia de triángulos.\\\\
\\\\
Por lo tanto, la diagonal principal AC divide el romboide ABCD en dos triángulos congruentes.\\\\
\\
{\textbf 6. }\\\\
Podemos demostrar que las diagonales de cualquier trapecio isósceles son iguales utilizando la propiedad de que los ángulos suplementarios de un trapecio son iguales.\\
\\
Consideremos un trapecio isósceles ABCD, donde AB y CD son los lados paralelos y AD y BC son las bases no paralelas. Sea E el punto de intersección de las diagonales AC y BD, como se muestra en la figura.\\
\\
(Inserte Imagen que no fue encontrada)\\
Notemos que los ángulos suplementarios en el trapecio son:\\
\\
-  Los ángulos opuestos en los lados paralelos, AB y CD, son iguales.\\
- Los ángulos adyacentes en las bases, AD y BC, son suplementarios.\\
Dado que el trapecio es isósceles, podemos deducir que los ángulos opuestos en los lados no paralelos, como los ángulos DAB y CDA, son iguales. Además, los ángulos opuestos en los lados no paralelos también son iguales, es decir, los ángulos ABC y CDA son iguales, y los ángulos ABD y BCD son iguales.\\
\\
Ahora, notemos que los triángulos ABE y CDE son congruentes, ya que tienen dos lados y un ángulo compartido iguales (los lados AB y CD son iguales por ser un trapecio isósceles, el ángulo EAB es igual al ángulo ECD por ser opuestos en los lados paralelos, y el ángulo ABE es igual al ángulo CDE por ser suplementarios a los ángulos iguales ABC y CDA).\\
\\
Por lo tanto, los lados opuestos en estos triángulos también son iguales, es decir, AE es igual a CE.\\
\\
De manera similar, podemos demostrar que los triángulos ADE y BCE son congruentes, ya que tienen dos lados y un ángulo compartido iguales. Por lo tanto, los lados opuestos en estos triángulos también son iguales, es decir, BE es igual a DE.\\
\\
Por lo tanto, las diagonales AC y BD del trapecio isósceles ABCD son iguales, ya que se cruzan en un punto E tal que AE = CE y BE = DE.\\
\\
En resumen, las diagonales de cualquier trapecio isósceles son iguales, ya que los ángulos suplementarios del trapecio son iguales y las diagonales se cruzan en un punto que divide cada diagonal en dos segmentos iguales.\\
\\
\\

{\textbf 7. }\\
El cuadrilátero PQRS es un paralelogramo.\\
\\
Si trazamos la diagonal PR del cuadrilátero PQRS, podemos demostrar que los triángulos PRS y PRQ son congruentes por el criterio LAL de congruencia de triángulos. Esto significa que los lados PS y QR son iguales, y que los ángulos opuestos en los lados PR y RS también son iguales. Por lo tanto, los lados opuestos de PQRS son paralelos (ya que los triángulos congruentes PRS y PRQ tienen lados paralelos), y el cuadrilátero es un paralelogramo.\\
\\
De manera similar, si trazamos la diagonal QS del cuadrilátero PQRS, podemos demostrar que los triángulos PQS y RQS son congruentes por el criterio LAL de congruencia de triángulos. Esto significa que los lados PQ y RS son iguales, y que los ángulos opuestos en los lados QS y PR también son iguales. De nuevo, esto implica que los lados opuestos de PQRS son paralelos, y por lo tanto, el cuadrilátero es un paralelogramo.\\
\\
En resumen, si el cuadrilátero PQRS tiene diagonales que dividen el cuadrilátero en triángulos congruentes, entonces es un paralelogramo.\\
\\
\\
{\textbf 8. }\\
No podemos determinar los valores de x e y sin información adicional. Para poder encontrar los valores de x e y, se necesitan más detalles sobre los triángulos ABC y DEF, como la longitud de sus lados o la medida de sus ángulos.\\
\\
Dado que los triángulos ABC y DEF son congruentes y isósceles, sabemos que tienen lados y ángulos correspondientes iguales. Sin embargo, esto no es suficiente para determinar los valores de x e y sin información adicional.\\
\\
Por lo tanto, necesitamos más información sobre los triángulos para poder encontrar los valores de x e y. Si se proporciona información adicional, como la longitud de un lado o la medida de un ángulo, entonces podríamos utilizar los criterios de congruencia de triángulos (como el criterio LLL, LAL o ASA) para encontrar los valores de x e y.\\
\\
{\textbf 9. }\\

{\textbf 10. }\\
{\textbf 11. }\\
{\textbf 12. }\\
{\textbf 13. }\\
{\textbf 14. }\\
{\textbf 15. }\\
{\textbf 16. }\\
a) Para construir un triángulo con lados de 6 cm y 4 cm, se puede trazar un segmento de 6 cm y otro de 4 cm, y luego trazar un círculo con centro en el extremo de uno de los segmentos y radio igual a la longitud del otro segmento. La intersección del círculo con el otro segmento es el tercer vértice del triángulo. Hay dos posibles triángulos que se pueden construir, dependiendo de cuál sea el segmento al que se le traza el círculo.\\
\\
b) Para construir un triángulo con ángulos de 80$\textdegree$ y 60$\textdegree$, se puede trazar un ángulo de 60$\textdegree$ y luego trazar un arco con centro en uno de los lados del ángulo y que pase por el otro lado. Luego se traza un arco con centro en el punto donde se cruzan los dos arcos anteriores y que pase por uno de los extremos del primer arco. La intersección de este último arco con el segundo arco es el tercer vértice del triángulo. Hay dos posibles triángulos que se pueden construir, dependiendo de cuál sea el lado común de los dos ángulos.\\
\\
c) Para construir un triángulo con ángulos de 90$\textdegree$, 60$\textdegree$ y 70$\textdegree$, se puede trazar un ángulo de 90$\textdegree$ y luego trazar un ángulo de 60$\textdegree$ adyacente a uno de los lados del ángulo recto. Luego se traza un arco con centro en el otro extremo del lado del ángulo recto y que pase por el otro extremo del ángulo de 60$\textdegree$. A continuación, se traza un arco con centro en el punto donde se cruzan los dos arcos anteriores y que pase por el otro extremo del ángulo de 60$\textdegree$. La intersección de este último arco con el lado del ángulo recto es el tercer vértice del triángulo. Solo hay un posible triángulo que se puede construir con estos ángulos.\\
\\
d) Para construir un triángulo con lados de 3 cm, 4 cm y 7 cm, se puede trazar un segmento de 7 cm y luego trazar un arco con centro en uno de los extremos del segmento y que pase por el otro extremo. Luego se traza un arco con centro en el otro extremo del segmento y que pase por uno de los extremos del primer arco. La intersección de este último arco con el primer arco es el tercer vértice del triángulo. Solo hay un posible triángulo que se puede construir con estos lados, ya que la suma de los dos lados más cortos (3 cm y 4 cm) es menor que el lado más largo (7 cm), lo que significa que no se puede cerrar un triángulo si se intenta unir los extremos de los dos lados más cortos.\\
{\textbf 17. }\\
Como el triángulo ABC es isósceles, los ángulos $\alpha$ y $\beta$ son iguales. Entonces, tenemos:\\
\\

\begin{equation}
\begin{array}{l}
\alpha = \beta\\
\alpha = 2x + 10$\textdegree$\\
\\
\text{También sabemos que la suma de los ángulos de un triángulo es igual a 180\textdegree. Por lo tanto:}\\
\\
\alpha + \beta + \gamma = 180\textdegree\\
2\alpha + \gamma = 180$\textdegree$\\
2(2x + 10$\textdegree$) + 7x + 8$\textdegree$ = 180$\textdegree$\\
4x + 20$\textdegree$ + 7x + 8$\textdegree$ = 180$\textdegree$\\
11x + 28$\textdegree$ = 180$\textdegree$\\
11x = 152$\textdegree$\\
x = 14$\textdegree$\\
\\
\text{Sustituyendo el valor de x en las ecuaciones anteriores, podemos encontrar la medida de cada ángulo:}
\\
\alpha = 2x + 10$\textdegree$ = 2(14$\textdegree$) + 10$\textdegree$ = 38$\textdegree$\\
\beta = \alpha = 38$\textdegree$\\
\gamma = 7x + 8$\textdegree$ = 7(14$\textdegree$) + 8$\textdegree$ = 106$\textdegree$\\
\end{array}
\end{equation}
\\
Por lo tanto, los ángulos del triángulo ABC miden 38$\textdegree$, 38$\textdegree$ y 106$\textdegree$.\\
{\textbf 18. }\\
Para construir un triángulo, se debe cumplir la propiedad de que la suma de las longitudes de dos de sus lados siempre debe ser mayor que la longitud del tercer lado. Si se cumple esta propiedad para todas las combinaciones posibles de dos lados, entonces se puede construir el triángulo. En caso contrario, no es posible construir el triángulo.\\
\\
a) Para la terna de medidas de lados 5cm, 4cm y 9cm, se debe verificar si se cumple la propiedad mencionada anteriormente:\\
\\
$5cm + 4cm > 9cm$ (Sí)\\
$5cm + 9cm > 4cm$ (Sí)\\
$4cm + 9cm > 5cm$ (Sí)\\
Por lo tanto, se puede construir un triángulo con estas medidas.\\
\\
b) Para la terna de medidas de lados 3cm, 5cm y 6cm, se debe verificar si se cumple la propiedad mencionada anteriormente:\\
\\
$3cm + 5cm > 6cm$ (Sí)\\
$3cm + 6cm > 5cm$ (Sí)\\
$5cm + 6cm > 3cm$ (No)\\
Como la suma de los dos lados más largos (5cm y 6cm) es igual a la longitud del lado más corto (3cm), no se puede construir un triángulo con estas medidas.\\
\\
c) Para la terna de medidas de lados 8cm, 2cm y 10cm, se debe verificar si se cumple la propiedad mencionada anteriormente:\\
\\
$8cm + 2cm > 10cm$ (Sí)\\
$8cm + 10cm > 2cm$ (No)\\
$2cm + 10cm > 8cm$ (Sí)\\
De nuevo, la suma de los dos lados más largos (8cm y 10cm) es igual a la longitud del lado más corto (2cm), por lo que no se puede construir un triángulo con estas medidas.\\
\\
Por lo tanto, solo es posible construir un triángulo con la terna de medidas de lados 5cm, 4cm y 9cm.\\
{\textbf 19. }\\
Para construir un triángulo, se debe cumplir la propiedad de que la suma de los ángulos interiores de un triángulo es igual a 180$\textdegree$. Si se cumple esta propiedad para las expresiones dadas de los ángulos interiores, entonces se puede construir el triángulo. En cuanto a su unicidad, si se puede construir el triángulo, será único ya que cualquier otro triángulo con ángulos interiores diferentes no cumpliría con la propiedad mencionada anteriormente.\\
\\
Entonces, tenemos:\\
\\
\begin{equation}
\begin{array}{l}
\alpha + \beta + \gamma = 180$\textdegree$\\
2x + 30$\textdegree$ + x + 40$\textdegree$ + x + 50$\textdegree$ = 180$\textdegree$\\
4x + 120$\textdegree$ = 180$\textdegree$\\
4x = 60$\textdegree$\\
x = 15$\textdegree$\\
\\
\text{Sustituyendo el valor de x en las ecuaciones dadas para cada ángulo, podemos encontrar su medida:}\\
\\
\alpha = 2x + 30$\textdegree$ = 2(15$\textdegree$) + 30$\textdegree$ = 60$\textdegree$\\
\beta = x + 40$\textdegree$ = 15$\textdegree$ + 40$\textdegree$ = 55$\textdegree$\\
\gamma = x + 50$\textdegree$ = 15$\textdegree$ + 50$\textdegree$ = 65$\textdegree$\\
\end{array}
\end{equation}
\\
La suma de estos ángulos es igual a 180$\textdegree$, por lo que se cumple la propiedad mencionada anteriormente y se puede construir un triángulo con estos ángulos. Además, ya que la suma de dos de los ángulos ($\alpha$ y $\gamma$) es mayor que el tercer ángulo ($\beta$), sabemos que este triángulo es un triángulo obtusángulo.\\
\\
Por lo tanto, se puede construir un triángulo con ángulos interiores de $\alpha = 60\textdegree$, $\beta = 55\textdegree$ y $\gamma = 65\textdegree$. Este triángulo es único ya que cualquier otra combinación de ángulos interiores no cumpliría con la propiedad de que la suma de los ángulos interiores de un triángulo es igual a 180$\textdegree$.\\
{\textbf 20. }\\
La suma de las medidas de los ángulos interiores de un polígono convexo con n lados se puede calcular con la fórmula:\\
\\
$sumaAngulosInteriores = (n - 2) x 180\textdegree$\\
\\
Por otro lado, la suma de las medidas de los ángulos exteriores de un polígono convexo es siempre igual a 360$\textdegree$. Cada ángulo exterior se forma por la prolongación de uno de los lados del polígono y otro de sus ángulos interiores. Por lo tanto, la medida de cada ángulo exterior es igual a la suma de la medida del ángulo interior correspondiente y 180$\textdegree$.\\
\\
La suma de los ángulos exteriores se puede calcular de dos maneras: sumando la medida de cada ángulo exterior o utilizando la fórmula:\\
\\
$sumaAngulosExteriores = n x 360\textdegree$\\
\\
Si la suma de las medidas de los ángulos interiores de un polígono es igual a la suma de las medidas de sus ángulos exteriores, entonces se cumple la siguiente igualdad:\\
\\
$(n - 2) x 180\textdegree = n x 360\textdegree$\\
\\
Resolviendo para n, obtenemos:\\
\\
n - 2 = 2n\\
-2 = n\\
n = -2\\
\\
Esto significa que no existe un polígono con una cantidad negativa de lados. Por lo tanto, la igualdad dada no es posible y no se cumple en ningún polígono convexo. En cualquier polígono convexo, la suma de las medidas de los ángulos interiores siempre es menor que la suma de las medidas de los ángulos exteriores.\\
\\
{\textbf 21. }\\
En un paralelogramo, los lados opuestos son paralelos entre sí, lo que significa que no se cruzan y mantienen la misma dirección. Cuando dos líneas paralelas son cortadas por una tercera línea, se forman pares de ángulos correspondientes y pares de ángulos alternos.

Los ángulos correspondientes son aquellos que se encuentran en lados paralelos y del mismo lado de la línea transversal. Estos ángulos tienen la misma medida. Por otro lado, los ángulos alternos son aquellos que se encuentran en lados paralelos y del lado opuesto de la línea transversal.
{\textbf 22. }\\
{\textbf 23. }\\
{\textbf 24. }\\
Sabemos que la suma de los ángulos interiores de un polígono con n lados se puede calcular con la fórmula:\\
\\
$sumaAngulosInteriores = (n - 2) x 180\textdegree$\\
\\
Por otro lado, la suma de los ángulos exteriores de un polígono convexo es siempre igual a 360$\textdegree$. Cada ángulo exterior se forma por la prolongación de uno de los lados del polígono y otro de sus ángulos interiores. Por lo tanto, la medida de cada ángulo exterior es igual a la suma de la medida del ángulo interior correspondiente y 180$\textdegree$.\\
\\
La diferencia entre la suma de los ángulos interiores y la suma de los ángulos exteriores de un polígono con n lados se puede calcular como:\\
\\
$sumaAngulosInteriores - sumaAngulosExteriores = [(n - 2) x 180\textdegree] - [n x 180\textdegree] = (n - 2 - n) x 180\textdegree = -2 x 180\textdegree = -360\textdegree$\\
\\
Entonces, si la diferencia de la suma de los ángulos interiores menos la suma de los ángulos exteriores es 900$\textdegree$, tenemos:\\
\\
$sumaAngulosInteriores - sumaAngulosExteriores = 900\textdegree$\\
$(n - 2) x 180\textdegree - n x 180\textdegree = 900\textdegree$\\
$-2 x 180\textdegree = 900\textdegree$\\
$n = -5$\\
\\
Esto significa que no existe un polígono con una cantidad negativa de lados. Por lo tanto, no hay un polígono que cumpla la condición dada en el problema. La suma de los ángulos interiores siempre es menor que la suma de los ángulos exteriores en cualquier polígono convexo, por lo que la diferencia entre estas cantidades nunca puede ser positiva, mucho menos 900$\textdegree$.\\
{\textbf 25. }\\
Primero, podemos utilizar la relación B = 1/2 A para encontrar la medida del ángulo B:\\
\\
B = 1/2 A = 1/2 * 125$\textdegree$ = 62.5$\textdegree$\\
\\
También sabemos que E = 3/2 B, por lo que podemos encontrar la medida del ángulo E:\\
\\
E = 3/2 B = 3/2 * 62.5$\textdegree$ = 93.75$\textdegree$\\
\\
Además, se nos da la relación D = 5/3 E, por lo que podemos encontrar la medida del ángulo D:\\
\\
D = 5/3 E = 5/3 * 93.75$\textdegree$ = 156.25$\textdegree$\\
\\
Finalmente, podemos utilizar la propiedad de que la suma de los ángulos interiores de un polígono es igual a (n-2) x 180$\textdegree$, donde n es el número de lados del polígono, para encontrar la medida del ángulo C:\\
\\
A + B + C + D + E = (n-2) x 180$\textdegree$\\
\\
Reemplazando las medidas que conocemos, tenemos:\\
\\
125$\textdegree$ + 62.5$\textdegree$ + C + 156.25$\textdegree$ + 93.75$\textdegree$ = (5-2) x 180$\textdegree$\\
\\
Resolviendo la ecuación, obtenemos:\\
\\
C = (3 x 180$\textdegree$ - 125$\textdegree$ - 62.5$\textdegree$ - 156.25$\textdegree$ - 93.75$\textdegree$) = 42.5$\textdegree$\\
\\
Por lo tanto, la medida del ángulo C es de 42.5$\textdegree$.\\
{\textbf 26. }\\
La suma de los ángulos interiores de un polígono de 5 lados se puede calcular utilizando la fórmula:\\
\\
$sumaAngulosInteriores = (n-2) x 180\textdegree$\\
\\
donde n es el número de lados del polígono. Para un polígono de 5 lados, n = 5, por lo que:\\
\\
$sumaAngulosInteriores = (5-2) x 180\textdegree = 540\textdegree$\\
\\
Podemos utilizar esta información para encontrar el valor de x y, a partir de ahí, calcular la medida de cada ángulo.\\
\\
A + B + C + D + E = 540$\textdegree$\\
\\
(x + 45$\textdegree$) + (2x - 40$\textdegree$) + (3x - 70$\textdegree$) + (2x + 25$\textdegree$) + (x + 85$\textdegree$) = 540$\textdegree$\\
\\
8x + 45$\textdegree$ = 540$\textdegree$\\
\\
8x = 495$\textdegree$\\
\\
x = 61.875$\textdegree$\\
\\
Ahora podemos calcular la medida de cada ángulo:\\
\\
A = x + 45$\textdegree$ = 61.875$\textdegree$ + 45$\textdegree$ = 106.875$\textdegree$\\
B = 2x - 40$\textdegree$ = 2(61.875$\textdegree$) - 40$\textdegree$ = 83.75$\textdegree$\\
C = 3x - 70$\textdegree$ = 3(61.875$\textdegree$) - 70$\textdegree$ = 117.375$\textdegree$\\
D = 2x + 25$\textdegree$ = 2(61.875$\textdegree$) + 25$\textdegree$ = 148.75$\textdegree$\\
E = x + 85$\textdegree$ = 61.875$\textdegree$ + 85$\textdegree$ = 146.875$\textdegree$\\
\\
Por lo tanto, la medida de cada ángulo interior del polígono ABCDE es:\\
\\
A = 106.875$\textdegree$\\
B = 83.75$\textdegree$\\
C = 117.375$\textdegree$\\
D = 148.75$\textdegree$\\
E = 146.875$\textdegree$\\
\\
{\textbf 27. }\\
Sí, es posible dibujar un polígono regular de 7 lados con un ángulo exterior de 152$\textdegree$.\\
\\
En un polígono regular, todos los ángulos interiores y exteriores tienen la misma medida. Para un polígono regular de n lados, la medida de cada ángulo exterior se puede calcular utilizando la fórmula:\\
\\
$medida_angulo_exterior = 360\textdegree / n$\\
\\
En el caso de un polígono regular de 7 lados, la medida de cada ángulo exterior es:\\
\\
$medida_angulo_exterior = 360\textdegree / 7 = 51.43\textdegree$\\
\\
Ahora bien, la medida de un ángulo interior de un polígono regular de n lados se puede calcular como:\\
\\
$medida_angulo_interior = (n - 2) x 180\textdegree$ / n\\
\\
En el caso de un polígono regular de 7 lados, la medida de cada ángulo interior es:\\
\\
$medida_angulo_interior = (7 - 2) x 180\textdegree / 7 = 128.57\textdegree$\\
\\
La suma de los ángulos interiores de un polígono de 7 lados es:\\
\\
$sumaAngulosInteriores = (7 - 2) x 180\textdegree = 900\textdegree$\\
\\
La suma de los ángulos exteriores de un polígono de 7 lados es:\\
\\
$sumaAngulosExteriores = 360\textdegree$\\
\\
Entonces, podemos calcular la medida del ángulo que falta para completar los 360$\textdegree$ de la suma de los ángulos exteriores:\\
\\
$360\textdegree - 152\textdegree = 208\textdegree$\\
\\
Como todos los ángulos exteriores de un polígono regular tienen la misma medida, podemos encontrar la cantidad de ángulos exteriores que faltan para completar los 360$\textdegree$ dividiendo 208$\textdegree$ entre la medida de un ángulo exterior:\\
\\
$cantidad_angulos_exteriores_faltantes = 208\textdegree / 51.43\textdegree = 4.04$\\
\\
Esto significa que faltan aproximadamente 4 ángulos exteriores para completar los 360$\textdegree$. Como el polígono es regular, debe tener 7 ángulos exteriores iguales, por lo que ya tenemos 3 ángulos exteriores de 152$\textdegree$ cada uno. Para completar los 7 ángulos exteriores, se pueden dibujar 4 ángulos exteriores de medida ligeramente distinta a 152$\textdegree$. Esto se puede hacer sin afectar la regularidad del polígono.\\
\\
Por lo tanto, sí es posible dibujar un polígono regular de 7 lados con un ángulo exterior de 152$\textdegree$.\\
{\textbf 28. }\\
a) Un ejemplo de cuadrilátero que cumple con las condiciones es un trapecio isósceles, donde los lados no paralelos tienen la misma longitud y los lados opuestos son paralelos. Los lados paralelos del trapecio cortan a los segmentos a y b en los puntos A, B, C y D, como se muestra en la siguiente figura:\\
\\
A ------- B\\
|         |\\
D ------- C\\
b) No es posible dibujar más de un cuadrilátero que cumpla exactamente con las mismas condiciones que se describen, porque los segmentos a y b fijan la posición de los lados paralelos del cuadrilátero y la longitud de los lados no paralelos debe ser la misma para que los lados opuestos sean paralelos. Sin embargo, se pueden dibujar diferentes cuadriláteros que cumplan con condiciones similares pero no idénticas, dependiendo de la longitud de los lados y los ángulos que se elijan. Por ejemplo, se podrían dibujar otros trapecios isósceles con lados no paralelos de diferente longitud o con ángulos diferentes entre los lados paralelos.\\
\\
\\
\\
{\textbf 29. }\\
{\textbf 30. }\\
a) No es cierto que si un paralelogramo tiene dos diagonales iguales, entonces es un cuadrado. Un paralelogramo con diagonales iguales se llama diagonal o romboide, y aunque tiene lados opuestos paralelos, no necesariamente tiene ángulos rectos o lados congruentes como en un cuadrado.\\
\\
b) Tampoco es cierto que si un paralelogramo tiene dos diagonales que forman ángulos rectos, entonces es un rombo. Un paralelogramo con diagonales perpendiculares se llama diagonal ortodiagonal, y aunque tiene diagonales congruentes, no necesariamente tiene lados congruentes como en un rombo. Un ejemplo de esto es un rectángulo, que es un tipo de paralelogramo con diagonales perpendiculares pero lados no congruentes, y por lo tanto no es un rombo.\\
\\
{\textbf 31. }\\
Sí, es posible construir un cuadrado utilizando solamente una regla no graduada y un compás, siempre y cuando se conozca la longitud de la diagonal a. A continuación, se describen los pasos para construir un cuadrado a partir de su diagonal:\\
\\
Dibujar la diagonal a utilizando la regla no graduada.\\
\\
Con el compás, trazar dos arcos de circunferencia con centro en los extremos de la diagonal a. La longitud del radio de cada arco debe ser igual a la mitad de la longitud de la diagonal a.\\
\\
Donde los arcos se cortan, trazar una línea perpendicular a la diagonal a utilizando la regla no graduada. Esta línea divide la diagonal en dos partes iguales y forma un ángulo recto con ella.\\
\\
Desde uno de los extremos de la diagonal a, trazar una línea perpendicular a la línea que se acaba de dibujar en el paso 3. Esta línea tiene la longitud de la mitad de la diagonal a.\\
\\
Repetir el paso 4 desde el otro extremo de la diagonal a.\\
\\
Donde las dos líneas perpendiculares se cruzan, se encuentra el vértice del cuadrado. Trazar las otras tres líneas para completar el cuadrado.\\
\\
Es importante destacar que, dado que solo se conoce la longitud de la diagonal a, solo se puede construir un cuadrado con esa diagonal específica. No es posible construir más de un cuadrado con la misma diagonal a menos que se permita variar la posición de los vértices del cuadrado.\\
\\
{\textbf 32. }\\
{\textbf 33. }\\
No se puede estar seguro de que un paralelogramo con ángulos opuestos iguales sea un rectángulo. Aunque es cierto que en un rectángulo los cuatro ángulos son iguales a 90 grados y, por lo tanto, los ángulos opuestos son iguales, la igualdad de los ángulos opuestos no es una condición suficiente para garantizar que el paralelogramo sea un rectángulo.\\
\\
De hecho, existe otro tipo de paralelogramo que tiene ángulos opuestos iguales, pero que no es un rectángulo. Este paralelogramo se llama rombo, y tiene lados congruentes en lugar de ángulos rectos. En un rombo, los ángulos opuestos son iguales, pero no necesariamente tienen una medida de 90 grados. Por lo tanto, la igualdad de los ángulos opuestos no es suficiente para determinar si un paralelogramo es un rectángulo o un rombo. Se necesitan más información o medidas de los ángulos o lados del paralelogramo para hacer una determinación concluyente.\\
\\

\end{document}


