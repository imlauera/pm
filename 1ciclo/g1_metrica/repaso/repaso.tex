\documentclass{article}
\usepackage[margin=1.3cm]{geometry}
\usepackage{times}
\usepackage{fancyhdr}
\pagestyle{fancy}

\begin{document}
Aquí tienes 20 preguntas de evaluación sobre el documento con sus respuestas:\\
\\
1. ¿Qué magnitudes se comparan para determinar la semejanza de dos figuras? Respuesta: Sus lados y ángulos.\\
\\
2. ¿Cuál es la condición necesaria para que dos triángulos sean semejantes? Respuesta: Que tengan dos ángulos iguales. \\
\\
3. Si dos triángulos tienen un lado proporcional y dos ángulos iguales, ¿son semejantes? Respuesta: Sí.\\
\\
4. ¿Qué triángulos siempre son semejantes? Respuesta: Los triángulos equiláteros. \\
\\
5. ¿Qué dice el teorema de Tales sobre triángulos semejantes? Respuesta: Que dos triángulos son semejantes si tienen un ángulo igual y los lados que lo forman son proporcionales.\\
\\
6. Si dos triángulos tienen dos ángulos iguales, ¿por qué son semejantes? Respuesta: Porque al tener dos ángulos iguales, el tercer ángulo también es igual y por ello sus lados son proporcionales. \\
\\
7. ¿Qué razón debe haber entre dos lados correspondientes de dos triángulos semejantes? Respuesta: La misma razón que hay entre los otros pares de lados correspondientes.\\
\\
8. Si un triángulo tiene lados de 6, 8 y 10 cm, y otro tiene lados de 12, x e y cm, y ambos son semejantes. ¿Cuánto miden x e y? Respuesta: x = 16 cm e y = 20 cm.\\
\\
9. Las razones de semejanza entre dos triángulos semejantes ¿dependen o no de las unidades de medida empleadas? Respuesta: No dependen de las unidades de medida.\\
\\
10. Si dos triángulos tienen la razón de semejanza 3:2 entre dos lados correspondientes. ¿Cuál es la razón entre los perímetros? Respuesta: También 3:2.\\
\\
11. ¿Qué ocurre con las áreas de dos triángulos semejantes? Respuesta: Son proporcionales a los cuadrados de sus dimensiones lineales. \\
\\
12. Si un triángulo tiene de base y altura 12 y 6 cm respectivamente y otro triángulo semejante tiene de base 18 cm, ¿cuál es su altura? Respuesta: 9 cm.\\
\\
13. ¿Qué teorema relaciona a dos triángulos rectángulos semejantes? Respuesta: El teorema de Pitágoras. \\
\\
14. En un triángulo rectángulo, ¿qué segmentos determinan la semejanza? Respuesta: Los catetos.\\
\\
15. ¿Qué propiedad tiene todo triángulo rectángulo? Respuesta: Tener un ángulo recto.\\
\\
16. ¿Qué es la medida de un cateto de un triángulo rectángulo? Respuesta: La longitud de uno de los lados que forman el ángulo recto. \\
\\
17. ¿Qué es la altura de un triángulo rectángulo? Respuesta: Es el segmento perpendicular trazado desde el vértice del ángulo recto hasta la hipotenusa. \\
\\
18. ¿Qué teorema relaciona los catetos de un triángulo rectángulo con la hipotenusa? Respuesta: El teorema de Pitágoras.\\
\\
19. Si un triángulo rectángulo tiene de catetos 12 y 5 cm, ¿cuál es la medida de la hipotenusa? Respuesta: 13 cm.\\
\\
20. ¿Qué es mayor, la suma de los catetos o la hipotenusa de un triángulo rectángulo? Respuesta: La hipotenusa.\\
\section{Otra serie de preguntas con multiple choice}

Aquí tienes 20 preguntas de evaluación de opción múltiple sobre el documento con sus respuestas:\\
\\
1. ¿Qué determina la semejanza de dos figuras? \\
A) Sus ángulos  B) Sus lados   C) Sus ángulos y lados  Respuesta: C\\
\\
2. ¿Qué condición se necesita para que dos triángulos sean semejantes?\\
A) Tener dos ángulos iguales   B) Tener dos lados proporcionales    C) Tener un lado igual y dos ángulos iguales   Respuesta: A \\
\\
3. Si dos triángulos tienen un lado proporcional y dos ángulos iguales, ¿son semejantes?\\
A) Sí   B) No     C) Tal vez   Respuesta: A\\
\\
4. ¿Qué triángulos son siempre semejantes?\\
A) Escalenos  B) Equiláteros   C) Isósceles   Respuesta: B\\
\\
5. Según el teorema de Tales, ¿cuándo son semejantes dos triángulos?  \\
A) Cuando tienen dos ángulos iguales   B) Cuando tienen un ángulo igual y lados proporcionales  C) Nunca son semejantes    Respuesta: B\\
\\
6. Si dos triángulos tienen dos ángulos iguales, ¿por qué son semejantes?\\
A) Porque tienen un lado igual   B) Porque tienen dos lados proporcionales   C) Porque el tercer ángulo también es igual y así sus lados son proporcionales   Respuesta: C\\
\\
7. ¿Qué razón deben tener dos lados correspondientes de dos triángulos semejantes?\\
A) Distinta razón   B) La misma razón que entre los otros pares de lados   C) Una razón de 1:1   Respuesta: B\\
\\
8. Si un triángulo tiene de lados 6, 8 y 10 cm, y otro triángulo tiene de lados 12, x e y cm, donde x e y son directamente proporcionales a los lados del primer triángulo, ¿cuánto miden x e y?  \\
A) x = 9 cm e y = 12 cm   B) x = 16 cm e y = 20 cm    C) x = 12 cm e y = 15 cm   Respuesta: B \\
\\
9. ¿Dependen o no dependen las razones de semejanza de dos triángulos semejantes de las unidades de medida empleadas? \\
A) Sí dependen    B) No dependen   C) Dependen sólo a veces    Respuesta: B\\
\\
10. Si dos triángulos tienen razón de semejanza 3:2 entre dos lados correspondientes, ¿qué razón hay entre sus perímetros? \\
A) 3:2   B) 9:4    C) 2:3   Respuesta: A\\
\\
11. ¿Qué ocurre con las áreas de dos triángulos semejantes? \\
A) Son iguales    B) Son proporcionales a sus dimensiones lineales   C) No tienen relación    Respuesta: B\\
\\
12. Si un triángulo tiene de base y altura 12 y 6 cm, y otro triángulo semejante tiene de base 18 cm, ¿cuál es su altura?\\
A) 3 cm   B) 9 cm    C) 12 cm   Respuesta: B \\
\\
13. ¿Qué teorema relaciona a dos triángulos rectángulos semejantes? \\
A) Teorema de Tales   B) Teorema de Pitágoras    C) Teorema de la altura    Respuesta: B\\
\\
14. En un triángulo rectángulo, ¿qué segmentos determinan la semejanza?  \\
A) La hipotenusa y un cateto   B) Los catetos   C) Ningún segmento la determina   Respuesta: B\\
\\
15. ¿Qué propiedad tiene todo triángulo rectángulo?  \\
A) Tener dos ángulos obtusos    B) Tener un ángulo recto   C) Ninguna propiedad     Respuesta: B  \\
\\
16. ¿Qué son los catetos de un triángulo rectángulo?\\
A) Sus lados oblicuos   B) Sus ángulos agudos  C) Sus lados que forman el ángulo recto   Respuesta: C  \\
\\
17. ¿Qué es la altura de un triángulo rectángulo? \\
A) Es el segmento perpendicular trazado desde el vértice del ángulo recto hasta la hipotenusa.   B) Es uno de sus lados.   C) Es uno de sus ángulos.    Respuesta: A\\
\\
18. ¿Qué teorema relaciona los catetos de un triángulo rectángulo con su hipotenusa?\\
A) Teorema de Pitágoras   B) Teorema de Tales   C) Teorema de la altura   Respuesta: A\\
\\
19. Si un triángulo rectángulo tiene de catetos 12 y 5 cm, ¿cuál es la medida de su hipotenusa?\\
A) 10 cm   B) 13 cm    C) 15 cm   Respuesta: B  \\
\\
20. ¿Qué es mayor, la suma de los catetos o la hipotenusa de un triángulo rectángulo?\\
A) La suma de los catetos.   B) La hipotenusa.   C) Son iguales.   Respuesta: B\\
Aquí tienen más preguntas de opción múltiple:\\
\\
21. Si la razón de semejanza entre dos triángulos es 5:3 y un triángulo tiene 20 cm de perímetro, ¿cuál es el perímetro del otro triángulo?\\
A) 12 cm   B) 30 cm   C) 36 cm   Respuesta: C\\
\\
22. Si el área de un triángulo es de 24 cm2 y su base mide 6 cm, ¿cuál es la altura del triángulo? \\
A) 2 cm   B) 3 cm   C) 4 cm   Respuesta: C\\
\\
23. Si dos triángulos rectángulos son semejantes, ¿qué relación hay entre sus hipotenusas?\\
A) Son iguales   B) Son proporcionales a los catetos   C) Son proporcionales al cuadrado de la razón de semejanza   Respuesta: C \\
\\
24. ¿Cuál es la razón de semejanza si dos triángulos rectángulos tienen una hipotenusa de 12 cm y la otra de 15 cm? \\
A) 3:4   B) 4:3   C) 5:3   Respuesta: B\\
\\
25. En dos triángulos rectángulos semejantes, si un cateto de uno mide 6 cm y el cateto correspondiente del otro mide 15 cm,¿cuánto mide el otro cateto del primer triángulo?\\
A) 10 cm   B) 12 cm   C) 9 cm   Respuesta: C\\
\\
26. El perímetro de un triángulo isósceles es de 30 cm y cada uno de sus lados iguales mide 10 cm. ¿Cuánto mide su base?\\
A) 5 cm   B) 10 cm   C) 15 cm   Respuesta: A \\
\\
27. Si cada ángulo de la base de un triángulo isósceles mide 40°, ¿cuánto mide cada ángulo de la base?  \\
A) 70°   B) 50°   C) 60°   Respuesta: C\\
\\
28. La base de un triángulo isósceles mide 15 cm y sus lados iguales miden cada uno 12 cm. ¿Cuánto mide su altura?\\
A) 5 cm   B) 9 cm   C) 6 cm   Respuesta: C\\
\\
29. Un triángulo tiene dos ángulos que miden 35° y 55°. ¿Cuánto mide su tercer ángulo?\\
A) 90°   B) 110°   C) 120°   Respuesta: A  \\
\\
30. La suma de dos ángulos de un triángulo escaleno es 125°. ¿Cuánto puede medir el tercer ángulo? \\
A) 35°   B) 55°   C) 50°   Respuesta: C\\
\\
¿Quieres que siga formulando más preguntas de opción múltiple? Dímelo y con gusto lo haré.\\
\end{document}
